
%==========================================================================
\chapter{The whole ID Station}
\label{sec:Whole}
%==========================================================================

\section{Startup}


When the TIS is powered up it needs to establish whether it is
enrolled or not. This is formally described by
\[
        TISStartUpC
\]

\section{The main loop}


\begin{traceunit}{FD.TIS.TISMainLoop}
\traceto{FS.TIS.TISMainLoop}
\end{traceunit}



The TIS achieves its function by repeatedly performing a number of 
activities within a main loop.

The main loop is broken down into several phases:

\begin{itemize}
\item   
{\em Poll} - Polling reads the simple real world entities
(door, time)
and the reads the presence or absence of the complex entities
(user token reader, admin token reader, fingerprint reader, floppy).
\item
{\em Early Updates} - Critical updates of the door latch and alarm are
performed as soon as new polled data is available.
\item
{\em TIS processing} - TIS processing is the activity performed by
TIS, this is influenced by the current $status$ of TIS and the
recently read inputs.
\item
{\em Updates} - Critical updates of the door latch and alarm are
repeated once the processing is complete to ensure any internal state
changes result in the latch and alarm being set correctly. Less critical updates of the screen and display
are also performed once the processing is complete.
\end{itemize}

The the TIS processing depends on the current internal $status$. 

Initially the only activity that can be performed is enrolment,
formally captured as
$TISEnrol$.

When it is in a quiescent state it can start a number of activities. These
are started by either reading a user token, an adminstrator token or
keyboard data. In addition an administrator may logoff.

If the conditions for performing activities are not satisfied then the
system is idle.

\begin{traceunit}{FD.TIS.Idle}
\traceto{FS.Enclave.WaitingAdminTokenRemoval}
\end{traceunit}


\begin{schema}{TISIdleC}
        \Xi IDStationC
\where
        \lnot EnrolmentIsInProgress
\\      \lnot AdminMustLogout
\\      \lnot CurrentUserEntryActivityPossible
\\      \lnot UserEntryCanStart
\\      \lnot CurrentAdminActivityPossible
\\      \lnot AdminLogonCanStart
\\      \lnot AdminOpCanStart        
\end{schema}

If the administrator is logged on and conditions change such that the
administrator should be logged off, either token removal or token
expiry, then the short lived administrator logoff activity is
performed, even during a user entry.

Once a user token has been presented to TIS the only activities
that can be performed are stages in the multi-phase user entry
authentication operation, formally captured as $TISProgressUserEntry$. 

Once an administrator token has been presented to TIS the
administrator is logged onto the ID Station, formally captured as
$TISProgressAdminLogon$. Having logged the administrator on TIS returns to a
$quiescent$ state waiting for the administrator to perform an
operation, without preventing user entry.

Once an operation request has been made by a logged on administrator
TIS performs the, potentially multi-phase, administrator operation,
formally captured as $TISAdminOpC$ captured below:

\begin{zed}
        TISAdminOpC \defs TISOverrideDoorLockOpC  \lor TISShutdownOpC 
\\      \t4     \lor TISUpdateConfigDataOpC \lor TISArchiveLogOpC
\end{zed}

The various possible activities with conditions that ensure the
desired priority of handling are given below.
\begin{zed}
        TISDoEnrolOp \defs EnrolmentIsInProgress \land TISEnrolOpC
\also
        TISDoAdminLogout \defs \lnot EnrolmentIsInProgress \land
        AdminMustLogout \land TISAdminLogoutC
\also
        TISDoProgressUserEntry \defs \lnot EnrolmentIsInProgress
        \land \lnot AdminMustLogout 
\\ \t2  \land
        CurrentUserEntryActivityPossible \land TISProgressUserEntry
\also
        TISDoProgressAdminActivity \defs \lnot EnrolmentIsInProgress
        \land \lnot AdminMustLogout 
\\ \t2  \land
        \lnot CurrentUserEntryActivityPossible \land 
\\ \t2  CurrentAdminActivityPossible \land (TISProgressAdminLogon \lor TISAdminOpC)
\also
        TISDoStartUserEntry \defs \lnot EnrolmentIsInProgress
        \land \lnot AdminMustLogout 
\\ \t2  \land
        \lnot CurrentUserEntryActivityPossible \land 
        \lnot CurrentAdminActivityPossible 
\\ \t2  \land UserEntryCanStart \land TISStartUserEntry
\also
        TISDoStartAdminActivity \defs \lnot EnrolmentIsInProgress
        \land \lnot AdminMustLogout 
\\ \t2  \land \lnot CurrentUserEntryActivityPossible \land 
        \lnot CurrentAdminActivityPossible 
\\ \t2  \land \lnot UserEntryCanStart 
\\ \t2  \land ( TISStartAdminLogonC  \lor TISStartAdminOpC )

\end{zed}

The TIS processing activity is described by the following:

\begin{zed}
        TISProcessingC \defs  
\\ \t2	     TISDoEnrolOp 
\\ \t2  \lor TISDoAdminLogout
\\ \t2  \lor TISDoProgressUserEntry
\\ \t2  \lor TISDoProgressAdminActivity
\\ \t2  \lor TISDoStartUserEntry
\\ \t2  \lor TISDoStartAdminActivity 
\\ \t2  \lor TISIdleC
\end{zed}
