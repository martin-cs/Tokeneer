
%=========================================================================
\chapter{Operations interfacing to the ID Station}
\label{sec:Interfaces}
%=========================================================================

%-------------------------------------------------------------------------
\section{Real World Changes}
%-------------------------------------------------------------------------
The monitored components of the real world can change at any time.
The only assumption we make of the real world is that time increases.

\begin{schema}{RealWorldTimeChanges}
        nowC, nowC' :TIME
\where
        nowC' \geq nowC
\end{schema}

\begin{zed}
        RealWorldChangesC \defs RealWorldTimeChanges \land \Delta RealWorldC
\end{zed}

\begin{schema}{RealWorldChanges}
        \Delta RealWorld
\where
        now' \geq now
\end{schema}

%-------------------------------------------------------------------------
\section{Obtaining inputs from the real world}
%-------------------------------------------------------------------------

Most data is polled from the real world on a periodic basis. 
Some items are however only read when the system is in a state to
receive data. This includes reading the contents of Tokens, the floppy
disk and the keyboard.


%.................................
\subsection{Polling the real world}
%.................................

\begin{traceunit}{FD.Interface.TISPoll}
\traceto{FS.Interface.TISPoll}
\end{traceunit}

We poll all of the real world entities. For those entities that take
time to read we simply check for the presence of data. 

\begin{traceunit}{FD.Interface.PollTime}
\traceto{FD.Interface.TISPoll}
\end{traceunit}

\begin{schema}{PollTimeC}
	\Delta DoorLatchAlarmC
\\       RealWorldC
\where
	currentTimeC' = nowC
\end{schema}

\begin{traceunit}{FD.Interface.PollDoor}
\traceto{FD.Interface.TISPoll}
\end{traceunit}

When polling the door, we do not change the alarm timeout or latch
timeout. 

\begin{schema}{PollDoorC}
	\Delta DoorLatchAlarmC
\\	RealWorldC
\where
	currentDoorC' = doorC
\\	latchTimeoutC' = latchTimeoutC
\\	alarmTimeoutC' = alarmTimeoutC
\\      doorAlarmC' = doorAlarmC
\\      currentLatchC' = currentLatchC
\end{schema}

The internal representation of the latch or the alarm may need to be
updated as a result of changes to the attributes that influence their
values. 

\begin{zed}
        PollTimeAndDoor \defs (PollTimeC \land PollDoorC) 
       \semi UpdateInternalLatch \semi UpdateInternalAlarm
\end{zed}

The system only polls for the presence of the tokens, finger, floppy and
keyboard data. This is a
refinement from the Formal Specification \cite{FS}, which made the
assumption that all inputs could be read sufficiently fast to perform
the read regularly, see Section \ref{sec:DemandReading}. These entities are either required
infrequently or are time consuming to read so within the design they are only read when
data is present and the system requires the information.

\begin{traceunit}{FD.Interface.PollUserToken}
\traceto{FD.Interface.TISPoll}
\end{traceunit}

\begin{schema}{PollUserTokenC}
        \Delta UserTokenC
\\	RealWorldC
\where
	userTokenPresenceC' = present \iff userTokenC \neq noTC
\\	currentUserTokenC' =   currentUserTokenC
\end{schema}

\begin{traceunit}{FD.Interface.PollAdminToken}
\traceto{FD.Interface.TISPoll}
\end{traceunit}

\begin{schema}{PollAdminTokenC}
	\Delta AdminTokenC
\\      RealWorldC
\where
	adminTokenPresenceC' = present \iff adminTokenC \neq noTC
\\	currentAdminTokenC' =  currentAdminTokenC
\end{schema}

\begin{traceunit}{FD.Interface.PollFinger}
\traceto{FD.Interface.TISPoll}
\end{traceunit}

\begin{schema}{PollFingerC}
	\Delta FingerC
\\	RealWorldC
\where
	fingerPresenceC' = present \iff fingerC \neq noFP
\end{schema}

\begin{traceunit}{FD.Interface.PollFloppy}
\traceto{FD.Interface.TISPoll}
\end{traceunit}

\begin{schema}{PollFloppyC}
        \Delta FloppyC
\\      RealWorldC
\where
	floppyPresenceC' = present \iff floppyC \neq noFloppyC
\\      currentFloppyC' =  currentFloppyC
\\      writtenFloppyC' = writtenFloppyC
\end{schema}

\begin{traceunit}{FD.Interface.PollKeyboard}
\traceto{FD.Interface.TISPoll}
\end{traceunit}

\begin{schema}{PollKeyboardC}
        \Delta KeyboardC
\\      RealWorldC
\where
        keyedDataPresenceC = present \iff keyboardC \neq noKB
\end{schema}

So the overall poll operation is obtained by combining all the
individual polling actions.

\begin{traceunit}{FD.Interface.DisplayPollUpdate}
\traceto{FD.Interface.TISPoll}
\end{traceunit}

If the user is
currently being invited to enter the enclave on the display and the 
door becomes latched then the display will change to indicate that the
system is no longer offering entry.

\begin{schema}{DisplayPollUpdate}
        \Delta IDStationC
\where
        currentLatchC' = locked
\\ \t1  \land (currentDisplayC = doorUnlocked 
\\ \t2  \land        (statusC \neq waitingRemoveTokenFail \land
        currentDisplayC' = welcome
\\ \t3  \lor statusC = waitingRemoveTokenFail \land currentDisplayC' =
removeToken)
\\ \t1  ~~~~\lor currentDisplayC \neq doorUnlocked 
\\ \t2  \land currentDisplayC' = currentDisplayC)
\\      \lor
        currentLatchC' \neq locked 
\\ \t1 \land currentDisplayC' = currentDisplayC
\end{schema}

We assume that while polling occurs the $RealWorld$ does not
change. This is a reasonable assumption since all information polled
is easy and quick to obtain.

\begin{schema}{PollC}
	\Delta IDStationC
\\      \Xi RealWorldC
\also
	PollTimeAndDoor
\\	PollUserTokenC
\\	PollAdminTokenC
\\	PollFingerC
\\      PollFloppyC
\\      PollKeyboardC
\\      DisplayPollUpdate
\also
        \Xi ConfigC
\\      \Xi KeyStoreC
\\      \Xi CertificateStore
\\      \Xi AdminC
\\      \Xi StatsC
\\      \Xi InternalC
\where
        currentScreenC' = currentScreenC
\end{schema}

Polling the real world may result in changes which need to be audited.
The only events that will appear in the audit log during polling are 
the user independent elements.

\begin{zed}
        TISPollC \defs PollC \land LogChangeC 
\\ \t2 \land
        [ AddElementsToLogC | auditTypes~ newElements? \subseteq
        USER\_INDEPENDENT\_ELEMENTS ] \hide ( newElements? )
\end{zed}

%..........................
\subsection{Reading Real World Values}
%..........................

Those entities that are read on demand are the tokens and floppy.

\begin{schema}{ReadUserTokenC}
        \Delta UserTokenC
\\	RealWorldC
\where
	userTokenPresenceC' = userTokenPresenceC
\\	currentUserTokenC' = userTokenC
\end{schema}

\begin{schema}{ReadAdminTokenC}
	\Delta AdminTokenC
\\      RealWorldC
\where
	adminTokenPresenceC' = adminTokenPresenceC
\\	currentAdminTokenC' = adminTokenC
\end{schema}

\begin{schema}{ReadFloppyC}
        \Delta FloppyC
\\      RealWorldC
\where
	floppyPresenceC' = floppyPresenceC
\\      currentFloppyC' =  floppyC
\\      writtenFloppyC' = writtenFloppyC
\end{schema}

%-------------------------------------------------------------------------
\section{The ID Station changes the world}
%-------------------------------------------------------------------------


%..........................
\subsection{Periodic Updates}
%..........................

We consider the process of updating the real world with the current
internal 
representation, one variable at a time.

\begin{traceunit}{FD.Interface.UpdateLatch}
\traceto{FD.Interface.TISUpdates}
\traceto{FD.Interface.TISEarlyUpdates}
\end{traceunit}

\begin{schema}{UpdateLatchC}
	\Xi DoorLatchAlarmC
\\	RealWorldChangesC
\where
	latchC' = currentLatchC
\end{schema}

\begin{traceunit}{FD.Interface.UpdateAlarm}
\traceto{FD.Interface.TISUpdates}
\traceto{FD.Interface.TISEarlyUpdates}
\end{traceunit}

\begin{schema}{UpdateAlarmC}
	\Xi DoorLatchAlarmC
\\      AuditLogC
\\	RealWorldChangesC
\where
        alarmC' = alarming \iff doorAlarmC = alarming \lor auditAlarmC = alarming 
\end{schema}

\begin{traceunit}{FD.Interface.UpdateDisplay}
\traceto{FD.Interface.TISUpdates}
\end{traceunit}

\begin{schema}{UpdateDisplayC}
	\Delta IDStationC
\\	RealWorldChangesC
\where
	displayC' = currentDisplayC
\\
        currentDisplayC' = currentDisplayC
\end{schema}

Configuration Data is only displayed if the security officer is
present. System statistics are only displayed if an administrator is
logged on.


\begin{traceunit}{FD.Interface.UpdateScreen}
\traceto{FD.Interface.TISUpdates}
\end{traceunit}

\begin{schema}{UpdateScreenC}
        \Xi IDStationC
\\      \Xi AdminC
\\      RealWorldChangesC
\where
           screenC'.screenMsgC = currentScreenC.screenMsgC
\\           screenC'.screenConfigC = \IF \The rolePresentC =
securityOfficer \THEN displayConfigDataC (\theta ConfigC) \ELSE clearC 
\\            screenC'.screenStatsC = \IF rolePresentC \neq \Nil
\THEN displayStatsC (\theta StatsC) \ELSE clearC
\\     screenC'.screenDoorAlarm = displayAlarm~ doorAlarmC
\\     screenC'.screenLogAlarm = displayAlarm~ auditAlarmC
\end{schema}
 
All these can be combined, along with no change in the remaining
real world variables,
to represent the regular updating of the world.

When updates to the real world occur it is possible that interfacing
with external devices will result in a system fault that is
audited. Not other aspects of TIS will change during updates of the
real world.

\begin{traceunit}{FD.Interface.TISEarlyUpdates}
\traceto{FS.Interface.TISEarlyUpdates}
\end{traceunit}



The alarm and the door latch will need to be updated as soon as
possible after polling the real world, this ensures that the system is
kept secure.

\begin{zed}
        TISEarlyUpdateC \defs UpdateLatchC \land UpdateAlarmC 
\\ \t3        \land [~ RealWorldChangesC | screenC' = screenC \land
        displayC' = displayC ~]
\\ \t3 \land \Xi UserTokenC \land \Xi AdminTokenC \land \Xi FingerC \land
\Xi FloppyC \land 
\\ \t3 \Xi ScreenC \land \Xi KeyboardC \land \Xi ConfigC \land
\Xi StatsC  
\\ \t3 \land \Xi KeyStoreC \land \Xi AdminC \land \Xi InternalC
\\ \t3 \land [ AddElementsToLogC | auditTypes~ newElements? \subseteq
\{ systemFaultElement \} ]  \hide (newElements?) 
\end{zed}

\begin{traceunit}{FD.Interface.TISUpdates}
\traceto{FS.Interface.TISUpdates}
\end{traceunit}

The alarm, door latch, display and TIS screen will be updated after performing any
calculations. 

\begin{zed}
        TISUpdateC \defs UpdateLatchC \land UpdateAlarmC \land UpdateDisplayC \land UpdateScreenC
\\ \t3 \land \Xi UserTokenC \land \Xi AdminTokenC \land \Xi FingerC \land
\Xi FloppyC \land 
\\ \t3 \Xi KeyboardC \land \Xi ConfigC \land
\Xi StatsC  
\\ \t3 \land \Xi KeyStoreC \land \Xi AdminC \land \Xi InternalC
\\ \t3 \land [ AddElementsToLogC | auditTypes~ newElements? \subseteq
\{ systemFaultElement \} ]  \hide (newElements?) 
\end{zed}

%......................
\subsection{Updating the user Token}
%......................

\begin{traceunit}{FD.Interface.UpdateToken}
\traceto{FS.Interface.UpdateToken}
\end{traceunit}

We have a further operation, which writes to the User Token only.
We treat this separately because we expect to update the other devices
regularly and frequently,
but we will only be updating the User Token when we have something to
write. 

\begin{schema}{UpdateUserTokenC}
	\Delta IDStationC
\\	RealWorldChangesC
\also
        \Xi TISControlledRealWorldC
\where
	userTokenC' = currentUserTokenC
\end{schema}

%......................
\subsection{Updating the Floppy}
%......................

\begin{traceunit}{FD.Interface.UpdateFloppy}
\traceto{FS.Interface.UpdateFloppy}
\end{traceunit}


We have an operation which writes to the Floppy only.
We will only be updating the Floppy disk when we have something to write.

\begin{schema}{UpdateFloppyC}
        \Delta IDStationC
\\      RealWorldChangesC
\also
        \Xi UserTokenC
\\      \Xi AdminTokenC
\\      \Xi FingerC
\\      \Xi DoorLatchAlarmC
\\      \Xi KeyboardC
\\      \Xi ConfigC
\\      \Xi StatsC
\\      \Xi KeyStoreC
\\      \Xi AdminC      
\\      \Xi AuditLogC
\\      \Xi InternalC
\also
	\Xi TISControlledRealWorldC
\where
	floppyC' = writtenFloppyC
\\      currentFloppyC' = badFloppyC
\also
        floppyPresenceC' = floppyPresenceC
\\      currentDisplayC' = currentDisplayC
\\      currentScreenC' = currentScreenC
\end{schema}
\begin{Zcomment}
\item
Having written the floppy we can assume nothing about the $currentFloppy$
until we next poll. We do not know what data is on the floppy as it
may have been corrupted during the write. This ensures that the
readback we do is forced to be effective. 
\end{Zcomment}


%------------------------------------------------------------------------
\section{Clearing Biometric Data}
%------------------------------------------------------------------------
\begin{traceunit}{FD.Interface.FlushFingerData}
\traceto{FDP\_RIP.2.1}
\end{traceunit}

The biometric device must be cleared of stale data after a fingerprint
has been verified and before an attempt is made to capture fingerprint
data. This will force the biometric device to capture fresh data.

\begin{schema}{FlushFingerDataC}
        fingerC, fingerC' : FINGERPRINTTRY
\where
        fingerC' = noFP
\end{schema}

