%==========================================================================
\chapter{Example Refinement}
\label{chap:refine}
%==========================================================================
%%ignore \shows
\def\The{the~}%
This chapter presents part of the refinement argument,
showing that the Formal Design  is a correct refinement of the Formal Specification.

The refinement that we have carried out from formal specification to design
is not particularly complex.
For this reason, and to constrain costs, we have focused on the parts we believe will give
the best cost-benefit. We have therefore carried out hand proofs of pre-conditions
(that the pre-conditions of the designed operations are at least as permissive as the
pre-conditions of the specified operations)
and of the correctness of the most complex design step:
auditing.

All of these proofs have hand-written documentation.
The benefit to the correctness of the system stems from the action of doing
the proofs, not of documenting them.
If we expected this system to have a long life and be subject to maintenence,
we would document the proofs in electronic form.

For the purposes of this project,
we have documented here the correctness proof for the audit actions.

%--------------------------------------------------------------------------
\section{Refinement proof obligations}
\label{refine:ProofOb}
%--------------------------------------------------------------------------
The general proof rules for refinement in Z are given below.
These are a simplification of the common `forward' proof rules, sufficient in most situations.

We use the following general schemas:

\begin{tabular}{ll}
  Abstract State			&	$A$
\\Abstract Initialisation	&	$AInit$
\\Abstract Operation		&	$AOp$
\\
\\Concrete State			&	$C$
\\Concrete Initialisation	&	$CInit$
\\Concrete Operation		&	$COp$
\\
\\Retrieve between $A$ and $C$	&	$R$
\end{tabular}

{\bf Initialisation}

Proof that whenever the concrete system can be initialised
($CInit$), it is possible to find an abstract state that both retrieves ($R$)
and correctly initialises ($AInit$).
``If you can switch on the concrete,
you could have achieved the same by switching on the abstract.''
\begin{argue}
	CInit \shows \exists A @ AInit \land R
\end{argue}

{\bf Applicability (pre-conditions)}
Proof that whenever there is a concrete state that retrieves to an abstract state
able to undergo the abstract operation ($R$ contains both $C$ and $A$),
then the concrete state is also able to undergo the equivalent concrete operation ($COp$).
``Concrete operations are applicable whenever the abstract operation is.''
\begin{argue}
	R | \pre AOp \shows \pre COp
\end{argue}

{\bf Correctness}
Proof that whenever a concrete operation ($COp$) is carried out when the abstract operation
would also have been allowed ($\pre AOp$),
then the answer achieved (the $C'$ in $COp$) is an allowed answer ($\exists A'$)
from the abstract operation ($AOp$).
``A concrete operation always yields an answer that could have been seen in the abstract.''
\begin{argue}
	R; COp | \pre AOp \shows \exists A' @ AOp \land R'
\end{argue}


%--------------------------------------------------------------------------
\section{Audit correctness proof}
\label{refine:corr}
%--------------------------------------------------------------------------
The most complex step in the design is the realisation of the abstract
auditing process as writing to a series of individual audit files.

We can draw the auditing part out by noticing
that it appears in the abstract and the concrete conjoined with the `meat' of each operation,
but acting on entirely independent variables.
$AddElementsToLog$ and $AddElementsToLogC$ act on $AuditLog$ and $AuditLogC$ respectively,
using only the variable $newElements?$,
which is defined by the meat of the operation.
Therefore, it is valid to consider refining $AddElementsToLog$ by $AddElementsToLogC$ in isolation.

The design tackles auditing in two stages:
first strictly declaratively,
and then recursively element-by-element.
We will consider the refinement of the declarative version first.

%--------------------------------------------------------------------------
\subsection{Declarative version}
\label{refine:decl}
%--------------------------------------------------------------------------
The abstract has the variable $newElements?$ embedded within it,
existentially quantified.
We can draw it out explicitly to make the signatures of the abstract and concrete compatible
without altering the underlying meaning of the schemas.
We can define a schema in the obvious way that has the property
%%\begin{schema}{AddElementsToLogExplicit}
%%	newElements: \finset Audit
%%\\	AddElementsToLog
%%\end{schema}
\begin{argue}
	AddElementsToLogExplicit \hide (newElements) \equiv AddElementsToLog
\end{argue}

The design is expressed as a disjunction of four behaviours.
The abstract operation is total,
provided that the recorded times in $newElements$
are all newer that all the times already in the logs.
This is equivalent to the concrete requirement that all $newElements?$
have times newer than $nowC$,
as all elements in the logs must have been added in previous cycles,
and time only increases.

(Note that we also require $newElements$ to be non-empty, which it will be in use.)

The concrete is a little less total:
$\# newElements? < maxLogFileEntries$.
We accept this as a practical limitation,
and ensure only that no cycle can ever produce more log entries than allowed by this constraint.

We have now simplified the correctness proof obligation to:

\begin{zed}
%%\forall
ConfigR; AuditLogR; AddElementsToLogC |
0 < \# newElements? < maxLogFileEntries
%%@
\\ \shows
\\ \exists AuditLog'; newElements: \finset Audit @ 
\\ \t1	AddElementsToLogExplicit
\\ \t1	\land newElements = auditR \limg newElements? \rimg
\\ \t1	\land AuditLogR'
\end{zed}

The four disjuncts cover the range of inputs:
no elements;
enough for current file;
too many, but got another file;
and
the rest.
So having shown they cover the pre-condition sufficiently,
we need only show that each one independently refines the abstract.

For simplicity,
we will take the bijection 
%%unchecked
\begin{zed}
	auditR: AuditC \bij Audit
\end{zed}
as read,
and identify $newElements?$ and $newElements$.

%--------------------------------------------------------------------------
\subsubsection{AddElementsToLog refined by AddNoElementsToLog}
\label{refine:NoElements}
%--------------------------------------------------------------------------
\begin{zed}
%%\forall
ConfigR; AuditLogR; AddNoElementsToLog |
0 < \# newElements? < maxLogFileEntries
%%@
\\ \shows
\\ \exists AuditLog' 
%% ; newElements: \finset Audit 
@ 
\\ \t1	AddElementsToLogExplicit
\\ \t1	\land AuditLogR'
\end{zed}

Extends pre-condition to empty $newElements?$, so hypothesis is always false.

The result is proved.

%--------------------------------------------------------------------------
\subsubsection{AddElementsToLog refined by AddElementsToCurrentFile}
\label{refine:Current}
%--------------------------------------------------------------------------
\begin{zed}
%%\forall
ConfigR; AuditLogR; AddElementsToCurrentFile |
0 < \# newElements? < maxLogFileEntries
%%@
\\ \shows
\\ \exists AuditLog' 
%% ; newElements: \finset Audit 
@ 
\\ \t1	AddElementsToLogExplicit
\\ \t1	\land AuditLogR'
\end{zed}

We choose to prove the first disjunct of $AddElementsToLog$ only
(which we are free to do,
and will in fact be the case because we are not truncating the logs.)

From the retrieve relation we know

\begin{zed}
%%\forall AuditLogR @
	auditLog = 
%%auditR 
%%\limg 
\bigcup (\ran logFiles)
%%\rimg
\end{zed}

then the predicate in $AddElementsToLogExplicit$

\begin{argue}
	auditLog' = auditLog \cup newElements?
\end{argue}

clearly retrieves from the predicate in $AddElementsToCurrentFile$

\begin{argue}
	logFiles' = logFiles \oplus
		\{ currentFile \mapsto logFiles~currentLogFile \cup newElements? \}
\end{argue}

(Note that the logic also works if we choose random log files rather than the current one.
We need only ensure that the file whose size we check is the file we use.)

To prove the predicates on $alarming$,
we need to relate the concrete sizes and numbers of audit elements to the abstract size functions.
From the $alarming$ predicate in $AddElementsToCurrentFile$ take

\begin{argue}
	numberLogEntries' \geq alarmThresholdEntries	
\end{argue}

Multiply both sides by $sizeAuditElement$

\begin{argue}
	numberLogEntries' * sizeAuditElement \geq alarmThresholdEntries * sizeAuditElements	
\end{argue}

But $ConfigC$ tells us that

\begin{argue}
	alarlThresholdEntries * sizeAuditElement \geq alarmThresholdSizeC
\end{argue}

and therefore we can deduce

\begin{argue}
	numberLogEntries' * sizeAuditElement \geq alarmThresholdSizeC
\end{argue}

From the retrieves,
and the properties of $sizeLog$ given with the retrieves,
these values can be replaced with

\begin{argue}
	sizeLog~auditLog' \geq alarmThresholdSize
\end{argue}

as needed for the abstract predicate.

The second predicate is derived similarly:

\begin{argue}
	numberLogEntries' < alarmThresholdEntries	
\end{argue}

Replace the strict less-than by reducing the RHS by 1
(they are integers)

\begin{argue}
	numberLogEntries' \leq alarmThresholdEntries - 1
\end{argue}

Multiply both sides by $sizeAuditElement$

\begin{argue}
	numberLogEntries' * sizeAuditElement \leq (alarmThresholdEntries - 1) * sizeAuditElements	
\end{argue}

But from $ConfigC$ the RHS is strictly less than $alarmThresholdSizeC$,
giving us

\begin{argue}
	numberLogEntries' * sizeAuditElement < alarmThresholdSizeC
\end{argue}

From the retrieves,
and the properties of $sizeLog$ given with the retrieves,
these values can be replaced with

\begin{argue}
	sizeLog~auditLog' < alarmThresholdSize
\end{argue}

This gives us the predicates on alarming, and completes this branch.

%--------------------------------------------------------------------------
\subsubsection{AddElementsToLog refined by AddElementsToNextFileNoTruncate}
\label{refine:NextFile}
%--------------------------------------------------------------------------
\begin{zed}
%%\forall
ConfigR; AuditLogR; AddElementsToNextFileNoTruncate |
0 < \# newElements? < maxLogFileEntries
%%@
\\ \shows
\\ \exists AuditLog' 
%% ; newElements: \finset Audit 
@ 
\\ \t1	AddElementsToLogExplicit
\\ \t1	\land AuditLogR'
\end{zed}
The argument runs exactly as above,
but now $newElements?$ is split between $elementsInCurrentFile$ and $elementsInNextFile$.
But these get combined directly in $\bigcup (\ran logFiles)$,
so all the same arguments hold.

%--------------------------------------------------------------------------
\subsubsection{AddElementsToLog refined by AddElementsToNextFileWithTruncate}
\label{refine:NextTrunc}
%--------------------------------------------------------------------------
\begin{zed}
%%\forall
ConfigR; AuditLogR; AddElementsToNextFileWithTruncate |
0 < \# newElements? < maxLogFileEntries
%%@
\\ \shows
\\ \exists AuditLog' 
%% ; newElements: \finset Audit 
@ 
\\ \t1	AddElementsToLogExplicit
\\ \t1	\land AuditLogR'
\end{zed}
Choose to refine the second branch of the abstract schema,
which we can choose whenever
\begin{argue}
	sizeLog~auditLog + sizeLog~newElements? > minPreservedLogSize	
\end{argue}

We know this is true from the hypothesis because only one file is discarded,
and as the implementation has the property that all-files-minus-one
is bigger than $maxSupportedLogSize$
(which is itself bigger than $minPreservedLogSize$),
we always preserve at least this size of audit information,
and we only ever consider truncating when larger than $minPreservedLogSize$.

The property on audit log holding the correct elements is again achieved
by the assignment of $logFiles'$,
together with correct time constraints.
The choice of the file to discard as the head of the list of used files ensures it is the oldest.

Alarm is explicitly set in both concrete and abstract operations.

Note the $numberLogEntries$ is calculated to preserve its correct value
as the number of entires actually stored.

%--------------------------------------------------------------------------
\subsubsection{Refinement}
\label{refine:end}
%--------------------------------------------------------------------------
We have therefore shown that the abstract audit operations,
including the option of truncating the audit log,
is correctly refined by the declarative design.

%--------------------------------------------------------------------------
\subsection{Recursive}
\label{refine:Rec}
%--------------------------------------------------------------------------
We now show that the concrete element-by-element additions are an alternative representation
of the same behaviour.

First, we show that $AddElementToLogC$ is just a specialisation of $AddElementsToLogC$
for single elements, i.e 

\begin{argue}
	[AddElementsToLogC | \# newElements? = 1] \equiv AddElementToLogC
\end{argue}

 We consider two cases:
%..........................................................................
\subsubsection{Truncate not required}
\label{refine:truncNot}
%..........................................................................
The precondition for the single element schema can be derived from
$TruncateLogNotRequired$ and $AddElementsToLogFile$.
It is

\begin{argue}
	freeLogFiles \neq \emptyset
	\land \# (logFiles~currentLogFile) = maxLogFileEntries
\\ 	\lor \# (logFiles~currentLogFile) < maxLogFilesEntries
\end{argue}

The precondition for the multiple element schema is

\begin{argue}
	freeLogFiles \neq \emptyset
 	\land \# newElements? + \# (logFiles~currentLogFile) > maxLogFilesEntries
\\ 	\lor \# newElements? + \# (logFiles~currentLogFile) \leq maxLogFilesEntries
\end{argue}
Assuming a single element in $newElements?$, 
we can replace $\# newElements?$ with $1$,
and given that the sizes are integers,
these can be seen to be identical
(given that we can show that the size of the log files never actually exceeds
$maxLogFileEntries$).

Both schemas break into two disjuncts:

{\bf current file}:
which can be seen to be identical in the two by inspection, and

{\bf next file}:
which can also be seen to be identical in the two by inspection,
given that we can choose $elementsInCurrentFile$ to be empty and hence
$elementsInNextFile = newElements?$.

%..........................................................................
\subsubsection{Truncate is required}
\label{refine:trunc}
%..........................................................................
The precondition for the single element schema is derived from the
sequential composition of three schemas,
but can be seen to be the negation of the precondition for no truncation:

\begin{argue}
	freeLogFiles = \emptyset
\\ 	\land \# (logFiles~currentLogFile) = maxLogFilesEntries
\end{argue}

(We do need to confirm that the apparent precondition seen in $TruncateLog$
is not restricted by the later sequential compositions.
But releasing a log file and reducing the number of log entries ensures
that the two applications of $AddElementToLog$ will proceed.)

The precondition for the multiple element schema is

\begin{argue}
	freeLogFiles = \emptyset
\\ 	\land \# newElements? + \# (logFiles~currentLogFile) \geq maxLogFilesEntries
\end{argue}
As before, these are the same when $\# newElements? = 1$ .

In the single element schema,
the log is truncated,
then the truncation element added,
then the real audit element is added.

We will now look at each predicate in the declarative version and see how its
equivalent is constructed by these sequential operations in the single element version.

{\em predicate 1:}
\begin{argue}
	numberLogEntries' = numberLogEntries + 1 - maxLogFileEntries + 1
\end{argue}
The subtraction is defined in $TruncateLog$,
and each of the additions comes from an application of $AddElementToLogFile$.

{\em predicate 2:}
\begin{argue}
	\exists truncElement \ldots
\end{argue}
Each component property can be compared with the equivalent in the single element
version and seen to be the same.

{\em predicate 3:}
\begin{argue}
	elementsInCurrentFile \subseteq newElements?
\end{argue}
Choose this to be empty.

{\em predicate 4:}
\begin{argue}
	\# (logFiles~currentLogFile) + \# elementsInCurrentFile = maxLogFileEntries
\end{argue}
$\# elementsInCurrentFile$ is zero by choice.
This predicate is then true by precondition.

{\em predicate 5:}
\begin{argue}
	elementsInNextFile = newElements? \setminus elementsInCurrentFile
\end{argue}
By choosing $elementsInCurrentFile$ empty,
this forces $elementsInNextFile$ to equal $newElements?$.

{\em predicate 6:}
\begin{argue}
	oldestLogTime~elementsInNextFile \geq truncElement.logTime
\end{argue}
See predicate 7.

{\em predicate 7:}
\begin{argue}
	truncElement.logTime \geq newestLogTimeC~elementsInCurrentFile 
\end{argue}
There are three time intervals in the sequential version:
$Truncate$, $AddElementToLogFile$ 
(which adds the truncate audit element),
and $AddElementToLogFile$ 
(which adds $newElement?$).
Time is forced to move on between each of these intervals,
and this constrains these two predicates 6 \& 7 to be true.

{\em predicate 8:}
\begin{argue}
	logFiles' = \ldots
\end{argue}
Application of $TruncateLog$ updates $head~usedLogFiles$ to empty,
then $AddElementToNextLogFile$ (this one because of pre-conditions)
adds the truncation element (due to renaming in composition) to this file,
updating $currentLogFile$ to this file
(which is the only one in the list of $freeLogFiles$, put there by $TruncateLog$),
and then $AddElementToCurrentLogFile$ 
(because only single element in this file now,
so conditions choose this one)
adds the $newElement?$ to this.


{\em predicate 9:}
\begin{argue}
	currentLogFile' = head~usedLogFiles
\end{argue}
Explained above with predicate 9.

{\em predicate 10:}
\begin{argue}
	usedLogFiles' = tail~usedLogFiles \cat \langle currentLogFile' \rangle
\end{argue}
$Truncate$ tails, then next adds the new one, then next leaves it alone.

{\em predicate 11:}
\begin{argue}
	freeLogFiles' = freeLogFiles
\end{argue}
Adds one, removes it, leaves alone.

{\em predicate 12:}
\begin{argue}
	logFilesStatus' = logFileStatus \oplus \{ currentLogFile' \mapsto used \}
\end{argue}
Set to free, then used, then left.

{\em predicate 13:}
\begin{argue}
	auditAlarmC' = alarming
\end{argue}
$Truncate$ sets, rest leave it alone.

So $AddElementToLogC$ is equivalent to $AddElementsToLogC$ for an individual element,
as we wished to show.

(We don't actually need to check the recursive definition,
because the implementation will actually apply $AddElementToLogC$ sequentially,
chronologically.)

This is sufficient to show that the design step made from the abstract
formal specification to the more concrete design specification is correct.
