%============================================================================
\chapter{Commentary on this Design}
\label{sec:Summary}
%============================================================================

This design is intended to give a representative formal
refinement of the Formal Specification \cite{FS}.

%--------------------------------------------------------------------------
\section{The structure of the Z}
%--------------------------------------------------------------------------
The formal design follows the structure of the Formal Specification.
This is done to aid the refinement process and provide a natural
refinement step from specification to implementation.

As in the specification every effort has been taken to ensure schemas are
simple.

The section containing internal operations and checks has been
expanded. 
A number of common constraints have been factored out
as checks that can be performed in the context of one or a small
number of subsystems. 
In order to simplify the step from
design to implementation invariants which define key values, such as
the door alarm, have been replaced by subsystem operations. The design
then shows where these operations need to be invoked to ensure that
the desired invariants are maintained.

%--------------------------------------------------------------------------
\section{Issues}
%--------------------------------------------------------------------------
A few issues arose while writing this design, some of which point to
shortfalls which would need to be resolved if EAL level 6 or 7 were
required.

We present the more interesting observations here:


\subsection{Peripheral Failures and System Faults}

The design does not address fully the possibility of peripheral
failures. This would certainly need to be addressed for EAL 6 or 7
where fully formal proof of the implementation conforming to the
design is required. 

We do model the possibility of a system fault being raised and this is
intended to cover peripheral failures; however, we do not elaborate what
should occur in the event of such a failure. It is likely that
peripheral failures would, in a full development be categorised in
terms of their criticality and the desired system behaviour as part of
the requirements elicitation activity. 

There are a number of points where the modelling of failures could be
improved. The manner in which these could be improved is discussed
below. 

The model makes the assumption that any attempt to read a token is
successful in that the internal representation exactly reflects the
real world contents of the token. In order to model the possibility of
failure during the read the model should allow the non-deterministic
possibility of the internal value of the token becoming $badT$
representing a corrupt or failed read. This non-determinism would also
need to be present in the specification to ensure that the design is a
refinement of the specification. 

A small number of system faults are deemed security
critical. These are likely to include  
\begin{itemize}
\item
failure to be able to write to the audit log; 
\item
any detectable failure in operating the latch
\item
any detectable failure to be able to monitor the state of the door.
\end{itemize}
These failures will occur non-deterministically and we should specify
the desired behaviour if each of these occur. 

A failure to write to the audit log is severe since it means that
activities could proceed un-audited.  
In the event of a failure to write to the audit log the audit alarm
should be raised and the system should be shutdown preventing it from
participating an any further activities. 

In the event of a failure we can assume little about the state of the
current log file, we assume that nothing old was lost but some
elements may have been added.

\begin{schema}{AuditLogFailure}
	\Delta AuditLogC
\where
	auditAlarmC' = alarming
\\      logFilesStatus' = logFilesStatus
\\      currentLogFile' = currentLogFile
\\      usedLogFiles' = usedLogFiles
\\      freeLogFiles' = freeLogFiles
\\      \{ currentLogFile \} \ndres logFiles' = \{ currentLogFile \}
\ndres logFiles
\\      logFiles~ currentLogFile \subseteq logFiles'~currentLogFile	
\end{schema}

In the event of such a failure the administrator should be logged off
and the system shutdown. The door should be locked to ensure the
enclave is left in a secure state.

\begin{schema}{ShutdownAuditFailure}
	\Delta IDStationC
\\      RealWorldChangesC
\also
        LockDoorC
\\	\Xi KeyStoreC
\\	\Xi CertificateStore
\\	\Xi ConfigC
\\      \Xi FloppyC
\\      \Xi KeyboardC
\\      \Xi AdminTokenC
\\      \Xi UserTokenC
\\      AdminLogoutC
\\      \Xi FingerC
\\      \Xi StatsC
\\      AuditLogFailure
\where
	enclaveStatusC' = shutdown
\\	statusC' = quiescent
\also
\\	currentDisplayC' = blank
\\	currentScreenC'.screenMsgC = clearC
\end{schema}

It is likely that the desired behaviour in the event of a failure of
the door or latch is to assume that the system is in an insecure state
and raise an alarm. It may also be desirable to shutdown the system,
preventing any further action. 

\begin{schema}{DoorLatchFailure}
	\Delta DoorLatchAlarmC
\where
	doorAlarmC' = alarming
\\	currentTimeC' = currentTimeC
\\	currentDoorC' = open
\\	currentLatchC' = locked
\\	latchTimeoutC' = zeroTime
\\	alarmTimeoutC' = zeroTime
\end{schema}

In the event of such a failure, the fault can be logged and the system shutdown.

\begin{schema}{ShutdownDoorLatchFailure}
	\Delta IDStationC
\\      RealWorldChangesC
\also
        DoorLatchFailure
\\	\Xi KeyStoreC
\\	\Xi CertificateStore
\\	\Xi ConfigC
\\      \Xi FloppyC
\\      \Xi KeyboardC
\\      \Xi AdminTokenC
\\      \Xi UserTokenC
\\      AdminLogoutC
\\      \Xi FingerC
\\      \Xi StatsC
\\      AddElementsToLogC
\\      LogChangeC
\where
	enclaveStatusC' = shutdown
\\	statusC' = quiescent
\also
\\	currentDisplayC' = blank
\\	currentScreenC'.screenMsgC = clearC
\also
        \exists_1 element : AuditC @ element \in newElements? 
\\ \t1  \land element.elementId = systemFaultElement
\\ \t1  \land element.logTime \in nowC \upto nowC'
\\ \t1  \land element.user = noUser
\\ \t1  \land element.severity = critical
\end{schema}

As faults are not modelled in the specification refinement would not
be achievable if system faults were modelled in the design. 

\subsection{Unelaborated aspects of the Design}
Normally all types within the design would be elaborated in terms of
entities that  
closely model the implementation type.

There are some aspects of certificates that have not been fully
 elaborated within this  design. These are $FINGERPRINT$, and
 $FINGERPRINTTEMPLATE$. All of these would normally be elaborated in
 terms of a  model of the
 implementation types. This is unnecessary for these three
 entities. The core TIS has no reason to use  the $FINGERPRINT$ or
 $FINGERPRINTTEMPLATE$, it 
 simply passes the information to the Biometric library.

The components of an issuer $USERID$ and $USERNAME$ are free types 
within the design. The only property that is utilised within the
design is equality of  
$USERID$. For this demonstration implementation the user Id is simplified to a 
numeric although this is not completely realistic so is not elaborated
within the  
design.

\subsection{Enrolment Protocol}
Enrolment is a simplified model of part of the enrolment protocol. The likely 
enrolment protocol would involve the following stages.
\begin{enumerate}
\item
 TIS generates a pubic/private key pair at 
initialisation and uses the public key to create a request for enrolment. 
\item
The enrolment request is presented to a CA. The CA would generate an Id certificate 
for the TIS, this will contain the authorised name of the TIS as its subject and the TIS 
public key. 
\item
An AA constructs the enrolment data. 
Enrolment data comprises a number of Id certificates, including the Id 
certificate of the TIS itself and the Id certificate of the CA that issued the TIS Id 
certificate. 
\item
TIS accepts the enrolment data and uses this to establish known issuers.
\end{enumerate}
Within the design we only model the final phase of enrolment. 

TIS would only actually participate in the first and last phase of
this protocol, the other two activities being performed by a CA and
AA.  

Due to budgetary limitations we have omitted the first phase of this
protocol from the design model. This is possible since this
demonstration mimics the keys and the encryption process. There is no
need for our demonstration to be supplied with the  
public key that corresponds to an internally held private key as the
private key is not used in the mimicked encryption. Instead TIS will
record the presence of the private  
key once enrolment has supplied its ID Certificate. 

\subsection{Reading Tokens}

The formal design shows all certificates on a token being read when
anything from the token is required. In actuality the authorisation
certificate will only ever be read from the administrator token, while
the reading of certificates from the user token will follow the
following ordering.
\begin{itemize}
\item
An attempt is made to read the authorisation certificate and ID certificate.
\item
If these are present then they are validated.
\item 
If they fail to validate or are not present then the remaining 
certificates are read.
\end{itemize}
Due to budgetary limitations this was not progressed within this
 design although it would be necessary to achieve EAL 6 or above to
 enable formal proof of the implementation satisfying the design. 

The design as it stands is not invalid, it just presents a slightly
larger step between design and implementation than might be desirable.

\subsection{Token Representation}
Within the formal design we represent tokens as containing a number of raw 
certificates. This is an effective model for the real world view of
the tokens but it is a  
less satisfactory model for the internal representation of the token.

Given more resources we would have modelled the internal tokens as
containing the contents of the certificates that we are interested
in. 
So for the administrators token  
only the contents of the Authorisation Certificate would be preserved,
while for the user token the contents of the all the certificates may
be maintained. 

This would have the advantage within the design of removing the need
to extract the required fields from the various tokens every time they 
are required.

If this design were to be progressed further it would be worth
modelling the internal  
representation of tokens as maintaining the contents of selected
certificates rather than  
the raw certificates. This would then result in a smaller step to
implementation. 

\subsection{Relating enclave entry and Auth Cert generation}

Within the specification independent configuration is used to
determine the authorisation period applied to authorisation
certificates and the times at which entry to the enclave should be
allowed.

It should be noted that if the user obtains an authorisation
certificate and the system is reconfigured before the authorisation
certificate expires then it is still possible for the user to possess
a current authorisation certificate and be denied entry.

\subsection{Comments on the refinement relation}
There are a number of circumstances where not all the abstract
entities in the specification can be retrieved from the concrete
entities in this design. In particular this applies to certificates on
tokens and the configuration data. In the case of certificates this is
due to the validity period used in the specification not necessarily
being 
contiguous. In the case of the configuration data this is due to the
enormous freedom in the definition of the abstract $authPeriod$ and
$entryPeriod$ functions.
\begin{itemize}
\item
The concrete $authPeriod$ and $entryPeriod$ are
independent of $role$. 
\item
There is no relationship between the $authPeriod$ and
$entryPeriod$. 
\item
The concrete $authPeriod$ is always a contiguous range
of times. 
\end{itemize}

As the specification has a very abstract view of what the real world
can do, this is acceptable. 

During TIS operations the real world undergoes change which is
relatively unconstrained, in both the concrete and abstract model time
must not decrease but all other real world entities that are not
controlled by TIS may change arbitrarily. In the specification there
are more possible states in which the real world can change into, the
abstract tokens can change to ones that cannot be represented in the
concrete model, floppy data can change to contain configuration data
that is not valid configuration data in the concrete model. In all
these cases the concrete ``bad'' value is a refinement of the abstract values
that are not attainable in the concrete real world.

This refinement is acceptable as long as our concrete real world still
allows all values that the requirements consider should be valid
inputs.

In a real development of a working product there would be a part 
of requirements elictation in which the exact nature of all the inputs is
discussed. This discussion may well be postponed until the Formal
Specification is in place providing a useful context for
discussion, this would very much depend on the nature of the inputs
and whether the product development can control the allowable range of values. 

Our design has constrained the validity periods on certificates to
reflect the contiguous ranges that can be specified reflecting true
requirements constraints on the nature of X509 certificates.

The new constraints on the configuration data have been introduced to
limit allowed configurations to those that can be specified with a
small number of parameters. In a real development these are design
constraints and would need to be discussed with the customer to ensure
that sufficient flexibility remains in the allowed configuration values.
 


