
%==========================================================================
\chapter{The Token ID Station}
\label{sec:TIS}
%==========================================================================
TIS maintains various state components, these are described and
elaborated within this section. 

%----------------------------------------------------------------------
\section{Configuration Data}
%----------------------------------------------------------------------

\begin{traceunit}{FD.ConfigData.State}
\traceto{FS.ConfigData.State}
\end{traceunit}

$ConfigData$ will be a structure with all the configuration data. 
Configuration data can only be modified by an administrator.
This data includes:
\begin{itemize}
\item 
Durations for internal timeouts, these effect
\begin{itemize}
\item
how long the
system waits before raising an audible (door) alarm;
\item
how long the system
leaves the door unlocked for;
\item
how long the system waits for a token to be removed before unloading
the door; and
\item
how long the system attempts to capture a matching fingerprint.
\end{itemize}
\item
The security classification of the enclave. For this implementation
only the $CLASS$ is considered.
\item
A definition of the current working hours, this is in terms of the
start and end of the working day. All days are considered working
days, so there is no special treatment of weekends.
\item 
A definition of the current maximum authorisation period applied to an
authorisation certificate if ``all hours'' access is given.
\item
The access policy used to determine the entry conditions and the 
authorisation period. 
\begin{itemize}
\item
The access policy is either ``working hours only'' or
``all hours''.
\item
When the access policy is ``working hours only'' the authorisation
period will be from the current time to the end of the current working
day. This may be empty if the current time is after the end of the
working day. The user will only be admitted to the enclave if the
current time is within working hours.
\item
When the access policy is ``all hours'' the authorisation period will
be from the current time for the maximum authorisation duration.
The user will always be allowed into the enclave if all identification
checks are satisified.
\end{itemize}
\item
The lowest security classification a user must hold to gain entry to
the enclave. If this condition is not met then entry will be denied.
\item
$minPreservedLogSizeC$ gives the minimum size of audit log that must be supported without
truncation.  A slightly smaller
value, $alarmThresholdSizeC$,
sets the number of audit entries at which an alarm is raised, 
with the intension that the
audit log will be archived and cleared before the maximum size is
reached.
\item
$minEntryClass$ must be no higher class than $enclaveClearanceC$. This
ensures that any authorisation certificate issued with this
configuration data will also permit entry.
\item
$systemMaxFAR$ gives the system minimum acceptable false accept rate. This
will override the FAR provided within a template where the template
FAR exceeds this system limit.
\end{itemize}

\begin{zed}
        ACCESS\_POLICY ::=workingHours | allHours
\end{zed}

\begin{schema}{ConfigData}
	alarmSilentDurationC, latchUnlockDurationC : TIME
\\      tokenRemovalDurationC : TIME
\\      fingerWaitDuration : TIME
\\      enclaveClearanceC : CLASS
\also
        workingHoursStart : DAYTIME
\\      workingHoursEnd : DAYTIME
\\      maxAuthDuration : DAYTIME
\\      accessPolicy : ACCESS\_POLICY
\\      minEntryClass : CLASS 
\also
        minPreservedLogSizeC : \nat
\\      alarmThresholdSizeC : \nat
\also
        systemMaxFar : INTEGER32
\where
        alarmThresholdSizeC < minPreservedLogSizeC   
\\      minPreservedLogSizeC \leq maxSupportedLogSize
\\      minEntryClass = minClass\{ minEntryClass, enclaveClearanceC \}
\end{schema}
\begin{Zcomment}
\item
The upper bound on the $minPreservedLogSizeC$ ensures that the system can
support the selected value for this.
\end{Zcomment}

Notice that the concrete configuration data is simplified so that
authorisation periods and entry criteria do not depend on the user's
privilege. This is a design decision to simplify
these. 

The authorisation period is always a contiguous range of times. This
is necessary due to the way that the authorisation period is encoded
in the authorisation certificate.

The entry period is the same for each day. 

$ConfigData$ defines the data that must be provided in order to
perform a configuration. $ConfigC$ contains
extra components which are derived from $ConfigData$.

\begin{schema}{ConfigC}
        ConfigData
\also
        authPeriodC : TIME \fun \power TIME
\\      entryPeriodC : CLASS \fun \power TIME
\\      authPeriodIsEmpty : \power TIME
\\      getAuthPeriod : TIME \pfun TIME \cross TIME
\\      alarmThresholdEntries : \nat
\where
        accessPolicy = allHours 
\\ \t1  \land 
        authPeriodC = \{ t : TIME @ t \mapsto t \upto max~ \{0, t +
        maxAuthDuration - 1 \} \}
\\ \t1  \land 
        entryPeriodC = 
        \{ c : CLASS | maxClass \{ c, minEntryClass \} = c @ c
        \mapsto TIME \}
\\ \t2        \cup \{ c : CLASS | maxClass \{ c, minEntryClass \} \neq
        c @ c \mapsto \emptyset \}
\\      \lor 
\\      accessPolicy = workingHours 
\\ \t1  \land authPeriodC = \{ t :
        TIME @ t \mapsto (t \div dayLength) * dayLength +
        workingHoursStart \upto  
\\ \t4  (t \div dayLength) * dayLength + workingHoursEnd \}
\\ \t1  \land entryPeriodC = 
\\ \t2        \{ c : CLASS | maxClass \{ c, minEntryClass \} = c 
\\ \t3  @ c \mapsto
        \{ t : TIME | t \mod dayLength \in
        workingHoursStart \upto workingHoursEnd \} \}          
\\ \t2  \cup \{ c : CLASS | maxClass \{ c, minEntryClass \} \neq c @ c
        \mapsto \emptyset \}
\also
        authPeriodIsEmpty = \{ t : TIME | authPeriodC~ t = \emptyset
        \}
\\      getAuthPeriod = \{ t : TIME | authPeriodC~ t \neq \emptyset @
        t \mapsto (min~ (authPeriodC~ t), max~ (authPeriodC~ t)) \}   
\also
\\      (alarmThresholdEntries - 1) * sizeAuditElement < alarmThresholdSizeC
\\      alarmThresholdEntries * sizeAuditElement \geq alarmThresholdSizeC 
\end{schema}
\begin{Zcomment}
\item
Invarients on $authPeriodC$ and $entryPeriodC$ define these functions
in terms of the other configuration items. These values will not be
supplied as part of configuration data.
\item
Invarients on  $alarmThresholdEntries$ define this
values in terms of other configuration items. 
$alarmThresholdEntries$ is the number of elements in the log after
which the audit alarm will be raised.
\item
$getAuthPeriod$ and $authPeriodIsEmpty$ are completely determined by
invarients relating these entities to other configuration items.
\end{Zcomment}

%--------------------------------------------------------------------------
\section{Audit Log}
%--------------------------------------------------------------------------

\begin{traceunit}{FD.AuditLog.State}
\traceto{FS.AuditLog.State}
\traceto{FAU\_GEN.1.1}
\traceto{FAU\_GEN.1.2}
\end{traceunit}

TIS maintains an audit log. This is a log of all auditable events and
actions performed or monitored by TIS. The audit log will be used to
analyse the interactions with the TIS.

$Audit$ will be a structure for each audit record, 
recording at least time of event, type of event, user if known, the
user is identified from the ID Certificate on the token and a free
text description. The free text may contain additional information
relating to the specific type of event.

\begin{zed}
	AUDIT\_ELEMENT ::= 
\\ \t1        startUnenrolledTISElement
        | startEnrolledTISElement
        | enrolmentCompleteElement
        | enrolmentFailedElement
\\ \t1       | displayChangedElement
        | screenChangedElement
        | doorClosedElement
        | doorOpenedElement
\\ \t1        | latchLockedElement
        | latchUnlockedElement
        | alarmRaisedElement
        | alarmSilencedElement
\\ \t1        | truncateLogElement
        | auditAlarmRaisedElement
        | auditAlarmSilencedElement
\\ \t1        | userTokenRemovedElement
        | userTokenPresentElement
        | userTokenInvalidElement
\\ \t1        | authCertValidElement
        | authCertInvalidElement
\\ \t1        | fingerDetectedElement
        | fingerTimeoutElement
        | fingerMatchedElement
        | fingerNotMatchedElement
\\ \t1        | authCertWrittenElement
        | authCertWriteFailedElement
\\ \t1        | entryPermittedElement
        | entryTimeoutElement
        | entryDeniedElement
\\ \t1        | adminTokenPresentElement
        | adminTokenValidElement
        | adminTokenInvalidElement
\\ \t1  | adminTokenExpiredElement
        | adminTokenRemovedElement
\\ \t1        | invalidOpRequestElement
        | operationStartElement
\\ \t1        | archiveLogElement
        | archiveCompleteElement
        | archiveCheckFailedElement
        | updatedConfigDataElement
\\ \t1        | invalidConfigDataElement
        | shutdownElement
        | overrideLockElement
        | systemFaultElement
\also
        AUDIT\_SEVERITY ::= information | warning | critical
\end{zed}

\begin{zed}
        USER\_INDEPENDENT\_ELEMENTS == \{ systemFaultElement,
        displayChangedElement,
        screenChangedElement,
\\ \t1  doorClosedElement,
        doorOpenedElement,
        latchLockedElement,
        latchUnlockedElement,
\\ \t1  alarmRaisedElement,
        alarmSilencedElement,
        auditAlarmRaisedElement,
        auditAlarmSilencedElement,
        truncateLogElement \}

\also
        USER\_ENTRY\_ELEMENTS == \{ 
        userTokenRemovedElement,
        userTokenPresentElement,
\\ \t1        userTokenInvalidElement,
         authCertValidElement,
        authCertInvalidElement,
        fingerDetectedElement,
\\ \t1  fingerTimeoutElement,
         fingerMatchedElement,
         fingerNotMatchedElement,
\\ \t1   authCertWrittenElement,
         authCertWriteFailedElement,
\\ \t1   entryPermittedElement,
         entryTimeoutElement,
         entryDeniedElement \}

\also
        ADMIN\_ELEMENTS == \{ 
        adminTokenPresentElement,
        adminTokenValidElement,
\\ \t1         adminTokenInvalidElement,
        adminTokenExpiredElement,
         adminTokenRemovedElement,
\\ \t1   invalidOpRequestElement,
         operationStartElement,
\\ \t1         archiveLogElement,
        archiveCompleteElement,
         archiveCheckFailedElement,
         updatedConfigDataElement,
\\ \t1   invalidConfigDataElement,
        shutdownElement,
        overrideLockElement \}
\also
        ENROL\_ELEMENTS == \{ 
         enrolmentCompleteElement,
         enrolmentFailedElement \}

\also
        STARTUP\_ELEMENTS == \{ 
       startUnenrolledTISElement,
         startEnrolledTISElement \}

        
\also
        INFO\_ELEMENTS == \{ 
        startUnenrolledTISElement,
        startEnrolledTISElement,
        enrolmentCompleteElement,
\\ \t1        displayChangedElement,
        screenChangedElement,
        doorClosedElement,
        doorOpenedElement,
\\ \t1        latchLockedElement,
        latchUnlockedElement,
        alarmSilencedElement,
        auditAlarmSilencedElement,
\\ \t1      userTokenRemovedElement,
        userTokenPresentElement,
        authCertValidElement,
\\ \t1        authCertInvalidElement,
      fingerDetectedElement,
        fingerMatchedElement,
        fingerNotMatchedElement,
\\ \t1      authCertWrittenElement,
      entryPermittedElement,
      adminTokenPresentElement,
        adminTokenValidElement,
\\ \t1        adminTokenRemovedElement,
        operationStartElement,
      archiveLogElement,
        archiveCompleteElement,
\\ \t1        updatedConfigDataElement,
        shutdownElement,
        overrideLockElement
        \} 
\also
	WARNING\_ELEMENTS == 
        \{
         enrolmentFailedElement,
        auditAlarmRaisedElement,
        userTokenRemovedElement,
\\ \t1         userTokenInvalidElement,
        fingerTimeoutElement,
         authCertWriteFailedElement,
         entryDeniedElement,
\\ \t1        entryTimeoutElement,
         adminTokenInvalidElement,
\\ \t1   adminTokenExpiredElement,
         adminTokenRemovedElement,
       invalidOpRequestElement,
\\ \t1         archiveCheckFailedElement,
        invalidConfigDataElement, 
        systemFaultElement
        \}
\also
	CRITICAL\_ELEMENTS == 
        \{
         alarmRaisedElement,
         truncateLogElement,
         systemFaultElement
        \}
\end{zed}

\begin{traceunit}{FD.AuditLog.ExtractUser}
\end{traceunit}

Each audit element has an associated user, if the user is not relevant
or not available then the $noUser$ value is used.

\begin{zed}
        USERTEXT ::= noUser | thisUser \ldata CertificateIdC \rdata
\end{zed}

There is an extraction function which obtains the user from the
current token. This will extract the CertificateIdC from any token
sufficiently valid to contain one or return $noUser$.

\begin{axdef}
        extractUser : TOKENTRYC \fun USERTEXT
\end{axdef}

Each audit element has a free text field. This is an informal
description of the entry and may contain no text.

\begin{zed}
        [ TEXT ]
\end{zed}

\begin{axdef}
        noDescription : TEXT
\end{axdef}


\begin{schema}{AuditC}
        logTime : TIME
\\      elementId : AUDIT\_ELEMENT
\\      severity : AUDIT\_SEVERITY
\\      user : USERTEXT
\\      description : TEXT  
\end{schema}
Most audit elements have a user associated with them, where this can
be determined it will be supplied.

Some audit elements have different severities depending on their
context. A token removal is erroneous during an operation but not at
the end of an operation for instance.

We define a function that gives the set of $AUDIT\_ELEMENT$s captured
within a set of Audit elements. 
\begin{axdef}
        auditType :  AuditC \fun  AUDIT\_ELEMENT
\\      auditTypes : \finset AuditC \fun \finset AUDIT\_ELEMENT
\where
        auditType = (\lambda AuditC @ elementId )
\\      auditTypes = \{ A : \finset AuditC @ A \mapsto auditType \limg
A \rimg \}
\end{axdef}


In this implementation the size of each audit element is fixed, we
also note that the capacity of a floppy is fixed.
\begin{axdef}
        sizeAuditElement : \nat
\\      floppyCapacity : \nat
\end{axdef}

The Audit log consists of a number of $Audit$ elements. 
An audit error alarm will be raised if the audit log becomes full and
needs to be archived and cleared.

The Audit log will be implemented as a number of files with a fixed
maximum capacity. 
The intention is to distribute the log across the these files, this should
enable truncation to be implemented simply. 
There will be an internal upper limit to the number of files
supported. The size of each file is fixed in terms of the number of
audit elements it holds.

\begin{axdef}
        maxNumberLogFiles : \nat
\\      maxLogFileEntries : \nat
\\      maxNumberArchivableFiles : \nat
\where
        maxNumberLogFiles > 2
\\      maxLogFileEntries \geq 100
\\      maxSupportedLogSize \leq sizeAuditElement * (maxNumberLogFiles -1)
* maxLogFileEntries
\\      maxNumberArchivableFiles > 1
\\      maxNumberArchivableFiles * maxLogFileEntries *
sizeAuditElement \leq floppyCapacity
\end{axdef}
\begin{Zcomment}
\item
The system supports at least three files.
\item
Each file can hold at least 100 elements.
\item 
The files have sufficient capacity to support the maximum supported
log size defined in the specification, even when one file is empty. 
This will ensure that truncation preserves the conditions within the
specification. 
\item
At least one file can be archived onto a floppy.
\end{Zcomment}

In order to simplify the implementation we make a number of
assumptions about the internal implementation of the log file.
\begin{itemize}
\item
When data is archived only full logFiles are removed.
\end{itemize}


\begin{zed}
        LOGFILEINDEX == 1 \upto maxNumberLogFiles
\end{zed}        

All audit elements have associated with them a timestamp so it is
possible to determine the times of the newest and oldest entries in the
log.

\begin{axdef}
        oldestLogTimeC : \finset_1 AuditC \fun TIME 
\\      newestLogTimeC : \finset_1 AuditC \fun TIME
\where
        \forall A, B : \finset AuditC @
\\  \t1 newestLogTimeC (A \cup B) \geq newestLogTimeC~ A
\\  \t1 \land oldestLogTimeC (A \cup B) \leq oldestLogTimeC~ A
\also
\\  \t1 \land A \neq \emptyset \implies oldestLogTimeC~ A = min~ \{ audit
: A @ audit.logTime \} 

\\  \t1 \land A \neq \emptyset \implies newestLogTimeC~ A = max~ \{ audit
: A @ audit.logTime \} 
\end{axdef}
\begin{Zcomment}
\item 
Both these functions are monotonic. 
\end{Zcomment}

At any time each log file will either be empty and $free$ for use,
$used$ (or in use) or $archived$, in that an attempt has been made to
archive the data.

\begin{zed}
        LOGFILESTATUS ::= free | archived | used
\end{zed}

As all log entries are time stamped there is no requirement to impose
an ordering on the entries in an audit file, however we do insist that
the log files can be ordered such that the all the elements in the
oldest file are older than all the elements in the other files. 

\begin{schema}{AuditLogC}
        logFiles : LOGFILEINDEX \fun \finset AuditC
\\      currentLogFile : LOGFILEINDEX
\\      usedLogFiles : \iseq LOGFILEINDEX
\\      freeLogFiles : \power LOGFILEINDEX
\\      logFilesStatus : LOGFILEINDEX \fun LOGFILESTATUS
\\      numberLogEntries : \nat
\also 
        auditAlarmC : ALARM
\where
        freeLogFiles = \dom (logFiles \rres \{ \emptyset \} )
\\      freeLogFiles = \dom (logFilesStatus \rres \{ free \} )
\\      \ran usedLogFiles = \dom (logFilesStatus \rres \{ archived,
used \} )
\also
      \forall file1, file2 : \ran usedLogFiles @
\\ \t1  usedLogFiles \inv file1 < usedLogFiles \inv file2 \implies
\\ \t2   newestLogTimeC~ (logFiles~ file1) \leq oldestLogTimeC~
(logFiles~ file2) 
\also
      usedLogFiles \neq \langle \rangle 
\\ \t1 \implies (\forall file : LOGFILEINDEX | file \in \ran
      (front~usedLogFiles) @ \# (logFiles~ file) = maxLogFileEntries)
 
\also
        usedLogFiles \neq \langle \rangle 
\\ \t2  \land currentLogFile = last~ usedLogFiles
\\ \t2  \land numberLogEntries = (\# usedLogFiles -1) *
                maxLogFileEntries 
                + \# (logFiles~ currentLogFile)
\\ \t1 \lor
\\      usedLogFiles = \langle \rangle 
        \land numberLogEntries = 0
\end{schema}
\begin{Zcomment}
\item
The $freeLogFiles$ are exactly those which are empty.
\item
The $usedLogFiles$ is a sequence of log files which are non-empty.
\item
The $usedLogFiles$ are ordered such that the oldest entries appear in
the first log file in the sequence.
\item
All but the last $usedLogFiles$ are filled to their maximum capacity.
\item
The $numberLogEntries$ is completely derived and is maintained for
convenience. 
\end{Zcomment}

%----------------------------------------------------------------------
\section{Key Store}
%----------------------------------------------------------------------

\begin{traceunit}{FD.KeyStore.State}
\traceto{FS.KeyStore.State}
\end{traceunit}


TIS maintains a key store, this is managed by the Crypto Library. 
It contains all Issuer keys relevant to
its function. This will include known CAs, AAs and its own key. 

The only private key part held will be for TIS's own key. Having a
private key within the set of keys indicates that the TIS knows who it is.

TIS will generate its key at the first start-up and request an Id
certificate from a CA. This activity is not modelled here and will not
be implemented. We model the private part of the TIS key as
$theTISKey$. This will be inserted into the keystore at enrolment. The
private part of the TIS key is used subsequently to sign authorisation
certificates.  

Only one public key is held for each Issuer.

\begin{schema}{KeyStoreC}
        keys : \finset KeyPart
\\      theTISKey : KEYPART
\also
        keyMatchingIssuer : USERID \fun \Optional KEYPART 
\\      privateKey : \Optional KeyPart
\where
        \{ key : keys | key.keyType = private @ key.keyData \}
\subseteq \{ theTISKey \}         
\also
        privateKey \neq \Nil  \implies 
\\ \t1  (\exists ownPub : keys @ ownPub.keyType = public 
\\ \t2          \land ownPub.keyOwner = (\The privateKey).keyOwner)
\also
        \# \{ key : keys | key.keyType = public \} =
        \# \{ key : keys | key.keyType = public @ key.keyOwner \}
\also
        keyMatchingIssuer = (USERID \cross \{ \emptyset \})
        \oplus \{ key : keys | key.keyType = public @  key.keyOwner.id
        \mapsto \{ key.keyData \} \}   
\\
        privateKey = \{ key : keys | key.keyType = private \}      
\end{schema}
\begin{Zcomment}
\item
The Crypto Library provides facilities to query information, these are
modelled by $keysMatchingIssuer$ and $privateKeys$.
\item
$keysMatchingIssuer$ and $privateKeys$ are completely defined by invarients.
\end{Zcomment}

%----------------------------------------------------------------------
\section{Certificate Store}
%----------------------------------------------------------------------

\begin{traceunit}{FD.CertificateStore.State}

\end{traceunit}

TIS issues certificates, these certificates have a unique identifier,
which is composed of the unique $USERID$ identification given to TIS
and a serial number.

TIS must maintain knowlege of the serial numbers already issued to
ensure that new certificates are issued with a unique serial number.

\begin{schema}{CertificateStore}
        nextSerialNumber : \nat
\end{schema} 

%--------------------------------------------------------------------
\section{System Statistics}
%--------------------------------------------------------------------

\begin{traceunit}{FD.Stats.State}
\traceto{FS.Stats.State}
\end{traceunit}

The system statistics recorded are as defined in the formal
specification \cite{FS}.

TIS keeps track of the number of times that a entry to the
enclave has been attempted (and denied) and the number of times it has succeeded. It
also records the number of times that a biometric comparison has been
made (and failed) and the number of times it succeeded.

By retaining these statistics it is possible for the performance of
the system to be monitored.

\begin{schema}{StatsC} 
        successEntryC : \nat
\\      failEntryC : \nat
\\      successBioC : \nat
\\      failBioC : \nat
\end{schema}

%--------------------------------------------------------------------
\section{Administration}
%--------------------------------------------------------------------

\begin{traceunit}{FD.Admin.State}
\traceto{FS.Admin.State}
\end{traceunit}

This component of TIS is not refined from the specification.

In addition to its role of authorising entry to the enclave, TIS 
supports a number of administrative operations. 

\begin{itemize}
\item ArchiveLog - writes the archive log to floppy and truncates the
internally held archive log.
\item UpdateConfiguration - accepts new configuration data from a floppy. 
\item OverrideDoorLock - unlocks the enclave door. 
\item Shutdown - stops TIS, leaving the protected entry to the enclave secure.
\end{itemize}

%%unchecked
\begin{syntax}
        ADMINOP ::=  archiveLog | updateConfigData |
        overrideLock | shutdownOp 
\end{syntax}
\begin{Zcomment}
\item Definition repeated from Formal Specification \cite{FS}
\end{Zcomment}

Only users with administrator privileges can make use of the TIS to
perform administrative functions. There are a number of different
administrator privileges that may be held.

%%unchecked
\begin{zed}
        ADMINPRIVILEGE == \{ guard, auditManager, securityOfficer \}
\end{zed}
\begin{Zcomment}
\item Definition repeated from Formal Specification \cite{FS}
\end{Zcomment}

The role held by the administrator will determine the operations
available to the administrator. For security reasons an administrator
can only hold one role.

\begin{schema}{AdminC}
        rolePresentC : \Optional ADMINPRIVILEGE
\\      availableOpsC : \power ADMINOP
\\      currentAdminOpC : \Optional ADMINOP
\where
        rolePresentC = \Nil \implies availableOpsC = \emptyset 
\\      (rolePresentC \neq \Nil \land \The rolePresentC = guard) \implies availableOpsC = 
        \{ overrideLock \}
\\      (rolePresentC \neq \Nil \land \The rolePresentC = auditManager) \implies availableOpsC = 
        \{ archiveLog \}
\\      (rolePresentC \neq \Nil \land \The rolePresentC = securityOfficer) \implies availableOpsC = 
        \{ updateConfigData, shutdownOp \}
\\      currentAdminOpC \neq \Nil \implies 
\\ \t1 (\The currentAdminOpC \in availableOpsC \land rolePresentC \neq \Nil )
\end{schema}
\begin{Zcomment}
\item
The $availableOpsC$ are completely determined by the roles present and
will be implemented using a constant table.
\end{Zcomment}

In order to perform an administrative operation an administrator must
be present. Presence will be determined by an appropriate token being
present in the administrator's card reader.

%-----------------------------------------------------------------------
\section{Real World Entities}
%-----------------------------------------------------------------------
%The real world entities are internally modelled identically to their external model.

\begin{traceunit}{FD.RealWorld.State}
\traceto{FS.RealWorld.State}
\end{traceunit}


The latch is allowed to be in two states:
$locked$ and $unlocked$.
When the latch is unlocked,
$latchTimeoutC$ will be set to the time at which the lock must again be $locked$.


The alarm is similar to the latch, 
in that it has a $silent$, and $alarming$,
with an $alarmTimeoutC$.
Once the door and latch move into a potentially insecure state
(door $open$ and latch $locked$)
then the $alarmTimeoutC$ is set to the time at which the alarm will sound.

Within the implementation the $currentLatchC$ and $doorAlarmC$ will
be explicitly calculated athough they can be entirely derived.

\begin{schema}{DoorLatchAlarmC}
	currentTimeC: TIME
\\	currentDoorC: DOOR
\\	currentLatchC: LATCH
\\	doorAlarmC: ALARM
\\	latchTimeoutC: TIME
\\	alarmTimeoutC: TIME
\end{schema}


The ID Station holds internal representations of all of the Real World,
plus its own data.
It holds separate indications of the presence of input in the real world peripherals
of the User Token, Admin Token, Fingerprint reader, and Floppy disk.
This is so that once the input has been read,
and the card, finger or disk removed,
the ID Station can continue to know what the value was,
even if it later detects that the real world entity has been removed.

\begin{schema}{UserTokenC}
	currentUserTokenC: TOKENTRYC
\\	userTokenPresenceC: PRESENCE
\end{schema}

\begin{schema}{AdminTokenC}
	currentAdminTokenC: TOKENTRYC
\\	adminTokenPresenceC: PRESENCE
\end{schema}

The core TIS does not need to know what the current fingerprint is
since it is always read directly from the real world by the 
Biometrics Library.

\begin{schema}{FingerC}
	fingerPresenceC: PRESENCE
\end{schema}

The core TIS does not need to preserve the value of the current keyed
data since it is always read directly from the real world and the
information does not need to be persistent.

\begin{schema}{KeyboardC}
      keyedDataPresenceC : PRESENCE
\end{schema}  


We need to retain an internal view of the last data written to the
floppy as well as the current data on the floppy, this is because we
need to check that writing to floppy works when we archive the log.

\begin{schema}{FloppyC}
	currentFloppyC: FLOPPYC
\\      writtenFloppyC: FLOPPYC
\\	floppyPresenceC: PRESENCE
\end{schema}

The ID Station screen within the enclave may display many pieces of
information. The majority of this data will be determined by state
invarients.
In addition to those identified in the specification we now identify
the alarm states as being necessary display elements.

\begin{schema}{ScreenC}
        screenStatsC : SCREENTEXTC
\\      screenMsgC : SCREENTEXTC
\\      screenConfigC : SCREENTEXTC
\\      screenDoorAlarm : SCREENTEXTC
\\      screenLogAlarm : SCREENTEXTC 
\end{schema}

%-----------------------------------------------------------------------
\section{Internal State}
%-----------------------------------------------------------------------

\begin{traceunit}{FD.Internal.State}
\traceto{FD.Internal.State}
\end{traceunit}


$STATUS$ and $ENCLAVESTATUS$ are a purely internal records of the progress through 
processing. $STATUS$ tracks progress through user entry, while
$ENCLAVESTATUS$ tracks progress through all activities performed
within the enclave. 

%%unchecked
\begin{zed}
	STATUS ::= quiescent | 
\also \t2          gotUserToken |  waitingFinger | gotFinger | waitingUpdateToken | waitingEntry |
\\ \t2          waitingRemoveTokenSuccess | waitingRemoveTokenFail 
\also
        ENCLAVESTATUS ::= notEnrolled | waitingEnrol | waitingEndEnrol |
\also \t2          enclaveQuiescent |
\also \t2          gotAdminToken | waitingRemoveAdminTokenFail |
waitingStartAdminOp | waitingFinishAdminOp |
\also \t2          shutdown 
\end{zed}
\begin{Zcomment}
\item Definitions repeated from Formal Specification \cite{FS}
\end{Zcomment}

The states $quiescent$ and $enclaveQuiescent$ represent the enclave
interface and the user entry interface being quiescent.

The states $gotUserToken$, .. 
$waitingRemoveTokenFail$ are all associated with the process of user
authentication and entry. These are described futher in Section \ref{sec:userEntry}.

The states $notEnrolled$, .. $waitingEnrolEnd$ reflect
enrolment activity that must be performed before TIS can offer any of
its normal operations. Once the TIS is successfully enrolled it
becomes $quiescent$.

The states $gotAdminToken$, .. $waitingFinishAdminOp$ 
reflect activity at the TIS console relating to administrator use of TIS. 

The state $shutdown$ models the system when it is shutdown.

There are two timeouts held internally, one of these controls 
the system wait for the user to remove their token before opening the
door in a successful user entry scenario. 
The other controls the system wait for the user to provide a good
fingerprint for verification before giving up on this part of the
authentication process.

\begin{schema}{InternalC}
        statusC : STATUS
\\      enclaveStatusC : ENCLAVESTATUS
\\      tokenRemovalTimeoutC : TIME
\\      fingerTimeout : TIME
\end{schema}


%-----------------------------------------------------------------------
\section{The whole Token ID Station}
%-----------------------------------------------------------------------

\begin{traceunit}{FD.TIS.State}
\traceto{FS.TIS.State}
\end{traceunit}


The whole Token ID Station is constructed from combining the described
state components. 

In addition there is a display outside the enclave and and screen
within the enclave. The ID Station screen within the enclave may
display many pieces of 
information. The majority of this data will be determined by state invariants.

If the authentication protocol has moved on to requesting a fingerprint,
then the User Token will have passed its validation checks.

Similarly if the system considers there to be an administrator present
then the Admin Token will have passed its validation checks.

Once the ID station has been enrolled it has a private key, its own key.

TIS is only ever in the two states $waitingStartAdminOp$ or
$waitingFinishAdminOp$ when then there is a current admin operation in
progress. For single phase operations the state
$waitingFinishAdminOp$ is not used.

TIS will only read the Admin Token to log on an administrator if there
is not an administrator role currently present.
 
\begin{schema}{IDStationC}
	UserTokenC
\\	AdminTokenC
\\	FingerC
\\	DoorLatchAlarmC
\\      FloppyC
\\      KeyboardC
\\      ConfigC
\\      StatsC
\\      KeyStoreC
\\      CertificateStore
\\      AdminC
\\      AuditLogC
\\      InternalC
\also
	currentDisplayC: DISPLAYMESSAGE
\\      currentScreenC: ScreenC
\where
	statusC \in \{~ gotFinger, waitingFinger, waitingUpdateToken, waitingEntry~\} \implies
\\ \t1		 (( \exists ValidTokenC @ 
			goodTC(\theta ValidTokenC) = currentUserTokenC)
\\ \t2  \lor ( \exists TokenWithValidAuthC @ 
			goodTC(\theta TokenWithValidAuthC) = currentUserTokenC))
\also
        rolePresentC \neq \Nil \implies       
\\ \t1		 ( \exists TokenWithValidAuthC @ 
			goodTC(\theta TokenWithValidAuthC) =
 currentAdminTokenC )
\also
        enclaveStatusC \notin \{~notEnrolled, waitingEnrol, waitingEndEnrol ~\} \implies       
\\ \t1         \# \{ key : keys | key.keyType = private \} = 1
\also
        enclaveStatusC \in \{~ waitingStartAdminOp, waitingFinishAdminOp ~\} \iff currentAdminOpC \neq \Nil
\also
       (currentAdminOpC \neq \Nil \land \The currentAdminOpC \in \{~
shutdownOp, overrideLock ~\}) 
\\ \t2          \implies enclaveStatusC = waitingStartAdminOp
\also
        enclaveStatusC = gotAdminToken \implies rolePresentC = \Nil
\also   % invarients that define the screen
        currentScreenC.screenStatsC = displayStatsC (\theta
        StatsC)
\\      currentScreenC.screenConfigC = displayConfigDataC (\theta ConfigC)
\\      currentScreenC.screenDoorAlarm = displayAlarm~ doorAlarmC
\\      currentScreenC.screenLogAlarm = displayAlarm~ auditAlarmC
\end{schema}
\begin{Zcomment}
\item
Note that the token may not still be current since time will have
moved on since the checks were performed.
\item
Operations that can be performed in a single phase do not result in
TIS entering the state $waitingFinishAdminOp$ as they are finished
when they are started. 
\item
TIS only enters the state $gotAdminToken$ when there is no
administrator present.
\item
Invarients define many of the screen components.
\end{Zcomment}

