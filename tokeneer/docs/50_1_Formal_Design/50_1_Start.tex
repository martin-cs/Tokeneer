
%=========================================================================
\chapter{The Initial System and Startup}
\label{sec:Start}
%=========================================================================
\section{The Initial System}
%----------------------------------

\begin{traceunit}{FD.TIS.InitIDStation}
\traceto{FS.TIS.InitIDStation}
\end{traceunit}

After initial installation the system has the following properties
\begin{itemize}
\item
an empty key store, which means it is unable to authorise entry to anyone;
\item
default configuration data, which does not permit entry to anyone;
\item
the door latched;
\item
an empty audit log;
\item
the internal times all set to zero (a time before the current time).
\end{itemize}

The door is assumed closed at initialisation, this ensures that the alarm
will not sound before the first time that data is polled.

\begin{schema}{InitDoorLatchAlarmC}
        DoorLatchAlarmC
\where
	currentTimeC = zeroTime
\\	currentDoorC = closed
\\	latchTimeoutC = zeroTime
\\	alarmTimeoutC = zeroTime
\\      doorAlarmC = silent
\\      currentLatchC = locked
\end{schema}

There are no keys held by the system.

\begin{schema}{InitKeyStoreC}
        KeyStoreC
\where
        keys = \emptyset
\end{schema}

The initial certificate store has the 0 as the available serial number.

\begin{schema}{InitCertificateStore}
        CertificateStore
\where
        nextSerialNumber = 0
\end{schema}        


This default configuration assumes the lowest classification possible
for the enclave. This ensures that it does not give inadvertently
high clearance to the authorisation certificate. The parameters that
define the $authPeriod$
and $entryPeriod$ functions are set to enable entry into the enclave
to re-configure the TIS. This configuration will
allow Auth Certificates to be generated with a validity of 2 hours
from the point of issue.
\begin{schema}{InitConfigC}
        ConfigC
\where
	alarmSilentDurationC = 10
\\      latchUnlockDurationC = 150
\\      tokenRemovalDurationC = 100
\\      fingerWaitDuration = 100
\\      enclaveClearanceC = unmarked
\\      minEntryClass = unmarked
\also   
        maxAuthDuration = 72000
\\      accessPolicy = allHours
\\      systemMaxFar = 1000
\end{schema}
\begin{Zcomment}
\item 
The initial values of $workingHoursStart$, $workingHoursEnd$ will not
impact the entry or authorisation periods so are not defined here,
they are free to be implemented with any value.
\end{Zcomment} 

Initially no administrator is logged on and no administator operations
are taking place.
\begin{schema}{InitAdminC}
        AdminC
\where
        rolePresentC = \Nil
\\      currentAdminOpC = \Nil
\end{schema}

Initially the statistics are set to zero, indicating no use of the
system to date.
\begin{schema}{InitStatsC}
        StatsC
\where
        successEntryC = 0
\\      failEntryC = 0
\\      successBioC = 0
\\      failBioC = 0
\end{schema}        

The initial audit Log is empty and there is no audit alarm.

\begin{schema}{InitAuditLogC}
        AuditLogC
\where
        logFiles = LOGFILEINDEX \cross \{ \emptyset \}
\\      auditAlarmC = silent
\end{schema}        

Initially the internal state
is $notEnrolled$.

\begin{schema}{InitInternalC}
        InternalC
\where
	enclaveStatusC = notEnrolled
\\      statusC = quiescent
\end{schema}
\begin{Zcomment}
\item
In the above states the timeouts $fingerTimeout$ and
$tokenRemovalTimeoutC$ are not used so their values are not
important. The implementation is free to set their initial value to
any valid value. 
\end{Zcomment}

Entities that model the real world and are polled and have no security
implications are not set 
at initialisation, these will be updated at the first poll of the real
world entities.

Initially the screen and the display are clear. 

\begin{schema}{InitIDStationC}
        IDStationC
\also
	InitDoorLatchAlarmC
\\	InitConfigC
\\      InitKeyStoreC
\\      InitStatsC
\\      InitAuditLogC
\\      InitAdminC
\\      InitInternalC
\\      InitCertificateStore
\where
        currentScreenC.screenMsgC = clearC
\also
	currentDisplayC = blank
\end{schema}

%-------------------------------------------------------------------------
\section{Starting the ID Station}
%-------------------------------------------------------------------------
\begin{traceunit}{FD.TIS.TISStartup}
\traceto{FS.TIS.TISStartup}
\end{traceunit}


We assume that some of the state within TIS is persistent through
shutdown and some is not. 
The persistent items are $ConfigC$, $KeyStoreC$, $CertificateStore$  and $AuditLogC$ all other state
components are set at startup. Those values that are polled can take
any valid value, we assume for simplicity that they remain unchanged.

\begin{schema}{StartContextC}
        \Delta IDStationC
\\      RealWorldChangesC
\also
\\      \Xi ConfigC
\\      \Xi KeyStoreC
\also
	InitDoorLatchAlarmC'
\\      InitStatsC'
\\      InitAdminC'
\\      AddElementsToLogC
\\      LogChangeC
\also
        \Xi UserTokenC  
\\      \Xi AdminTokenC 
\\      \Xi FingerC
\\      \Xi FloppyC 
\\      \Xi KeyboardC 
\where
        auditTypes~newElements? \subseteq STARTUP\_ELEMENTS \cup USER\_INDEPENDENT\_ELEMENTS
\end{schema}

In the case that TIS does not have any private keys in the $KeyStoreC$ 
the ID station is assumed to require enrolment.

\begin{schema}{StartNonEnrolledStationC}
        StartContextC
\also
        InitCertificateStore'
\where 
        privateKey = \Nil
\also
        currentScreenC'.screenMsgC = clearC
\also
	currentDisplayC' = blank
\\	enclaveStatusC' = notEnrolled
\\      statusC' = quiescent
\also
        auditTypes~ newElements? \cap STARTUP\_ELEMENTS = 
        \{ startUnenrolledTISElement \} 
\also
        \exists_1 element : AuditC @ element \in newElements? 
\\ \t1  \land element.elementId = startUnenrolledTISElement
\\ \t1  \land element.logTime \in nowC \upto nowC'
\\ \t1  \land element.user = noUser
\\ \t1  \land element.severity = information
\\ \t1  \land element.description = noDescription
\end{schema}

In the case that TIS does have a private key the ID station is
assumed to have been previously enrolled.

\begin{schema}{StartEnrolledStationC}
        StartContextC
\also
        \Xi CertificateStore
\where 
        privateKey \neq \Nil
\also
        currentScreenC'.screenMsgC = welcomeAdminC
\also
	currentDisplayC' = welcome
\\	enclaveStatusC' = enclaveQuiescent
\\      statusC' = quiescent
\also
        auditTypes~ newElements? \cap STARTUP\_ELEMENTS = 
        \{ startEnrolledTISElement \} 
\also
        \exists_1 element : AuditC @ element \in newElements? 
\\ \t1  \land element.elementId = startEnrolledTISElement
\\ \t1  \land element.logTime \in nowC \upto nowC'
\\ \t1  \land element.user = noUser
\\ \t1  \land element.severity = information
\\ \t1  \land element.description = noDescription
\end{schema}

The complete startup operation is given by:

\begin{zed}
        TISStartupC \defs (StartEnrolledStationC \lor
        StartNonEnrolledStationC ) \hide (newElements?)
\end{zed}







