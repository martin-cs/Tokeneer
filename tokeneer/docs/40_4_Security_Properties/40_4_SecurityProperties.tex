%==========================================================================
\chapter{Introduction}
%==========================================================================
In order to demonstrate that developing highly secure systems to the
level of rigour required by the higher assurance levels of the Common
Criteria is possible, the NSA has asked Praxis High Integrity Systems to
undertake a research project to develop part of an existing secure
system (the Tokeneer System) in accordance with their high-integrity
development process. This development work will then be used to
show the security community that is is possible to develop secure
systems rigorously in a cost effective manner.

This document is the formal statement of the key security properties
the Token ID Station (TIS) shall possess,
written using the Z notation.

%--------------------------------------------------------------------------
\section{Structure of this Document}
%--------------------------------------------------------------------------
This document should be read in conjunction with the formal specification,
\cite{FS}.

Section \ref{sec:SecProp} states the key security properties as theorems
on the formal specification.

Section \ref{sec:Arg} gives mathematical arguments that the formal specification
given in \cite{FS} does in fact possess each of the properties stated.

%==========================================================================
\chapter{Security Properties}
\label{chap:SecProp}
%==========================================================================
%%ignore \shows
\def\The{the~}%
We define a general operation, 
representing all the calculation and decisions,
but excluding the Polling of the peripherals and the Updating of the peripherals.
Many of the security properties refer to properties that hold during one execution of this
general operation.

\begin{zed}
	TISOp \defs (TISEnrolOp
\\ \t1			\lor TISUserEntryOp
\\ \t1			\lor TISAdminLogon
\\ \t1			\lor TISStartAdminOp
\\ \t1			\lor TISAdminOp
\\ \t1			\lor TISAdminLogout
\\ \t1			\lor TISIdle) 
\\ \t2				\land LogChange
\end{zed}
%--------------------------------------------------------------------------
\section{Security Properties}
\label{sec:SecProp}
%--------------------------------------------------------------------------
%--------------------------------------------------------------------------
\subsection{Property 1: Unlock with Token}
\label{sec:UnlockProp}
%--------------------------------------------------------------------------
{\em If the latch is unlocked by the TIS,
then the TIS must be in possession of either a User Token or an Admin Token.
The User Token must either have a valid Authorisation Certificate,
or must have valid ID, Privilege, and I\&A Certificates,
together with
a template that allowed the TIS to successfully validate the user's fingerprint.
Or, if the User Token does not meet this,
the Admin Token must have a valid Authorisation Certificate,
with role of ``guard''.
}

As the property refers to the real world ($latch$),
we must look at the combined effect of carrying out a TIS calculation
and then updating the world.
Hence we define:

\begin{zed}
	TISOpThenUpdate \defs TISOp \semi TISUpdate
\end{zed}

We also make statements about validity in the absence of currency checks.
We therefore define the following two schemas,
which are direct copies of the schemas
$UserTokenWithOKAuthCert$ and
$UserTokenOK$,
but with the currency checks removed.

\begin{schema}{UserTokenWithOKAuthCertNoCurrencyCheck}
        KeyStore
\\      UserToken
\\      currentTime : TIME
\where
        currentUserToken \in \ran goodT
\\	\exists TokenWithValidAuth @ 
\\ \t1		(
		goodT(\theta TokenWithValidAuth) = currentUserToken
\\ \t1          \land (\exists AuthCert @ \theta AuthCert = \The
authCert \land AuthCertOK )  
		)
\end{schema}


\begin{schema}{UserTokenOKNoCurrencyCheck}
        KeyStore
\\      UserToken
\\      currentTime : TIME
\where
 	currentUserToken \in \ran goodT
\\	\exists CurrentToken @ 
\\ \t1		(
		goodT(\theta ValidToken) = currentUserToken
\\ \t1          \land (\exists IDCert @ \theta IDCert = idCert \land CertOK )
\\ \t1          \land (\exists PrivCert @ \theta PrivCert = privCert
\land CertOK )
\\ \t1          \land (\exists IandACert @ \theta IandACert =
iandACert \land CertOK )  
                )
\end{schema}

\begin{zed}
%%\forall
\Delta IDStation; \Delta RealWorld |
\\ \t1	TISOpThenUpdate
\\ \t1	\land latch = locked \land latch' = unlocked
%%@
\\ \shows
\\ \t1	(\exists ValidToken @
			goodT(\theta ValidToken) = currentUserToken
\\ \t2		\land UserTokenOKNoCurrencyCheck
\\ \t2		\land FingerOK
		)
\\ \t1	\lor
\\ \t1	(\exists TokenWithValidAuth @
			goodT(\theta TokenWithValidAuth) = currentUserToken
\\ \t2		\land UserTokenWithOKAuthCertNoCurrencyCheck
		)
\\ \t1 	\lor
\\ \t1  	(\exists ValidToken @
			goodT(\theta ValidToken) = currentAdminToken
\\ \t2			\land authCert \neq \emptyset 
				\land (\The authCert).role = guard
		)
\end{zed}

%--------------------------------------------------------------------------
\subsection{Property 2: Unlock at allowed time}
\label{sec:UnlockTimeProp}
%--------------------------------------------------------------------------
{\em If the latch is unlocked automatically by the TIS,
then the current time must be close to being within the allowed entry period defined
for the User requesting access.
}

``close'' needs to be defined,
but is intended to allow a period of grace between checking that access is allowed and
actually unlocking the latch.
``Automatically'' refers to the latch being unlocked by the system in response to a user token
insertion,
rather than being manually unlocked by the guard.

\begin{zed}
%%\forall
\Delta IDStation; \Delta RealWorld |
\\ \t1	TISOpThenUpdate
\\ \t1	\land latch = locked \land latch' = unlocked
\\ \t1	\land adminTokenPresence = absent
%%@
\\ \shows
\\ \t1	(\exists ValidToken @ goodT(\theta ValidToken) = currentUserToken
\\ \t2	\land
		(\exists recentTime: timesRecentTo~currentTime @
\\ \t3			recentTime \in entryPeriod~privCert.role~privCert.clearance.class
		)
	)
\end{zed}

where we define a function $timesRecentTo$
that gives a set of times close to a given time.
We define it here loosely.
\begin{axdef}
	timesRecentTo: TIME \fun \power TIME
\where
	\forall t:TIME @ t \in timesRecentTo~t
\end{axdef}
%--------------------------------------------------------------------------
\subsection{Property 3: Alarm when insecure}
\label{sec:AlarmProp}
%--------------------------------------------------------------------------
{\em An alarm will be raised whenever the door/latch is insecure.
}

``insecure'' is defined to mean the latch is locked,
the door is open,
and too much time has passed since the last explicit request to lock the latch.

There are two places in which real world updates occur:
\begin{zed}
	Update \defs 
\\ \t1		TISEarlyUpdate \lor TISUpdate
\end{zed}
\begin{zed}
%%\forall
Update |
\\ \t1	latch' = locked 
\\ \t1	\land currentDoor' = open
\\ \t1	\land currentTime' \geq alarmTimeout
%%@
\\ \shows
\\ \t1	alarm' = alarming
\end{zed}

%--------------------------------------------------------------------------
\subsection{Property 4: No loss of audit}
\label{sec:AuditLossProp}
%--------------------------------------------------------------------------
{\em No audit data is lost without an audit alarm being raised.
}

\begin{zed}
%%\forall
TISOp | auditAlarm = auditAlarm' = silent
%%@
\\ \shows
\\ \t1	auditLog \cup 	(\IF currentFloppy \in \ran auditFile
			\THEN auditFile \inv currentFloppy
			\ELSE \emptyset
			)
		\subseteq
\\ \t2	auditLog' \cup 	(\IF currentFloppy' \in \ran auditFile
			\THEN auditFile \inv currentFloppy'
			\ELSE \emptyset
			)
\end{zed}
%--------------------------------------------------------------------------
\subsection{Property 5: Audit records are consistent}
\label{sec:AuditLinksProp}
%--------------------------------------------------------------------------
The presence of an audit record of one type
(e.g. recording the unlocking of the latch)
will always be preceded by certain other audit records
(e.g. recording the successful checking of certificates, fingerprints, etc.)

Such a property would need to be defined in detail,
explaining the data relationship rules exactly for each case.
This has not been done.
%--------------------------------------------------------------------------
\subsection{Property 6: Configuration/floppy changed by admin}
\label{sec:ConfigProp}
%--------------------------------------------------------------------------
The configuration data will be changed,
or information written to the floppy,
only if there is an Admin person logged on to the TIS.

\begin{zed}
%%\forall
TISOp | adminTokenPresence = absent
%%@
\\ \shows
\\ \t1	\Xi Config \land \Xi Floppy
\end{zed}








%--------------------------------------------------------------------------
\section{Arguments that Security Properties hold}
\label{sec:Arg}
%--------------------------------------------------------------------------
We present informal arguments that each of the security properties is indeed a property
of the system as specified.


Note that $TISOp$ has signature
\[
	\Delta IDStation
\\	\Xi TISControlledRealWorld
\\	\Delta TISMonitoredRealWorld
\]
and $TISUpdate$ has signature
\[
	\Xi IDStation
\\	\Delta TISControlledRealWorld
\\	\Delta TISMonitoredRealWorld
\]
	
%--------------------------------------------------------------------------
\subsection{Property 1: Unlock with Token}
\label{sec:UnlockArg}
%--------------------------------------------------------------------------
Restating the property:

\begin{zed}
%%\forall
\Delta IDStation; \Delta RealWorld |
\\ \t1	TISOpThenUpdate
\\ \t1	\land latch = locked \land latch' = unlocked
\\ \t1	\land (latch = currentLatch)
%%@
\\ \shows
\\ \t1	(\exists ValidToken @
			goodT(\theta ValidToken) = currentUserToken
\\ \t2		\land UserTokenOKNoCurrencyCheck
\\ \t2		\land FingerOK
		)
\\ \t1	\lor
\\ \t1	(\exists TokenWithValidAuth @
			goodT(\theta TokenWithValidAuth) = currentUserToken
\\ \t2		\land UserTokenWithOKAuthCertNoCurrencyCheck
		)
\\ \t1 	\lor
\\ \t1  	(\exists ValidToken @
			goodT(\theta ValidToken) = currentAdminToken
\\ \t2			\land authCert \neq \emptyset 
				\land (\The authCert).role = guard
		)
\end{zed}
The hypothesis considers only the combination of the general operation and
the standard update step.
This is justified because $latch$ can only change in an $Update$,
which is always either
\[
	TISOp \semi TISUpdate
\]
or
\[
	Poll \semi TISEarlyUpdate
\]
The second of these can only result in an increase in $currentTime$,
which means that the only change to $latch$ possible is to become $locked$.
This is not the change we are investigating here.

It is therefore reasonable to limit our hypothesis to the case
$TISOp \semi TISUpdate$.

We have started by adding the predicate $latch = currentLatch$
to the hypothosis.
This is justified because we are concerned only with the case when
the latch is entirely under the control of the TIS,
and has been to date.
We can therefore assume that at the beginning of this operation
the TIS's view of the position of the latch
($currentLatch$) agrees with the actual state of the latch ($latch$).

We will refer to the intermediate state by double-dash, the before-state
by no dashes, and the after-state by a single dash.

Taking the hypothesis $latch' = unlocked$,
we deduce that $currentLatch''=unlocked$,
from the behaviour of $TISUpdate$.
$currentLatch = locked$ from the predicates in the hypothesis.

So we have a change of $currentLatch$.
This state of $currentLatch$ is entirely defined in the invariant to
$DoorLatchAlarm$ by $currentTime$ and $latchTimeout$.
A change can only happen in $TISOp$, due to $\Xi IDStation$ in $TISUpdate$.

There are three components of $TISOp$ that do not have 
$\Xi DoorLatchAlarm$:
$UnlockDoorOK$,
$ShutdownOK$,
and $OverrideDoorLockOK$.

We can ignore $ShutdownOK$ because it involves a change to $locked$,
not $unlocked$.

{\bf UnlockDoorOK}

$status = waitingRemoveTokenSuccess$

The sequence of states that must have been passed through is therefore either:

$quiescent$ to $gotUserToken$ to $waitingFinger$ to $gotFinger$ to 
$waitingUpdateToken$ to $waitingEntry$ to $waitingRemoveTokenSuccess$

or

$quiescent$ to $gotUserToken$ to 
$waitingEntry$ to $waitingRemoveTokenSuccess$

In the first, $ValidToken$ and $UserTokenOK$ are assured in passing 
to the $waitingFinger$
state,
and $FingerOK$ is assured in passing to the $waitingUpdateToken$ state.
This proves the first branch.

In the second, $TokenWithValidAuth$ and $UserTokenWithOKAuthCert$ are both
assured in passing to the $waitingEntry$ state.
This proves the first branch of the theorem.

{\bf OverrideDoorLockOK}

$enclaveStatus = waitingStartAdminOp$

and

$\The currentAdminOp = overrideLock$.

This means that the system must have passed through the following states:

$enclaveQuiescent, role=nil$ to $gotAdminToken$
to $enclaveQuiescent, roll \neq nil$ to $waitingStartAdminOp$

In passing to the $enclaveQuiescent, role \neq nil$ state
the $rolePresent$ is set from the $role$ in the $authCert$,
and the Admin Token is checked for validity
(and in particular, checked for being present).

In passing to the $waitingStartAdminOp$ state the check on the validity
of the requested operation ensures that
$currentAdminOp \in keyedOp \inv \limg availableOps \rimg$,
and in turn the value of $availableOps$ is tied to the value of the
$rolePresent$ by a state invariant.
The state invariant ensures that the operation to override the door lock
(the operation we must be considering here)
is only available to the $guard$.
Therefore, as we know that the hypothesis implies that the operation being
carried out is $OverrideDoorLockOK$,
we know that $\The currentAdminOp = overrideLock$,
so we know that the $rolePresent$ is a $guard$,
and hence the role in the $authCert$ is a $guard$, as required.
%--------------------------------------------------------------------------
\subsection{Property 2: Unlock at allowed time}
\label{sec:UnlockTimeArg}
%--------------------------------------------------------------------------
Restating the property:

\begin{zed}
%%\forall
\Delta IDStation; \Delta RealWorld |
\\ \t1	TISOpThenUpdate
\\ \t1	\land latch = locked \land latch' = unlocked
\\ \t1	\land adminTokenPresence = absent
\\ \t1	\land (latch = currentLatch)
\\ \t1	\land (\forall t:TIME @
\\ \t2		((t-tokenRemovalDuration) \upto (t + tokenRemovalDuration)) \cap TIME \subseteq timesRecentTo~t)
%%@
\\ \shows
\\ \t1	(\exists ValidToken @ goodT(\theta ValidToken) = currentUserToken
\\ \t2	\land
		(\exists recentTime: timesRecentTo~currentTime @
\\ \t3			recentTime \in entryPeriod~privCert.role~privCert.clearance.class)
	)
\end{zed}

As in the proof of Property 1,
we consider the combination of one general operation and one update.

We add the assumption that all times within $tokenRemovalDuration$ of a given
time are considered to be ``recent'' to that time.

We can follow the same proof as for Property 1,
and with the additional hypothesis of $adminTokenPresence = absent$
we can exclude the $OverrideDoorLockOK$ operation,
as intended.

Following the sequence of state changes,
we find that in passing to the $waitingRemoveTokenSuccess$ state
we carry out the $EntryOK$ operation,
which checks that the $currentTime$ is within the entry period defined
for the class and role given in the Privilege Cert:

$currentTime \in entryPeriod~privCert.role~privCert.clearance.class$.

However, this relates to the $currentTime$ given in passing {\em to}
this state, 
whereas the security property we are proving relates to the $currentTime$
{\em leaving} this state, to $quiescent$.

To deal with this, consider the state changes allowed subsequently.
$WaitingTokenRemoval$ allows the system to stay in this state,
but for no longer than $tokenRemovalDuration$. 
After this time, the system is forced into $waitingRemoveTokenFail$,
via $TokenRemovalTimeout$,
and ceases to be of concern.
The only other exit from this state is via $UnlockDoor$,
which actually carries out the change to $currentLatch$ we are investigating.

Therefore, we can show that the value of $currentTime$ when the latch is
unlocked cannot differ from the value of $currentTime$ when the validity
period is checked by more than $tokenRemovalDuration$.

The additional assumption on the proof is that our definition of ``recently''
is sufficiently broad to accommodate the time lag that our system
actually works with.

%--------------------------------------------------------------------------
\subsection{Property 3: Alarm when insecure}
\label{sec:AlarmArg}
%--------------------------------------------------------------------------
Restating the property:

\begin{zed}
%%\forall
Update |
\\ \t1	latch' = locked 
\\ \t1	\land currentDoor' = open
\\ \t1	\land currentTime' \geq alarmTimeout
%%@
\\ \shows
\\ \t1	alarm' = alarming
\end{zed}

This follows quite directly from the state invariant in $DoorLatchAlarm$.

$alarm'$ is set to $currentAlarm''$ (the only way it is ever set)
in both of the Update operations in the hypothesis,
and the invariant in the included $\Xi DoorLatchAlarm$ specifically
ties the conclusion to the remaining predicates in the hypothesis.

%--------------------------------------------------------------------------
\subsection{Property 4: No loss of audit}
\label{sec:AuditLossArg}
%--------------------------------------------------------------------------
Proof not done.
%--------------------------------------------------------------------------
\subsection{Property 5: Audit records are consistent}
\label{sec:AuditLinksArg}
%--------------------------------------------------------------------------
Proof not done.
%--------------------------------------------------------------------------
\subsection{Property 6: Configuration/floppy changed by admin}
\label{sec:ConfigArg}
%--------------------------------------------------------------------------
Proof not done.
%--------------------------------------------------------------------------
\section{Note on arguments}
\label{sec:NoteArg}
%--------------------------------------------------------------------------
Many of these arguments depend upon a sequence of state transitions.
The arguments presented argue that because the system is in a state with
a certain $status$,
then a specific series of states (different $status$es) must have been passed
through.
But because Z specifies the system in terms of atomic state transitions,
this argument is not fully justified.

It can be made fully justified by augmenting the state invariants
with the properties relied upon in each state.
It is possible to prove that these properties are established as the
system passes through its state transitions,
and because they are carried in the state invariant,
they are available to support the proofs.

We have not done this,
as we believe it will add little to the assurance of correctess,
and is very time consuming.
At higher levels of the CC assurance we would be required to carry out
more formal proofs,
in which case these modifications would be done.

