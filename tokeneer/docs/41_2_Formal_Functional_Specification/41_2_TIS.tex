%==========================================================================
\chapter{The Token ID Station}
\label{sec:TIS}
%==========================================================================
TIS maintains various state components, these are described and
elaborated within this section. 

%----------------------------------------------------------------------
\section{Configuration Data}
%----------------------------------------------------------------------

\begin{traceunit}{FS.ConfigData.State}
\end{traceunit}


$Config$ will be a structure with all the configuration data. 
Configuration data can only be modified by an administrator.
This data includes:
\begin{itemize}
\item 
Durations for internal timeouts. These effect how long the
system waits before raising an audible alarm, how long the system
leaves the door unlocked for, and how long the system waits for a
successful token removal.
\item
The security classification of the enclave.
\item
The rules for allocating validity periods to authorisation
certificates. These rules will depend on the time at which the
certificate was issued, and may also depend on the role of the user,
for example some roles may not be given use of the workstations ``out
of hours''. 
\item
The rules for allowing entry to the enclave. These rules will depend
on the role and security classification of the user, for example some
roles may not be given
access to the enclave ``out of hours''. 
\item
The minimum size of the audit
log before truncation may occur, $minPreservedLogSize$, which is configured to be within available file
store capacity of the TIS. A slightly smaller value, $alarmThresholdSize$,
sets the size of the audit log at which an alarm is raised, 
with the intention that the
audit log will be archived and cleared before the truncation occurs.
We acknowledge that there will be a system limit which affects the
largest size of log that can be guaranteed to be preserved.
\end{itemize}

\begin{axdef}
        maxSupportedLogSize : \nat
\end{axdef}


\begin{schema}{Config}
	alarmSilentDuration, latchUnlockDuration : TIME
\\      tokenRemovalDuration : TIME
\\      enclaveClearance : Clearance
\\      authPeriod : PRIVILEGE \fun TIME \fun \power TIME
\\      entryPeriod : PRIVILEGE \fun CLASS \fun \power TIME
\\      minPreservedLogSize : \nat
\\      alarmThresholdSize : \nat
\where
        alarmThresholdSize < minPreservedLogSize 
\\      minPreservedLogSize \leq maxSupportedLogSize     
\end{schema}

In practice there will be constraints on the authorisation
periods and entry periods. These constraints will be considered during
the design. There will also be constraints on the maximum FAR
permitted by the biometic verification. This will be introduced in
the design.


%--------------------------------------------------------------------------
\section{Audit Log}
%--------------------------------------------------------------------------

\begin{traceunit}{FS.AuditLog.State}
\end{traceunit}

TIS maintains an audit log. This is a log of all auditable events and
actions performed or monitored by TIS. The audit log will be used to
analyse the interactions with the TIS.

$Audit$ will be a structure for each audit record, 
recording at least time of event, type of event, user if known.
We use title case because we know this is a type we will be elaborating later.
\begin{zed}
	[ Audit ]
\end{zed}

Each audit element has associated with it a size, which may vary
between audit elements. The size of an audit log can be determined
from the size of its elements.

\begin{axdef}
        sizeElement : Audit \fun \nat
\\      sizeLog : \finset Audit \fun \nat
\where
        sizeLog \emptyset = 0
\\      \forall log : \finset Audit; entry : Audit @
\\ \t1  entry \in log \implies sizeLog~ log = sizeLog~ (log
\setminus \{ entry \}) + sizeElement~ entry
\end{axdef}

The Audit log consists of a number of $Audit$ elements. 
An audit error alarm will be raised if the audit log becomes full and
needs to be archived and cleared.

\begin{schema}{AuditLog}
        auditLog :\finset Audit
\\      auditAlarm : ALARM
\end{schema}

All audit elements have associated with them a timestamp so it is
possible to determine the times of the newest and oldest entries in the
log.

\begin{axdef}
        oldestLogTime : \finset Audit \fun TIME 
\\      newestLogTime : \finset Audit \fun TIME
\where
        \forall A, B : \finset Audit @
\\  \t1 newestLogTime (A \cup B) \geq newestLogTime~ A
\\  \t1 \land oldestLogTime (A \cup B) \leq oldestLogTime~ A
\end{axdef}
\begin{Zcomment}
\item 
Both these functions are monotonic. In particular the $newestLogTime~
\emptyset$ is the earliest time and the 
$oldestLogTime~ \emptyset$ is the latest
possible time.
\end{Zcomment}

%----------------------------------------------------------------------
\section{Key Store}
%----------------------------------------------------------------------

\begin{traceunit}{FS.KeyStore.State}
\end{traceunit}


TIS maintains a key store which contains all Issuer keys relevant to
its function. This will include known CAs, AAs and its own key. 
Once enrolled, the key store also contains the ID station's name
and own key.

\begin{schema}{KeyStore}
        issuerKey : ISSUER \pfun KEYPART
\\      ownName : \Optional ISSUER
\where
        ownName \neq \Nil \implies \The ownName \in \dom issuerKey 
\end{schema}
\begin{Zcomment}
\item
An ID Station is issued with a name at enrolment. Prior to enrolment
it will not have a name.
\item
This ID Station, once named, will have its key held with the other 
issuers' keys.
\end{Zcomment}

%--------------------------------------------------------------------
\section{System Statistics}
%--------------------------------------------------------------------

\begin{traceunit}{FS.Stats.State}
\end{traceunit}


TIS keeps track of the number of times that a entry to the
enclave has been attempted (and denied) and the number of times it has succeeded. It
also records the number of times that a biometric comparison has been
made (and failed) and the number of times it succeeded.

By retaining these statistics it is possible for the performance of
the system to be monitored.

\begin{schema}{Stats} 
        successEntry : \nat
\\      failEntry : \nat
\\      successBio : \nat
\\      failBio : \nat
\end{schema}

%--------------------------------------------------------------------
\section{Administration}
%--------------------------------------------------------------------

\begin{traceunit}{FS.Admin.State}
\traceto{SFP\_DAC}
\traceto{FDP\_ACC.1.1}
\traceto{FDP\_ACF.1.1}
\traceto{FDP\_ACF.1.2}
\traceto{FDP\_ACF.1.3}
\traceto{FDP\_ACF.1.4}
\traceto{FMT\_MOF.1.1}
\traceto{FIA\_USB.1.1}
\traceto{FMT\_MSA.1.1}
\traceto{FMT\_MTD.1.1}
\traceto{FMT\_SMR.2.1}
\traceto{FMT\_SAE.1.1}
\end{traceunit}

In addition to its role of authorising entry to the enclave, TIS 
supports a number of administrative operations. 

\begin{itemize}
\item ArchiveLog - writes the archive log to floppy and truncates the
internally held archive log.
\item UpdateConfiguration - accepts new configuration data from a floppy. 
\item OverrideDoorLock - unlocks the enclave door. 
\item Shutdown - stops TIS, leaving the protected entry to the enclave secure.
\end{itemize}

\begin{syntax}
        ADMINOP ::=  archiveLog | updateConfigData |
        overrideLock | shutdownOp 
\end{syntax}

Other operations that could be supported are 
{\em DisplayLog}, 
{\em CancelAlarm}, 
{\em ClearStats}, 
{\em Decommission}, 
{\em AddIssuers}, 
{\em RemoveIssuers}.
These additional operations will be considered out of scope of this
re-development. 

Only users with administrator privileges can make use of the TIS to
perform administrative functions. There are a number of different
administrator privileges that may be held.

\begin{zed}
        ADMINPRIVILEGE == \{ guard, auditManager, securityOfficer \}
\end{zed}

The role held by the administrator will determine the operations
available to the administrator. An administrator
can only hold one role.

\begin{schema}{Admin}
        rolePresent : \Optional ADMINPRIVILEGE
\\      availableOps : \power ADMINOP
\\      currentAdminOp : \Optional ADMINOP
\where
        rolePresent = \Nil \implies availableOps = \emptyset 
\\      (rolePresent \neq \Nil \land \The rolePresent = guard) \implies availableOps = 
        \{ overrideLock \}
\\      (rolePresent \neq \Nil \land \The rolePresent = auditManager) \implies availableOps = 
        \{ archiveLog \}
\\      (rolePresent \neq \Nil \land \The rolePresent = securityOfficer) \implies availableOps = 
        \{ updateConfigData, shutdownOp \}
\\      currentAdminOp \neq \Nil \implies 
\\ \t1 (\The currentAdminOp \in availableOps \land rolePresent \neq \Nil )
\end{schema}
\begin{Zcomment}
\item
The $availableOps$ are completely determined by the roles present.
\end{Zcomment}

In order to perform an administrative operation an administrator must
be present. Presence will be determined by an appropriate token being
present in the administrator's card reader.

%-----------------------------------------------------------------------
\section{Real World Entities}
%-----------------------------------------------------------------------

\begin{traceunit}{FS.RealWorld.State}
\traceto{FAU\_ARP.1.1}
\traceto{FAU\_SAA.1.1}
\traceto{FAU\_SAA.1.2}
\traceto{FPT\_RVM.1.1}
\end{traceunit}


The latch is allowed to be in two states:
$locked$ and $unlocked$.
When the latch is unlocked,
$latchTimeout$ will be set to the time at which the lock must again be $locked$.


The alarm is similar to the latch, 
in that it has a $silent$, and $alarming$,
with an $alarmTimeout$.
Once the door and latch move into a potentially insecure state
(door $open$ and latch $locked$)
then the $alarmTimeout$ is set to the time at which the alarm will sound.


The state of $currentLatch$ is entirely derived from whether the $latchTimeout$
has fired or not.
The state of $doorAlarm$ is also entirely derived ---
if the state is potentially insecure and the $alarmTimeout$ has fired,
the alarm must be $alarming$.

\begin{schema}{DoorLatchAlarm}
	currentTime: TIME
\\	currentDoor: DOOR
\\	currentLatch: LATCH
\\	doorAlarm: ALARM
\\	latchTimeout: TIME
\\	alarmTimeout: TIME
\where
	currentLatch = locked \iff currentTime \geq latchTimeout
\\	doorAlarm = alarming \iff 
\\ \t1		(currentDoor = open
\\ \t2			\land currentLatch = locked
\\ \t2			\land currentTime \geq alarmTimeout
			)
\end{schema}


The ID Station holds internal representations of all of the Real World,
plus its own data.
It holds separate indications of the presence of input in the real world peripherals
of the User Token, Admin Token, Fingerprint reader, and Floppy disk.
This is so that once the input has been read,
and the card, finger or disk removed,
the ID Station can continue to know what the value was,
even if it later detects that the real world entity has been removed.

\begin{schema}{UserToken}
	currentUserToken: TOKENTRY
\\	userTokenPresence: PRESENCE
\end{schema}

\begin{schema}{AdminToken}
	currentAdminToken: TOKENTRY
\\	adminTokenPresence: PRESENCE
\end{schema}

\begin{schema}{Finger}
	currentFinger: FINGERPRINTTRY
\\	fingerPresence: PRESENCE
\end{schema}

We need to retain an internal view of the last data written to the
floppy as well as the current data on the floppy, this is because we
need to check that writing to floppy works when we archive the log.
\begin{schema}{Floppy}
	currentFloppy: FLOPPY
\\      writtenFloppy: FLOPPY
\\	floppyPresence: PRESENCE
\end{schema}

\begin{schema}{Keyboard}
        currentKeyedData: KEYBOARD
\\      keyedDataPresence : PRESENCE
\end{schema}  

%-----------------------------------------------------------------------
\section{Internal State}
%-----------------------------------------------------------------------

\begin{traceunit}{FS.Internal.State}
\traceto{FPT\_RVM.1.1}
\end{traceunit}


$STATUS$ and $ENCLAVESTATUS$ are purely internal records of the progress through 
processing. $STATUS$ tracks progress through user entry, while
$ENCLAVESTATUS$ tracks progress through all activities performed
within the enclave. 

\begin{zed}
	STATUS ::= quiescent | 
\also \t2          gotUserToken |  waitingFinger | gotFinger | waitingUpdateToken | waitingEntry |
\\ \t2          waitingRemoveTokenSuccess | waitingRemoveTokenFail 
\also
        ENCLAVESTATUS ::= notEnrolled | waitingEnrol | waitingEndEnrol |
\also \t2          enclaveQuiescent |
\also \t2          gotAdminToken | waitingRemoveAdminTokenFail |
waitingStartAdminOp | waitingFinishAdminOp |
\also \t2          shutdown 
\end{zed}

The states $quiescent$ and $enclaveQuiescent$ represent the enclave
interface and the user entry interface being quiescent.

The states $gotUserToken$, .. 
$waitingRemoveTokenFail$ are all associated with the process of user
authentication and entry. These are described futher in Section \ref{sec:userEntry}.

The states $notEnrolled$, .. $waitingEnrolEnd$ reflect
enrolment activity that must be performed before TIS can offer any of
its normal operations. Once the TIS is successfully enrolled it
becomes $quiescent$.

The states $gotAdminToken$, .. $waitingFinishAdminOp$ 
reflect activity at the TIS console relating to administrator use of TIS. 

The state $shutdown$ models the system when it is shutdown.

Internally the system maintains knowledge of the status of the user
entry operation and the enclave. It also holds a timeout which is only
relevant when the status is on $waitingRemoveTokenSuccess$.

\begin{schema}{Internal}
        status : STATUS
\\      enclaveStatus : ENCLAVESTATUS
\\      tokenRemovalTimeout : TIME
\end{schema}

%-----------------------------------------------------------------------
\section{The whole Token ID Station}
%-----------------------------------------------------------------------

\begin{traceunit}{FS.TIS.State}
\end{traceunit}


The whole Token ID Station is constructed from combining the described
state components. 

In addition there is a display outside the enclave and screen
within the enclave. The ID Station screen within the enclave may
display many pieces of 
information. The majority of this data will be determined by state invariants.

The alarm, door and latch conform to their consistency rules.
The relationships between available operations and roles present are
preserved. 

If the authentication protocol has moved on to requesting a fingerprint,
then the User Token will have passed its validation checks.

Similarly if the system considers there to be an administrator present
then the Admin Token will have passed its validation checks.

Once the ID station has been enrolled it has a name.

TIS is only ever in the two states $waitingStartAdminOp$ or
$waitingFinishAdminOp$ when then there is a current admin operation in
progress. For single phase operations the state
$waitingFinishAdminOp$ is not used.

TIS will only read the Admin Token to log on an administrator if there
is not an administrator role currently present.
 
\begin{schema}{IDStation}
	UserToken
\\	AdminToken
\\	Finger
\\	DoorLatchAlarm
\\      Floppy
\\      Keyboard
\\      Config
\\      Stats
\\      KeyStore
\\      Admin
\\      AuditLog
\\      Internal
\also
	currentDisplay: DISPLAYMESSAGE
\\      currentScreen: Screen
\where
	status \in \{~ gotFinger, waitingFinger, waitingUpdateToken, waitingEntry~\} \implies
\\ \t1		 (( \exists ValidToken @ 
			goodT(\theta ValidToken) = currentUserToken )
\\ \t2  \lor ( \exists TokenWithValidAuth @ 
			goodT(\theta TokenWithValidAuth) = currentUserToken))
\also
        rolePresent \neq \Nil \implies       
\\ \t1		 ( \exists TokenWithValidAuth @ 
			goodT(\theta TokenWithValidAuth) = currentAdminToken )
\also
        enclaveStatus \notin \{~notEnrolled, waitingEnrol, waitingEndEnrol ~\} \implies       
\\ \t1          ( ownName \neq \Nil )
\also
        enclaveStatus \in \{~ waitingStartAdminOp, waitingFinishAdminOp ~\} \iff currentAdminOp \neq \Nil
\also
       (currentAdminOp \neq \Nil \land \The currentAdminOp \in \{~
shutdownOp, overrideLock ~\}) 
\\ \t2          \implies enclaveStatus = waitingStartAdminOp
\also
        enclaveStatus = gotAdminToken \implies rolePresent = \Nil
\also   % invariants that define the screen
        currentScreen.screenStats = displayStats (\theta
        Stats)
\\      currentScreen.screenConfig = displayConfigData (\theta Config)
\end{schema}
\begin{Zcomment}
\item
Note that the token may not still be current since time will have
moved on since the checks were performed.
\item
Operations that can be performed in a single phase do not result in
TIS entering the state $waitingFinishAdminOp$ as they are finished
when they are started. 
\item
TIS only enters the state $gotAdminToken$ when there is no
administrator present.
\item
Invariants define many of the screen components.
\end{Zcomment}


