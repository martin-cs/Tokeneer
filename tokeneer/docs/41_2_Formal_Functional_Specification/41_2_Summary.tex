%============================================================================
\chapter{Commentary on this Specification}
\label{sec:Summary}
%============================================================================

This specification is intended to give a representative formal
specification of a realistic system. Budgetary restricitions have
meant that the number of administrative operations have been kept to a
minimum, although we intend that sufficient have been provided to make
the specification representative.

%--------------------------------------------------------------------------
\section{The structure of the Z}
%--------------------------------------------------------------------------
Throughout the specification care has been taken to ensure that each
schema is relatively simple. This is an important characteristic in
ensuring that a reader can understand the purpose of each
schema. Excessive complexity risks making the specification obscure. 
Schema composition has been used to build up complex
operations from simple schemas.

This Z specification is larger than was originally anticipated. We
have considered the reason for this and conclude that it is because
\begin{itemize}
\item
the functionality is larger than originally expected (especially at
the administrative interface).
\item
The core TIS has a fairly large number of interfaces to its
environment, two card readers, biometric verifier, door, latch, alarm,
internal and external display, floppy disk and keyboard. Each of these
has been modelled and the formal notation requires us to explicitly
describe what happens to each of these during every operation.
\item
The entry process contains more steps than originally expected - each
interaction with an interface requires a new step.
\end{itemize}

The relatively large number of interfaces and the desire to compose
the system from a number of relatively 
simple schemas has resulted in the size of specification presented here.

%--------------------------------------------------------------------------
\section{Issues}
%--------------------------------------------------------------------------
A few issues arose while writing this specification; this is expected
when formalizing requirements that are stated in any natural language.

We present the more interesting observations here:

\subsection{The choice of Real World Model}
The way in which the real world was modelled is interesting. We would
conventionally  use Z $inputs?$ and $outputs!$, however, these are
global through schema composition, which means that one cannot compose
two schemas with common inputs or outputs without constraining these
entities to be the same. 

Within our main loop we want to be able to update the alarm and latch
twice; once directly after polling and once after the main
processing.
Avoiding $inputs?$ and $outputs!$ allows us to reason about the main loop as
a composition of polling, calculations and updates.

This has resulted in us defining the real world using a state schema
$RealWorld$ and modelling the possible changes to this state. This
gives a model that is easier to reason about formally. In particular
we can sensibly consider the effect of composing several iterations
around the main loop.

We are also able to state and reason about security properties
involving real world entities.

\subsection{Denial of service}
When this specification was written the assumption was taken that
only one operation could be performed at a time, so once a user had
started to attempt authentication and entry in to the enclave, no
administrive functions could be supported within the enclave. This
assumption was first introduced in the SRS \cite{SRS}. With this
assumption it seemed natural that there only need be one internal
state component tracking what the system was doing. 

The model does not cover the details of how a token interacts with the
card reader, in particular there is no modelling of the
Answer-to-Reset (ATR), which is
transmitted when a token is first presented to a reader. This was a
deliberate abstraction. Within the model we capture the presence of a
token and the values held on the token. For operations that are
triggered by the presence of a token we allow the operation to start
if the system is quiescent and the token is present. One such
operation is the user authentication and entry operation. We
also require that the token is removed in order for the user
authentication and entry process to be considered complete. 
Otherwise, the system would countinually reprocess 
a token with bad data until the point that it is removed. 
The result of this modelling assumption was that we needed an
internal state where the system was waiting for a token to be removed
following a failure. 

The result of having only one internal state component was that
placing an invalid token in the reader outside the enclave and leaving
it there would block any adminstrator use of the system. Similarly
leaving an invalid token in the reader inside the enclave and leaving
it there would block any attempt by a user to enter the enclave. This
was an unacceptable denial of service. A malicious user could lockup the
system. To overcome this problem the internal state was divided in
two. One part, $status$, manages the multi-phase user authentication and
entry process, the other part, $enclaveStatus$, manages all the
activities which involve interaction with TIS from within the enclave.
The result of this change was that we were able to eliminate the
denial of service attack resulting from a token being left in a
reader. A token left in a reader now only blocks other activities that
would make use of that reader.

There are other points in the model where the system will wait
indefinitely, these have been left as they all arise during operations
that can only be performed by an authenticated administrator. We make
the assumption that an administrator with privileges to perform these
operations will not maliciously leave the system waiting for a floppy
disk in order to deny user entry. 

\subsection{The Audit Log}
The audit log was the most complex component of the system to
model. We wanted the model of the log to be abstract within this
specification and we wanted to postpone details of the exact elements
that would be placed in the log until the formal design. However we did
want to capture some of the key points such as the effect of log
overflow and the fact that entries were made to the audit log when key
events occurred.

We had to take care that the specification was not too tight, for
example we need to allow both the specification and the design to add 
several entries to the log during the course of an operation and the
design may introduce more log entries than are captured in the
specification. If the model had been too prescriptive in this area
then 
there would be no viable refinement relation allowing additional 
log entries in the design over those introduced in the specificaiton.
To achieve this within this model we allow a number of entries to be
added to the log at a time. 

This specification only details when the audit log
must be updated, it places no restriction on further entries being
added than are detailed in this specification. Once all audit entries
have been defined in the design we can place further restrictions on
when values may or may not be added to the log.

\subsection{Malicious attack on the audit log}
The Protection Profile \cite{PP} requires that old entries in the
audit log should be overwritten in the event of the audit log becoming
full. This has been modelled in this specification. However this
functionality raises
the possibility of data being erased from the audit log by performing
events that are audited. For example a user could replace the audit
log with events corresponding to repeated attempts to log on as an
administrator. We did consider preventing all operations, other than
archiving the log once the $auditAlarm$ has been raised, but since
this requires the $AuditManager$ to be logged on to the system we
cannot exclude the administrator logon activity. 

The only other obvious solution would be to shutdown TIS once there
was a risk of the audit log becoming full. This assumes that there is
a mechanism outside of the TIS function for archiving the log. 
We have therefore left the functionality allowing unlimited
overwriting on the grounds that the alarm is sufficient protection.

\subsection{Relating enclave entry and Auth Cert generation}

The final specification presented here uses indepentent configuration
information to determine the authorisation period applied to
authorisation certificates and the times at which entry to the enclave
should be allowed. 

Originally we only allowed entry if the current time was within the
authorisation period on the certificate. This seemed to confuse the
distinct activities of issuing an Authorisation Certificate and
allowing user entry. The original restriction can still be achieved by
constraints on the configuration data.

We have also decided not to write an authorisation certificate if its
authorisation period is empty. This is a requirements issue that was
raised during the production of the specification.

\subsection{Detail postponed until the design}

There are a number of points of detail which are not required to
express the system functionality at the abstract level presented
within this specification. 

One example of this is the deliberate omission of the serial number
from the formal model of the certificate Id. It was found that for the
purpose of describing the functionality of the TIS the serial number
of a certificate was irrelevant. This is because there is no need to
demonstrate uniqueness of certificate ids. All that is important
within this model is that we can deduce who issued a certificate; this
enables us to validate the certificate. The serial number will be
introduced in the design where it appears in the detailed information
that is audited. 

\subsection{Assumptions on Real World behaviour}
In order for the specified TIS to function correctly we need to make a
number of assumptions on the behaviour of the modelled real world and
its interactions with TIS.

The first assumption is made explicitly in Section
\ref{sec:ReadWorld} and is a requirement that the 
time source can be trusted to provide us with time that increases.

The second assumption is more subtle and was uncovered while
performing the precondition proof for $TISPoll$ (see page
\pageref{sec:PollPre}). The second assumption
is that TIS polls the real world sufficiently frequently that it will
always observe the absence of a token before it observes the presence
of the next token. This assumption ensures that the Tokens in the
reader cannot be swapped without TIS noticing. TIS is capable of
noticing token tears, the case where a token is removed mid-processing.
However the system as specified will not notice a change in the token
contents if it can be swapped without TIS detecting the absence of the
first token. This assumption ensures that TIS only makes use of a
token that it has previously validated and guarantees that errors such
as writing an Authorisation certificate to the wrong token do not occur.

It is necessary to validate both these assumptions to ensure that the 
system implemented from this specification is indeed secure. 
If, for example, analysis of the second assumption indicates that 
it is unreasonable then 
it would be necessary to reconsider the mechanism by which we monitor 
the token readers. For instance, it might be necessary to introduce an 
interrupt triggered by the removal of a token.
