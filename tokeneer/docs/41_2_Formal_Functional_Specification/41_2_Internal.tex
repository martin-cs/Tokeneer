
%=======================================================================
\chapter{Internal Operations}
\label{sec:Internal}
%=======================================================================
In this section we present a number of operations performed internally
by the TIS. These operations are combined to create the operations
available to the user.

%-------------------------------------------------------------------------
\section{Updating the Audit Log}
%-------------------------------------------------------------------------

%.....................................
\subsection{Adding elements to the Log}
%.....................................

\begin{traceunit}{FS.AuditLog.AddElementsToLog}
\traceto{ScGeneral.Fail.Audit}
\traceto{ScGainInitial.Fail.AuditPreserve}
\traceto{ScProhibitInitial.Fail.AuditPreserve}
\traceto{ScGainRepeat.Fail.AuditPreserve}
\traceto{ScStart.Fail.AuditPreserve}
\traceto{ScShutdown.Fail.AuditPreserve}
\traceto{ScConfig.Fail.AuditPreserve}
\traceto{ScUnlock.Fail.AuditPreserve}
\traceto{ScLogOn.Fail.AuditPreserve}
\traceto{ScLogOff.Fail.AuditPreserve}
\traceto{FAU\_ARP.1.1}
\traceto{FAU\_SAA.1.1}
\traceto{FAU\_SAA.1.2}
\traceto{FAU\_STG.2.3}
\traceto{FAU\_STG.4.1}
\end{traceunit}


When we add a set of entries to the log, either there is sufficient
room in the log for the new entries, in which case the new entries are
added to the log, or there is insufficient room in the log to add the
new entries and the oldest part of the log is truncated to make room
for the new log entries. 
We don't specify here how much of the log is truncated
although it is likely to be sufficient to continue adding some data
without further truncations.
If the log is truncated or is close to its maximum size, an alarm is raised 
to notify the administrator that the log is full. 

\begin{schema}{AddElementsToLog}
        Config
\\      \Delta AuditLog
\where
        \exists newElements : \finset_1 Audit @
\\ \t1        oldestLogTime~ newElements \geq newestLogTime~ auditLog
\also
\\ \t1 \land  (auditLog' = auditLog \cup newElements
\\ \t2  \land (sizeLog~ auditLog' < alarmThresholdSize \land
auditAlarm' = auditAlarm   
\\ \t3  \lor  sizeLog~ auditLog' \geq alarmThresholdSize \land
auditAlarm' = alarming)
\\ \t4  \lor
\\ \t1    sizeLog~ auditLog + sizeLog~ newElements > minPreservedLogSize 
\\ \t2  \land (\exists oldElements : \finset Audit  @ 
\\ \t3  oldElements \cup auditLog' = auditLog \cup newElements 
\\ \t3  \land oldestLogTime~ auditLog' \geq newestLogTime~ oldElements)
\\ \t3  \land sizeLog~ auditLog' \geq minPreservedLogSize
\\ \t3  \land auditAlarm' = alarming  )            
\end{schema}
\begin{Zcomment}
\item
We make an assuption that all data added to the log is no older than
the data already in the log. 
\item
This operation is non-deterministic when the addition of the set of
$newElements$ will make the log larger than the
$minPreservedLogSize$. At this point the log may, or may not be
truncated.
\item
If the configuration data changes it is possible that the $minPreservedLogSize$
becomes larger or smaller, any new value for this configurable item
will not take effect until configuration is complete.
\end{Zcomment}

%..............................
\subsection{Archiving the Log}
%..............................

\begin{traceunit}{FS.AuditLog.ArchiveLog}
\end{traceunit}

When we archive the log an audit element is added to the log and an
archive is generated which can be written to floppy. 

This activity does not clear the log since a check will be made to
ensure the archive was successful before clearing the log.

\begin{schema}{ArchiveLog}
        Config
\\      \Delta AuditLog
\\      archive : \finset Audit
\where
        \exists  notArchived,
        newElements : \finset Audit @
\\ \t1           archive \subseteq auditLog \cup newElements
\\ \t1          \land auditLog' \subseteq archive \cup notArchived
\\ \t1          \land newestLogTime~ archive \leq oldestLogTime~ notArchived 
\\ \t1          \land AddElementsToLog
\end{schema}
\begin{Zcomment}
\item
The explicit contraints on this schema define the component of the
audit log that will be the $archive$. The constraints ensure that the
$archive$ includes the oldest elements and has no gaps in it.
\item
This operation is used in the total operation that writes the archive
log to floppy. $archive$ is the audit log that is written to floppy.
\item
The $archive$ only contains some of the final log. The part of the log that is
not archived is represented by $notArchived$.
\end{Zcomment}

%..............................
\subsection{Clearing the Log}
%..............................

\begin{traceunit}{FS.AuditLog.ClearLog}
\traceto{FAU\_ARP.1.1}
\end{traceunit}

The log should only be cleared if it can be verified that an archive
has been created of the data that is about to be cleared.

When the log is cleared the component that has been archived is
eliminated from the log. There may still be some elements in the log,
these will have been added since the archive. Where the log has
overflowed since the time of the archive the archive may contain
entries older than those in the log.

If the log is cleared successfully then the $auditAlarm$ is cancelled
(provided that the size of the audit log is not larger than the alarm threshold size).

\begin{schema}{ClearLog}
        Config
\\      \Delta AuditLog
\\      archive : \finset Audit
\where
      (\exists sinceArchive,
        lostSinceArchive : \finset Audit @  
\\ \t1    archive \cup sinceArchive = lostSinceArchive \cup auditLog 
\\ \t1    \land oldestLogTime~ sinceArchive \geq newestLogTime~ archive
\\ \t1    \land newestLogTime~ lostSinceArchive \leq oldestLogTime~ auditLog
\\ \t1    \land auditLog' = sinceArchive )
\also
        (sizeLog~ auditLog' < alarmThresholdSize \land auditAlarm' = silent
\\ \t1  \lor sizeLog~ auditLog' \geq alarmThresholdSize \land auditAlarm' = alarming)

\end{schema}
\begin{Zcomment}
\item
This operation is not total, it will only be used to construct a total
operation that makes $archive$  the value read back successfully from
the floppy. Thus $archive$ will have been the whole audit log at some
point in the past.
\end{Zcomment}

%.....................................................
\subsection{Auditing Changes}
%.....................................................

\begin{traceunit}{FS.AuditLog.LogChange}
\traceto{ScGainInitial.Suc.Audit}
\traceto{ScProhibitInitial.Suc.Audit}
\traceto{ScGainRepeat.Suc.Audit}
\traceto{ScUnlock.Suc.Audit}
\traceto{FAU\_ARP.1.1}
\traceto{FAU\_SAA.1.1}
\traceto{FAU\_SAA.1.2}
\end{traceunit}


TIS adds audit entries whenever any of the following changes occurs:
\begin{itemize}
\item
The door is opened or closed.
\item
The door is latched or unlatched.
\item
The alarm  starts alarming or becomes silenced.
\item
The audit alarm  starts alarming or becomes silenced.
\item
The text displayed on the display changes.
\item
The text displayed on the screen changes.
\end{itemize}

\begin{schema}{AuditDoor}
        \Delta DoorLatchAlarm
\\        AddElementsToLog
\where
        currentDoor \neq currentDoor'
\end{schema}

\begin{schema}{AuditLatch}
        \Delta DoorLatchAlarm
\\        AddElementsToLog
\where
        currentLatch' \neq currentLatch 
\end{schema}

\begin{schema}{AuditAlarm}
        \Delta DoorLatchAlarm
\\      AddElementsToLog
\where
        doorAlarm \neq doorAlarm' 
\also
\end{schema}

\begin{schema}{AuditLogAlarm}
        AddElementsToLog
\where
        auditAlarm \neq auditAlarm'
\end{schema}

\begin{schema}{AuditDisplay}
        AddElementsToLog
\\      \Delta IDStation 
\where
        currentDisplay' \neq currentDisplay
\end{schema}

\begin{schema}{AuditScreen}
        \Delta IDStation 
\\      AddElementsToLog
\where
        currentScreen'.screenMsg \neq currentScreen.screenMsg 
\end{schema}

If none of these changes occur then the audit log may still be updated due
to the operation being executed; if no operation driven events
occur it will not change.

\begin{schema}{NoChange}
      \Delta IDStation
\where
        currentDoor = currentDoor'
\\      currentLatch' = currentLatch 
\\      doorAlarm = doorAlarm' 
\\      auditAlarm = auditAlarm'
\\      currentDisplay' = currentDisplay
\\      currentScreen'.screenMsg = currentScreen.screenMsg 
\also
        AddElementsToLog \lor \Xi AuditLog
\end{schema}
\begin{Zcomment}
\item
This is a very weak statement in the specification, because we are
postponing elaboration of $Audit$ elements until the design.
\end{Zcomment}


\begin{zed}
        LogChange \defs AuditAlarm \lor AuditLatch \lor AuditDoor
        \lor AuditLogAlarm \lor AuditScreen \lor AuditDisplay 
\\ \t3 \lor NoChange
\end{zed}
%-------------------------------------------------------------------------
\section{Updating System Statistics}
%-------------------------------------------------------------------------

\begin{traceunit}{FS.Stats.Update}
\end{traceunit}

System statistics are updated as actions that are being monitored for
the statistics occur.

We provide operations to increment the count of each of the events
being monitored.

\begin{schema}{AddSuccessfulEntryToStats}
        \Delta Stats
\where
        failEntry' = failEntry 
\\      successEntry' = successEntry + 1
\\      failBio' = failBio
\\      successBio' = successBio 
\end{schema}

\begin{schema}{AddFailedEntryToStats}
        \Delta Stats
\where
        failEntry' = failEntry + 1
\\      successEntry' = successEntry
\\      failBio' = failBio
\\      successBio' = successBio 
\end{schema}

\begin{schema}{AddSuccessfulBioCheckToStats}
        \Delta Stats
\where
        failEntry' = failEntry 
\\      successEntry' = successEntry
\\      failBio' = failBio
\\      successBio' = successBio + 1
\end{schema}

\begin{schema}{AddFailedBioCheckToStats}
        \Delta Stats
\where
        failEntry' = failEntry 
\\      successEntry' = successEntry
\\      failBio' = failBio + 1
\\      successBio' = successBio
\end{schema}


%------------------------------------------------------------------------
\section{Operating the Door}
%------------------------------------------------------------------------

\begin{traceunit}{FS.Door.UnlockDoor}
\end{traceunit}


The door is unlatched by updating the timeouts on the door
latch and alarm. 

\begin{schema}{UnlockDoor}
        \Delta DoorLatchAlarm
\\      Config
\where
        latchTimeout' = currentTime + latchUnlockDuration
\\      alarmTimeout' = currentTime + latchUnlockDuration + alarmSilentDuration
\\      currentTime' = currentTime
\\      currentDoor' = currentDoor
\end{schema}


\begin{traceunit}{FS.Door.LockDoor}
\end{traceunit}

The door is explicitly latched and timeouts on the door
latch and alarm are reset. 
Resetting the timeouts to the current time will ensure that the door
will be latched directly and the alarm sound if there is a breach of security.

\begin{schema}{LockDoor}
        \Delta DoorLatchAlarm
\where
        currentLatch' = locked
\\      latchTimeout' = currentTime
\\      alarmTimeout' = currentTime 
\\      currentTime' = currentTime
\\      currentDoor' = currentDoor
\end{schema}


%------------------------------------------------------------------------
\section{Certificate Operations}
%------------------------------------------------------------------------

%.....................................
\subsection{Validating Certificates}
%.....................................

\begin{traceunit}{FS.Certificate.CertificateOK}
\end{traceunit}

When a certificate is checked in the context of a key store it is
only acceptable if the certificate issuer is known to the key store
and the signature can be verified by the key store.

A certificate must have been issued by a known issuer.

\begin{schema}{CertIssuerKnown}
        KeyStore
\\      Certificate
\where
        id.issuer \in \dom issuerKey
\end{schema}

A certificate must have been signed by the issuer. 

\begin{schema}{CertOK}
        CertIssuerKnown
\where
        issuerKey(id.issuer) \in isValidatedBy
\end{schema}

%........................................

\begin{traceunit}{FS.Certificate.AuthCertificateOK}
\end{traceunit}


In addition the Authorisation certificate must have been issued by this ID
station; we make the assumption that a single ID station protects an enclave.

\begin{schema}{CertIssuerIsThisTIS}
        KeyStore
\\      Certificate
\where
        ownName \neq \Nil
\\      id.issuer = \The ownName
\end{schema}

\begin{zed}
        AuthCertOK \defs CertIssuerIsThisTIS \land CertOK
\end{zed}

%...............................................
\subsection{Generating Authorisation Certificates}
%...............................................

\begin{traceunit}{FS.Certificate.NewAuthCert}
\traceto{FDP\_UIT.1.1}
\traceto{FDP\_UIT.1.2}
\traceto{FIA\_UAU.3.2}
\end{traceunit}

An authorisation certificate can be constructed using information from
a valid token and the current configuration of TIS.
TIS can only generate the authorisation certificate if it has its own
key to perform the signing with; this is modelled as the TIS knowing
its own name.

\begin{schema}{NewAuthCert}
        ValidToken
\\      KeyStore
\\      Config
\\      newAuthCert : AuthCert
\\      currentTime : TIME
\where
        ownName \neq \Nil
\also
        newAuthCert.id.issuer = \The ownName
\\      newAuthCert.validityPeriod = authPeriod~ privCert.role~
currentTime
\\      newAuthCert.baseCertId = idCert.id
\\      newAuthCert.tokenID = tokenID
\\      newAuthCert.role = privCert.role
\\      newAuthCert.clearance = minClearance ( enclaveClearance ,
privCert.clearance)
\\      newAuthCert.isValidatedBy = \{~ issuerKey (\The ownName) ~\} 
\end{schema}



%..................................
\subsection{Adding Authorisation Certificates to User Token}
%..................................

\begin{traceunit}{FS.UserToken.AddAuthCertToUserToken}
\end{traceunit}


If a valid user token is present in the system then an authorisation
certificate can be added to it.


\begin{schema}{AddAuthCertToUserToken}
        \Delta UserToken
\\      KeyStore
\\      Config
\\      currentTime : TIME
\where
        userTokenPresence = present
\\      currentUserToken \in \ran goodT
\also
      \exists ValidToken; ValidToken' @ \theta ValidToken = (goodT \inv
currentUserToken) 
\\ \t1  \land \theta ValidToken' = (goodT \inv currentUserToken')
\\ \t1  \land (\exists newAuthCert : AuthCert @ \The authCert' = newAuthCert
        \land NewAuthCert)
\\ \t1  \land tokenID' = tokenID
\\ \t1  \land idCert' = idCert
\\ \t1  \land privCert' = privCert
\\ \t1  \land iandACert' = iandACert
\also
        userTokenPresence' = userTokenPresence
\end{schema}


%------------------------------------------------------------------------
\section{Updating the Key Store}
%------------------------------------------------------------------------

\begin{traceunit}{FS.KeyStore.UpdateKeyStore}
\end{traceunit}


The key store is updated using the supplied enrolment data 
to add issuers and their public keys.

\begin{schema}{UpdateKeyStore}
        \Delta KeyStore
\\      ValidEnrol
\where
        \The ownName' = idStationCert.subject
\\      issuerKey' = issuerKey \oplus\{ c : issuerCerts @ c.subject \mapsto c.subjectPubK
        \}
\\      \t1     \oplus ~\{ \The ownName \mapsto idStationCert.subjectPubK \} 
\end{schema}
\begin{Zcomment}
\item
This operation uses union and override so that it can be used to add
issuers as well as initial enrolment. 
\end{Zcomment}

The enrolment data will always be supplied on a floppy disk.

\begin{schema}{UpdateKeyStoreFromFloppy}
        \Delta KeyStore
\\      Floppy
\where
        currentFloppy \in \ran enrolmentFile
\\
        (\exists ValidEnrol @ \theta ValidEnrol = enrolmentFile \inv
currentFloppy
\\ \t1  \land UpdateKeyStore)   
\also
\end{schema}

%------------------------------------------------------------------------
\section{Administrator Changes}
%------------------------------------------------------------------------
An administrator may log on to the TIS console, logoff, or start an
operation.

%........................
\subsection{Logon Administrator}
%........................

\begin{traceunit}{FS.Admin.AdminLogon}
\end{traceunit}

An administrator can only log on if there is no-one currently logged on.

\begin{schema}{AdminLogon}
        \Delta Admin
\\      AdminToken
\where
        rolePresent = \Nil
\also
        \exists ValidToken @
\\ \t1  (goodT(\theta ValidToken )  = currentAdminToken
\\ \t1  \land \The rolePresent' = (\The authCert).role)
\also
        currentAdminOp' = \Nil
\end{schema}

%........................
\subsection{Logout Administrator}
%........................

\begin{traceunit}{FS.Admin.AdminLogout}
\end{traceunit}

An adminstrator, who is currently logged on can always log off.
This will terminate the current operation.

\begin{schema}{AdminLogout}
        \Delta Admin
\where
        rolePresent \neq \Nil
\also
        rolePresent' = \Nil
\\      currentAdminOp' = \Nil
\end{schema}

%........................
\subsection{Administrator Starts Operation}
%........................

\begin{traceunit}{FS.Admin.AdminStartOp}
\end{traceunit}

An adminstrator, who is currently logged on, can start any of the
operations that he is allowed to perform. An operation can only be
started if there is no operation currently in progress.

\begin{schema}{AdminStartOp}
        \Delta Admin
\\      Keyboard
\where
        rolePresent \neq \Nil
\\      currentAdminOp = \Nil
\\      currentKeyedData \in keyedOps \limg availableOps \rimg
\also
        rolePresent' = rolePresent
\\      \The currentAdminOp' = keyedOps \inv currentKeyedData
\end{schema}

%........................
\subsection{Administrator Finishes Operation}
%........................

\begin{traceunit}{FS.Admin.AdminFinishOp}
\end{traceunit}

An adminstrator, who is currently logged on, can finish an operation.

\begin{schema}{AdminFinishOp}
        \Delta Admin
\where
        rolePresent \neq \Nil
\\      currentAdminOp \neq \Nil
\also
        rolePresent' = rolePresent
\\      currentAdminOp' = \Nil 
\end{schema}



