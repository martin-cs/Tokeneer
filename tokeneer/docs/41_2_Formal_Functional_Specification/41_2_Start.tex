
%=========================================================================
\chapter{The Initial System and Startup}
\label{sec:Start}
%=========================================================================
\section{The Initial System}
%----------------------------------

\begin{traceunit}{FS.TIS.InitIDStation}
\traceto{FMT\_MSA.3.1}
\end{traceunit}

After initial installation the system has the following properties
\begin{itemize}
\item
an empty key store, which means it is unable to authorise entry to anyone;
\item
default configuration data, which does not permit entry to anyone;
\item
the door latched;
\item
an empty audit log;
\item
the internal times all set to zero (a time before the current time).
\end{itemize}

The door is assumed closed at initialisation, this ensures that the alarm
will not sound before the first time that data is polled.

\begin{schema}{InitDoorLatchAlarm}
        DoorLatchAlarm
\where
	currentTime = zeroTime
\\	currentDoor = closed
\\	latchTimeout = zeroTime
\\	alarmTimeout = zeroTime
\end{schema}

There are no keys held by the system and the TIS does not know its
name, this is supplied as part of enrolment.

\begin{schema}{InitKeyStore}
        KeyStore
\where
        issuerKey = \emptyset
\\      ownName = \Nil
\end{schema}

This default configuration assumes the lowest classification possible
for the enclave. This ensures that it does not give inadvertently
high clearance to the authorisation certificate. The $authPeriod$
and $entryPeriod$ functions are set to enable a $securityOfficer$ to
enter the enclave and re-configure the TIS. This configuration will
allow Auth Certificates to be generated with a validity of 2 hours
fron the point of issue (assuming that the unit of time is 1/10 sec).
\begin{schema}{InitConfig}
        Config
\where
	alarmSilentDuration = 10
\\      latchUnlockDuration = 150
\\      tokenRemovalDuration = 100
\\      enclaveClearance.class = unmarked
\\      authPeriod = PRIVILEGE \cross \{ \{ t: TIME @ t \mapsto t \upto t
+ 72000 \} \}
\\      entryPeriod = PRIVILEGE \cross \{ CLASS \cross \{ TIME \} \}
\end{schema}

Initially no administrator is logged on and no administator operations
are taking place.
\begin{schema}{InitAdmin}
        Admin
\where
        rolePresent = \Nil
\\      currentAdminOp = \Nil
\end{schema}

Initially the statistics are set to zero, indicating no use of the
system to date.
\begin{schema}{InitStats}
        Stats
\where
        successEntry = 0
\\      failEntry = 0
\\      successBio = 0
\\      failBio = 0
\end{schema}        

The initial audit Log is empty and there is no audit alarm.

\begin{schema}{InitAuditLog}
        AuditLog
\where
        auditLog = \emptyset
\\      auditAlarm = silent
\end{schema}        

Entities that model the real world and are polled and have no security
implications are not set 
at initialisation, these will be updated at the first poll of the real
world entities.

Initially the screen and the display are clear and the internal state
is $notEnrolled$.

\begin{schema}{InitIDStation}
        IDStation
\also
	InitDoorLatchAlarm
\\	InitConfig
\\      InitKeyStore
\\      InitStats
\\      InitAuditLog
\\      InitAdmin
\where
        currentScreen.screenMsg = clear
\also
	currentDisplay = blank
\\	enclaveStatus = notEnrolled
\\      status = quiescent
\end{schema}

%-------------------------------------------------------------------------
\section{Starting the ID Station}
%-------------------------------------------------------------------------
\begin{traceunit}{FS.TIS.TISStartup}
\traceto{FPT\_FLS.1.1}
\end{traceunit}


We assume that some of the state within TIS is persistent through
shutdown and some is not. 
The persistent items are $Config$, $KeyStore$ and $AuditLog$ all other state
components are set at startup. Those values that are polled can take
any valid value, we assume for simplicity that they remain unchanged.

\begin{schema}{StartContext}
        \Delta IDStation
\\      RealWorldChanges
\also
\\      \Xi Config
\\      \Xi KeyStore
\also
	InitDoorLatchAlarm'
\\      InitStats'
\\      InitAdmin'
\also
        \Xi UserToken  
\\      \Xi AdminToken 
\\      \Xi Finger 
\\      \Xi Floppy 
\\      \Xi Keyboard 
\end{schema}

In the case that TIS does not have an allocated name the ID station is
assumed to require enrolment.

\begin{schema}{StartNonEnrolledStation}
        StartContext
\where 
        ownName = \Nil
\also
        currentScreen'.screenMsg = insertEnrolmentData
\also
	currentDisplay' = blank
\\	enclaveStatus' = notEnrolled
\\      status' = quiescent
\also
        (\exists newElements : \finset Audit;
        startUnenrolledTISElement : Audit @ AddElementsToLog 
\\ \t1  \land startUnenrolledTISElement \in newElements )
\end{schema}
\begin{Zcomment}
\item
The $startUnenrolledTISElement$ is the audit entry recording that the
TIS has been started and requires enrolment. 
\end{Zcomment}

In the case that TIS does have an allocated name the ID station is
assumed to have been previously enrolled.

\begin{schema}{StartEnrolledStation}
        StartContext
\where 
        ownName \neq \Nil
\also
        currentScreen'.screenMsg = welcomeAdmin
\also
	currentDisplay' = welcome
\\	enclaveStatus' = enclaveQuiescent
\\      status' = quiescent
\also
        (\exists newElements : \finset Audit;
        startEnrolledTISElement : Audit @ AddElementsToLog 
\\ \t1  \land startEnrolledTISElement \in newElements )
\end{schema}
\begin{Zcomment}
\item
The $startEnrolledTISElement$ is the audit entry recording that an enrolled
TIS has been started. 
\end{Zcomment}

The complete startup operation is given by:

\begin{zed}
        TISStartup \defs StartEnrolledStation \lor
        StartNonEnrolledStation
\end{zed}







