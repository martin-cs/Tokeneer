
%==========================================================================
\chapter{The whole ID Station}
\label{sec:Whole}
%==========================================================================

\section{Startup}


When the TIS is powered up it needs to establish whether it is
enrolled or not. This is formally described by
\[
        TISStartUp
\]

\section{The main loop}


\begin{traceunit}{FS.TIS.TISMainLoop}
\end{traceunit}



The TIS achieves its function by repeatedly performing a number of 
activities within a main loop.

The main loop is broken down into several phases:

\begin{itemize}
\item   
{\em Poll} - Polling reads the simple real world entities
(door, time)
and the reads the complex entities
(user token reader, admin token reader, fingerprint reader, floppy).
\item
{\em Early Updates} - Critical updates of the door latch and alarm are
performed as soon as new polled data is available.
\item
{\em TIS processing} - TIS processing is the activity performed by
TIS, this is influenced by the current $status$ of TIS and the
recently read inputs.
\item
{\em Updates} - Critical updates of the door latch and alarm are
repeated once the processing is complete to ensure any internal state
changes result in the latch and alarm being set correctly. Less critical updates of the screen and display
are also performed once the processing is complete.
\end{itemize}

%...........................
\subsection{Polling}
%...........................

The polling activity is captured by the schema:
\[
TISPoll
\]
%...........................
\subsection{Early Updates}
%...........................
The early updates, which only update security critical outputs, are
described by:
\[
TISEarlyUpdate
\]
%...........................
\subsection{Processing}
%...........................

The the TIS processing depends on the current internal $status$ and $enclaveStatus$. 

Initially the only activity that can be performed is enrolment,
formally captured as
$TISEnrol$.

When it is in a $quiescent$ state it can start a number of activities. These
are started by either reading a user token, an adminstrator token or
keyboard data. In addition an administrator may logoff.

Formally the quiecent activities are:

\[
        TISReadUserToken \lor TISReadAdminToken \lor TISStartAdminOp
\lor TISAdminLogout
\]
\begin{Zcomment}
\item
$TISReadUserToken$ and $TISReadAdminToken$ are the first stages of
$TISUserEntry$ and $TISAdminLogon$.
\end{Zcomment}
Alternatively there is no token present, and no-one logged on, in this
case TIS is idle.

\begin{schema}{TISIdle}
        \Xi IDStation
\\      \Xi TISControlledRealWorld
\where
        status = quiescent
\\      enclaveStatus = enclaveQuiescent
\\      userTokenPresence = absent
\\      adminTokenPresence = absent
\\      rolePresent = \Nil
\end{schema}

Once a user token has been presented to TIS the only activities
that can be performed are stages in the multi-phase user entry
authentication operation, formally captured as $TISUserEntry$. Since
the user entry process is long lived it is necessary to check whether
the admin token has been removed during each stage of this operation
and act accordingly.

Once an administrator token has been presented to TIS the
administrator is logged onto the ID Station, formally captured as
$TISAdminLogon$. Having logged the administrator on TIS returns to a
$quiescent$ state waiting for the administrator to perform an
operation, without preventing user entry.

Once an operation request has been made by a logged on administrator
TIS performs the, potentially multi-phase, administrator operation,
formally captured as $TISAdminOp$ captured below:

\begin{zed}
        TISAdminOp \defs TISOverrideDoorLockOp  \lor TISShutdownOp 
\\      \t4     \lor TISUpdateConfigDataOp \lor TISArchiveLogOp
\end{zed}

The overall processing activity is described by:

\begin{zed}
        TISProcessing \defs (TISEnrolOp
\\ \t4  \lor TISUserEntryOp
\\ \t4  \lor TISAdminLogon 
\\ \t4  \lor TISStartAdminOp
\\ \t4  \lor TISAdminOp 
\\ \t4  \lor TISAdminLogout
\\ \t4  \lor TISIdle ) \land LogChange
\end{zed}

%...........................
\subsection{Final Updates}
%...........................
The updates performed following processing are described by:
\[
        TISUpdate
\]


