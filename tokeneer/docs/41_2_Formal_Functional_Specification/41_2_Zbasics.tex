%==========================================================================
\chapter{Reading Z, a small introduction}
\label{chap:readZ}
%==========================================================================
In this section we explain the basics of how to read Z. 

The main building block in Z is a {\em schema}. 
A Z schema takes the form of a number of state components and,
optionally, constraints on the state.

%%unchecked
\begin{schema}{SchemaName}
        declarations
\where
        constraints
\end{schema} 

For example we might declare a counter with an upper bound. The
counter variable $x$ is declared and it is constrained
to be less than 100.

%%unchecked
\begin{schema}{Counter}
x : \nat
\where
x < 100
\end{schema} 

Within the declarative part of a schema we can include schema
names. The effect of this inclusion is to bring into scope all the
variables and constraints of the schemas that have been included.
So in the following $NewCounter$ is a counter with a lower bound as
well as the upper bound inherited from $Counter$.

%%unchecked
\begin{schema}{NewCounter}
        Counter
\where
        40 < x
\end{schema}

Schemas are used to describe behaviour under change. Using a
convention of decorating state with a $'$ after change we can describe an
effect of an operation by describing the new values of the state in
terms of its relationship to
the old value of the state.  

We can describe a simple increment
of our counter by the following schema in which the new value of $x$
is the old value of $x$ incremented by 1. Note that the underlying
constraints on the variables from the $Counter$ schema still apply, so
$x' < 100$ is still true.
%%unchecked
\begin{schema}{IncrementCounter}
        Counter
\\      Counter'
\where
x' = x + 1
\end{schema}

In general, instead of writing $Counter, Counter'$ in our schema declaration we
make use of the definition
%%unchecked
\begin{schema}{\Delta Counter}
        Counter
\\      Counter'
\end{schema}

the $\Delta$ indicating a change to the variables in the schema $Counter$.

Another useful definition is $\Xi Counter$ which is describes the case
where the state of the schema is unchanged
%%unchecked
\begin{schema}{\Xi Counter}
        \Delta Counter
\where
        \theta Counter = \theta Counter'
\end{schema} 
\begin{Zcomment}
\item
The $\theta$ here is read as ``the state of ''.
\end{Zcomment}

In addition to schemas, Z allows us to define basic types which give
the types of the basic components of our schemas. We might want to
introduce the concept of a ``string'' as a basic type in our system,
this will appear as:

%%unchecked
\begin{zed}
        [ STRING ]
\end{zed}

We can define constants and functions. Here we define a constant
$clearString$ and print function that
turns natural numbers into strings.

%%unchecked
\begin{axdef}
        clearString : STRING
\\      printNat : \nat \inj STRING
\end{axdef}

Where we know all the possible entities of a basic type we can
declare it as a free type. $NEWSTRING$ is a free type with a named
value element $clearNewString$ and a function, $newPrintNat$ returning 
elements of type $NEWSTRING$.

%%unchecked
\begin{zed}
        NEWSTRING ::= clearNewString | newPrintNat \ldata \nat \rdata
\end{zed} 
A key property of the free type is that all elements of the free type
are distinct.

These ideas, along with a basic appreciation of predicate logic, should be sufficient to aid reading this
specification. For a more detailed description of the Z notation refer
to \cite{Spivey}.

