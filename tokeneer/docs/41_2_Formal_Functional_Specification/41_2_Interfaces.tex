
%=========================================================================
\chapter{Operations interfacing to the ID Station}
\label{sec:Interfaces}
%=========================================================================

%-------------------------------------------------------------------------
\section{Real World Changes}
%-------------------------------------------------------------------------
\label{sec:ReadWorld}
The monitored components of the real world can change at any time.
The only assumption we make of the real world is that the time
supplied by the external time source increases. If the external time
source does not supply increasing times then our system is not
guaranteed to work.

\begin{schema}{RealWorldChanges}
        \Delta RealWorld
\where
        now' \geq now
\end{schema}

%-------------------------------------------------------------------------
\section{Obtaining inputs from the real world}
%-------------------------------------------------------------------------

In this model all data is polled from the real world on a periodic
basis. 

%.................................
\subsection{Polling the real world}
%.................................

\begin{traceunit}{FS.Interface.TISPoll}
\traceto{FPT\_STM.1.1}
\end{traceunit}

We poll all of the real world entities.

Changes to the time, may affect the state of the latch. 

\begin{schema}{PollTime}
	\Delta DoorLatchAlarm
\\       RealWorld

\where
	currentTime' = now
\end{schema}


When polling the door, we do not change the alarm timeout or latch
timeout. The internal representation of the latch or the alarm may
change as a result of changes to the attributes that influence their
values. 

\begin{schema}{PollDoor}
	\Delta DoorLatchAlarm
\\	RealWorld
\where
	currentDoor' = door
\\	latchTimeout' = latchTimeout
\\	alarmTimeout' = alarmTimeout
\end{schema}

The system polls the tokens, finger, floppy and keyboard and the last
present value is stored. This allows the 
peripheral to be removed before TIS has completed use of the data.

\begin{schema}{PollUserToken}
        \Delta UserToken
\\	RealWorld
\where
	userTokenPresence' = present \iff userToken \neq noT
\\	currentUserToken' =  \IF userToken \neq noT \THEN userToken \ELSE currentUserToken
\end{schema}

\begin{schema}{PollAdminToken}
	\Delta AdminToken
\\      RealWorld
\where
	adminTokenPresence' = present \iff adminToken \neq noT
\\	currentAdminToken' = \IF adminToken \neq noT \THEN
adminToken \ELSE currentAdminToken
\end{schema}

\begin{schema}{PollFinger}
	\Delta Finger
\\	RealWorld
\where
	fingerPresence' = present \iff finger \neq noFP
\\	currentFinger' = \IF finger \neq noFP \THEN finger \ELSE currentFinger
\end{schema}

\begin{schema}{PollFloppy}
        \Delta Floppy
\\      RealWorld
\where
	floppyPresence' = present \iff floppy \neq noFloppy
\\      currentFloppy' = \IF floppy \neq noFloppy \THEN floppy \ELSE currentFloppy
\\      writtenFloppy' = writtenFloppy
\end{schema}

\begin{schema}{PollKeyboard}
        \Delta Keyboard
\\      RealWorld
\where
        keyedDataPresence = present \iff keyboard \neq noKB
\\        currentKeyedData' = \IF keyboard \neq noKB \THEN keyboard
        \ELSE currentKeyedData
\end{schema}

As a result of polling the time and door the alarm may become raised
or cleared and the latch locked or unlocked. Both of these events
should be recorded in the audit.
The opening and shutting of the door is also audited (auditing is
defined later in the specification).

So the overall poll operation is obtained by combining all the
individual polling actions.

If the user is
currently being invited to enter the enclave on the display and the 
door becomes latched then the display will change to indicate that the
system is no longer offering entry.

We assume that while polling occurs the $RealWorld$ does not change. 

\begin{schema}{TISPoll}
	\Delta IDStation
\\      \Xi RealWorld
\also
	PollTime
\\      PollDoor
\\	PollUserToken
\\	PollAdminToken
\\	PollFinger
\\      PollFloppy
\\      PollKeyboard
\\      LogChange
\also
        \Xi Config
\\      \Xi KeyStore
\\      \Xi Admin
\\      \Xi Stats
\\      \Xi Internal
\where
        currentScreen' = currentScreen
\also
        currentDisplay = doorUnlocked \land
        currentLatch' = locked
\\ \t1 \land        (status \neq waitingRemoveTokenFail \land
        currentDisplay' = welcome
\\ \t2 \lor status = waitingRemoveTokenFail \land currentDisplay' =
removeToken)
\\      \lor
        \lnot (currentDisplay = doorUnlocked \land currentLatch' =
        locked)
\\ \t1  \land currentDisplay' = currentDisplay
\end{schema}

%-------------------------------------------------------------------------
\section{The ID Station changes the world}
%-------------------------------------------------------------------------

%..........................
\subsection{Periodic Updates}
%..........................

We consider the process of updating the real world with the current
internal 
representation, one variable at a time.

\begin{schema}{UpdateLatch}
	\Xi DoorLatchAlarm
\\	RealWorldChanges
\where
	latch' = currentLatch
\end{schema}

\begin{schema}{UpdateAlarm}
	\Xi DoorLatchAlarm
\\      AuditLog
\\	RealWorldChanges
\where
        alarm' = alarming \iff doorAlarm = alarming \lor auditAlarm = alarming 
\end{schema}

\begin{schema}{UpdateDisplay}
	\Delta IDStation
\\	RealWorldChanges
\where
	display' = currentDisplay
\\      currentDisplay' = currentDisplay
\end{schema}

Configuration Data is only displayed if the security officer is
present.  System statistics are only displayed if an administrator is present.

\begin{schema}{UpdateScreen}
        \Delta IDStation
\\      \Xi Admin
\\      RealWorldChanges
\where
           screen'.screenMsg = currentScreen.screenMsg
\\           screen'.screenConfig = \IF \The rolePresent =
securityOfficer \THEN currentScreen.screenConfig \ELSE clear 
\\            screen'.screenStats = \IF rolePresent \neq \Nil
\THEN currentScreen.screenStats \ELSE clear
\end{schema}

All these can be combined, along with no change in the remaining
real world variables,
to represent the regular updating of the world. 
 
When updates to the real world occur it is possible that interfacing
with external devices will result in a system fault that is
audited. Not other aspects of TIS will change during updates of the
real world.

\begin{traceunit}{FS.Interface.TISEarlyUpdate}
\traceto{ScGainInitial.Suc.Locked}
\traceto{ScGainRepeat.Suc.Locked}
\traceto{ScUnlock.Suc.Locked}
\traceto{FAU\_ARP.1.1}
\traceto{FAU\_SAA.1.1}
\end{traceunit}


The alarm and the door latch will need to be updated as soon as
possible after polling the real world, this ensures that the system is
kept secure.

\begin{zed}
        TISEarlyUpdate \defs UpdateLatch \land UpdateAlarm 
\\ \t3        \land [~ RealWorldChanges | screen' = screen \land
        display' = display ~]
\\ \t3   \land [\Delta IDStation | currentDisplay = currentDisplay' ]
\\ \t3 \land \Xi UserToken \land \Xi AdminToken \land \Xi Finger \land
\Xi Floppy \land 
\\ \t3  \Xi Keyboard \land \Xi Config \land
\Xi Stats  
\\ \t3 \land \Xi KeyStore \land \Xi Admin \land \Xi Internal
\\ \t3 \land ( AddElementsToLog \lor \Xi AuditLog ) 
\end{zed}

\begin{traceunit}{FS.Interface.TISUpdate}
\traceto{ScGainInitial.Suc.Locked}
\traceto{ScGainRepeat.Suc.Locked}
\traceto{ScUnlock.Suc.Locked}
\traceto{FAU\_ARP.1.1}
\traceto{FAU\_SAA.1.1}
\traceto{FAU\_SAA.1.2}
\traceto{SFP.DAC}
\traceto{FMT\_MSA.1.1}
\traceto{FMT\_SMR.2.2}
\traceto{FMT\_SAE.1.1}
\end{traceunit}



The alarm, door latch, display and TIS screen will be updated after performing any
calculations. 

\begin{zed}
        TISUpdate \defs UpdateLatch \land UpdateAlarm \land UpdateDisplay \land UpdateScreen
\\ \t3 \land \Xi UserToken \land \Xi AdminToken \land \Xi Finger \land
\Xi Floppy \land 
\\ \t3  \Xi Keyboard \land \Xi Config \land
\Xi Stats  
\\ \t3 \land \Xi KeyStore \land \Xi Admin \land \Xi Internal
\\ \t3 \land (AddElementsToLog \lor \Xi AuditLog) 
\end{zed}

%......................
\subsection{Updating the user Token}
%......................

\begin{traceunit}{FS.Interface.UpdateToken}
\end{traceunit}

We have a further operation, which writes to the User Token only.
We treat this separately because we expect to update the other devices
regularly and frequently,
but we will only be updating the User Token when we have something to
write. 

\begin{schema}{UpdateUserToken}
	\Xi IDStation
\\	RealWorldChanges
\also
        \Xi TISControlledRealWorld
\where
	userToken' = currentUserToken
\end{schema}

%......................
\subsection{Updating the Floppy}
%......................

\begin{traceunit}{FS.Interface.UpdateFloppy}
\end{traceunit}


We have an operation which writes to the Floppy only.
We will only be updating the Floppy disk when we have something to write.

\begin{schema}{UpdateFloppy}
        \Delta IDStation
\\      RealWorldChanges
\also
        \Xi UserToken
\\      \Xi AdminToken
\\      \Xi Finger
\\      \Xi DoorLatchAlarm
\\      \Xi Keyboard
\\      \Xi Config
\\      \Xi Stats
\\      \Xi KeyStore
\\      \Xi Admin      
\\      \Xi AuditLog
\\      \Xi Internal
\also
	\Xi TISControlledRealWorld
\where
	floppy' = writtenFloppy
\also
        currentFloppy' = badFloppy
\\      floppyPresence' = floppyPresence
\\      currentDisplay' = currentDisplay
\\      currentScreen' = currentScreen
\end{schema}
\begin{Zcomment}
\item
Having written the floppy we can assume nothing about the $currentFloppy$
until we next poll. We do not know what data is on the floppy as it
may have been corrupted during the write. This ensures that the
readback we do is forced to be effective. 
\end{Zcomment}
