%============================================================================
\chapter{Justification of Preconditions}
\label{sec:Pre}
%============================================================================

%---------------------------------------------------------------------------
\section{Properties}
%---------------------------------------------------------------------------
We claim the following important properties of the whole system:

There is an initial state:

\begin{Zpobtrace}{FS.TIS.State.InitPOB}
\begin{theorem}
\thrm \exists InitIDStation @ true
\end{theorem}
\end{Zpobtrace}

If there is no state that satisfies the system state invariants then
other proof obligations become vacously trivial. It is therefore
important to demonstrate that an initial state exists.

The start-up operation is total.

\begin{Zpobtrace}{FS.TIS.StartUp.PreTotal}
\begin{theorem}
IDStation \thrm \pre TISStartUp
\end{theorem}
\end{Zpobtrace}

The processing operation is available whenever the system is not in a
$shutdown$ state.

\begin{Zpobtrace}{FS.TIS.Processing.PreExp}
The processing is not total, however we can show that it's precondition
is no weaker than $enclaveStatus \neq shutdown$. The internal state
$shutdown$ represents the system once it is not running so we would
expect no processing to occur under this circumstance. 

\begin{theorem}
        IDStation; RealWorld | 
\\ \t1 \lnot (enclaveStatus = shutdown \land status = quiescent) \thrm \pre TISProcessing
\end{theorem}
\end{Zpobtrace}

The polling operation is available on the assumption that the tokens
are not changed in the reader so fast that TIS misses observing the
absence of the first token before detecting the presence of the 
second token.
\label{sec:PollPre}

We need to know that while TIS is making use of a validated token the
token does not change without TIS noticing it has been removed. This
can be expressed formally as follows:

\begin{schema}{WorldChangesSlowly}
RealWorld
\\ IDStation
\where
status \in \{ gotFinger, waitingFinger, waitingUpdateToken,
waitingEntry \} \implies
\\ \t1
(userToken = 
currentUserToken 
\lor userToken = noT)
\also
rolePresent \neq \Nil \implies
(adminToken = 
currentAdminToken 
\lor adminToken = noT)
\end{schema}

\begin{Zpobtrace}{FS.TIS.Poll.PreExp}
Polling is not total, it relies on changes to the tokens in the token
reader occurring sufficiently slowly that the absence of a token is
observered before the presence of a second token is observed.
\begin{theorem}
WorldChangesSlowly \thrm \pre TISPoll
\end{theorem}
\end{Zpobtrace}

The update and early update operations are total.

\begin{Zpobtrace}{FS.TIS.EarlyUpdate.PreTotal}
\begin{theorem}
IDStation; RealWorld \thrm \pre TISEarlyUpdate
\end{theorem}
\end{Zpobtrace}

\begin{Zpobtrace}{FS.TIS.Update.PreTotal}
\begin{theorem}
IDStation; RealWorld \thrm \pre TISUpdate
\end{theorem}
\end{Zpobtrace}


%---------------------------------------------------------------------------
\section{Justifications}
%---------------------------------------------------------------------------

%.....................................................
\subsection{Justification of FS.TIS.State.InitPOB}
%.....................................................

\begin{Zpobtrace}{FS.TIS.State.InitPOB}
\begin{theorem}
\thrm \exists InitIDStation @ true
\end{theorem}
\end{Zpobtrace}

To demonstrate this it is sufficient to show that the state invariants
hold with the constraints on the initial values. We also need to
ensure that each value is in type, however by simple inspection we can
deduce this to be the case.

For all state components that are left free by the initial state
schema it is sufficient to ensure that the types are non-empty. This
is the case in all places.

We need to consider the invariants on each of the schemas that are
used to build $IDStation$ as well as the invariants on the overall
$IDStation$.

The invariants on $DoorLatchAlarm$ completely define $currentLatch$
and $currentAlarm$, the values given in $InitDoorLatchAlarm$ result in
$currentLatch = locked$ and $currentAlarm = silenced$.

The invariants on $Admin$ completely define $availableOps$, the values
given in $InitAdmin$ result in $availableOps = \emptyset$. The
remaining constraint is only applicable when $currentAdminOp \neq
\Nil$ so is true by false implication.

The invariants on $Config$ can be satisfied by the following arbitrary
choice of variable bindings. 
\[
\langle minPreservedLogSize \bind 10, alarmThresholdSize \bind 5 \rangle
\]

Considering the constraints on the $IDStation$ in turn we note that:
\begin{itemize}
\item
As $status = quiescent$ first constraint is true by false
implication.
\item
As $rolePresent = \Nil$ the second constraint is true by false
implication.
\item
As $enclaveStatus = notEnrolled$ the third and sixth constraints are true by false
implication.
\item
As $enclaveStatus = notEnrolled$ and $currentAdminOp = \Nil$ the forth
constraint reduces to $false \iff false$, which is true.
\item
As $currentAdminOp' = \Nil$ the fifth constraint is true by false implication.
\item
the final constraints define the screen elements $screenStats$ and
$screenConfig$. As $displayStats$ and $displayConfigData$ are total
functions, we can deduce that these constraints hold.
\end{itemize}

We therefore deduce that an initial state exists.

%.....................................................
\subsection{Justification of FS.TIS.StartUp.PreTotal}
%.....................................................

\begin{Zpobtrace}{FS.TIS.StartUp.PreTotal}
\begin{theorem}
IDStation \thrm \pre TISStartUp
\end{theorem}
\end{Zpobtrace}

To demonstrate this we need to show that for any initial values held
by $IDStation$ there is a binding to the variables in $IDStation'$
that preserves the state invariant.

We first note that, from properties of $\pre$ and disjunctions and the
definition of $TISStartUp$  
\[
\pre TISStartUp \equiv \pre StartEnrolledStation \lor \pre
StartNonEnrolledStation
\]
so we can reduce the problem to considering the preconditions of each
of these schemas and ensuring that their disjunction is total.

In a formal proof we would need to ensure that each schema preserves
all system and subsystem state invariants. Care has been taken in
writing this specification to ensure that operations are sufficiently
simple that the preservation of state invariants is easy to check.

We claim that 
\[
\pre StartEnrolledStation \equiv [~ IDStation | ownName \neq \Nil ~] 
\also
\pre StartNonEnrolledStation \equiv [~ IDStation | ownName = \Nil ~]
\]

The required result follows. 

We consider $\pre StartEnrolledStation$ here by way of an example of the
type of arguments that are required to deduce the preconditions of an
operation schema.

We need to determine the conditions on the initial state that
guarantee that a final state exists and this final state satisfies all
the state invariants of $IDStation'$

We note that the state components $UserToken$, $AdminToken$, $Finger$,
$Floppy$, $Keyboard$, $Config$ and $KeyStore$ are defined as not
changing. We therefore need not consider invariants that only refer to
these state components.

The invariants on $DoorLatchAlarm$ completely define $currentLatch$
and $currentAlarm$, the values given in $InitDoorLatchAlarm'$ result in
$currentLatch = locked$ and $currentAlarm = silenced$.

We note that the precondition on $AddElementsToLog$ is a requirement
that the $newElements$ are no older than the elements already in the
log. Without loss of generality we can assume the element
$startUnenrolledTISElement$ satisifies this property so $newElements =
\{ startUnenrolledTISElements \}$ is a possible solution. Hence the new
$auditLog'$ is defined by $AddElementsToLog$.

Considering the constraints on the $IDStation$ in turn we note that:
\begin{itemize}
\item
As $status' = quiescent$ first constraint is true by false
implication.
\item
As $rolePresent' = \Nil$ the second constraint is true by false
implication.
\item
As $enclaveStatus' = enclaveQuiescent$  we need to note that $ownName
\neq \Nil$ and $KeyStore$ is unchanged to deduce the the third
constraint holds.
\item
As $enclaveStatus' = enclaveQuiescent$ and $currentAdminOp' = \Nil$ the forth
constraint reduces to $false \iff false$.
\item
As $currentAdminOp' = \Nil$ the fifth constraint is true by false implication.
\item
As $enclaveStatus' = enclaveQuiescent$ the sixth constraint is true by false
implication.
\item
the final constraints define the screen elements $screenStats'$ and
$screenConfig'$. As $displayStats$ and $displayConfigData$ are total
functions so we can deduce that these constraints hold.
\end{itemize}

So the only constraint is the explicit contraint on the before state
of the $StartEnrolledStation$, namely $ownName \neq \Nil$. Giving the
result.
\[
\pre StartEnrolledStation \equiv [~ IDStation | ownName \neq \Nil ~] 
\]

%.....................................................
\subsection{Justification of FS.TIS.Processing.PreExp}
%.....................................................

\begin{Zpobtrace}{FS.TIS.Processing.PreExp}
The processing operation is not total, however we can show that it's precondition
is no weaker than $enclaveStatus \neq shutdown$. The internal state
$shutdown$ represents the system once it is not running so we would
expect processing not to occur under this circumstance. 

\begin{theorem}
        IDStation; RealWorld | 
\\ \t1 \lnot (enclaveStatus = shutdown \land status = quiescent) \thrm \pre TISProcessing
\end{theorem}
\end{Zpobtrace}

To prove this we rely heavily on the following property:

For any schemas $S$ and $T$, $\pre$ distributes through disjunction
\begin{argue}
\pre(S \lor T) \equiv (\pre S) \lor (\pre T)
\end{argue}

Expanding the definition of $TISProcessing$.

\begin{argue}
\pre TISProcessing
\\  \t1 \equiv 
 \pre ((TISEnrolOp \lor TISUserEntryOp \lor TISAdminLogon 
\\ \t4                  \lor TISStartAdminOp \lor  TISAdminOp \lor
TISAdminLogout \lor TISIdle) \land LogChange) 
\end{argue}

We should at this point distribute the $LogChange$ through the
disjunction and consider the precondition of $TISEnrolOp \land
LogChange$ etc. However it transpires that $LogChange$ does not add
any constraints to the preconditions of each of these operations, it
only modifies the AuditLog and all operations on the audit log are
sufficiently free to allow these modifications in all circumstances.

So we consider the preconditions of each of the components of the
operations. We decompose $TISUserEntryOp$ (which is constructed as a
disjunction) and consider the
preconditions of each of the components:

\begin{argue}
        \pre TISReadUserToken \equiv 
\\      \t2 [~IDStation; RealWorld |
    status = quiescent \land userTokenPresence = present 
\\ \t3          \land enclaveStatus \in 
                \{ enclaveQuiescent, waitingRemoveAdminTokenFail \} ~]
\also
        \pre TISValidateUserToken \equiv 
%\\      \t2
 [~IDStation; RealWorld | status = gotUserToken  ~]
\also
        \pre TISReadFinger \equiv 
%\\      \t2 
[~IDStation; RealWorld | status = waitingFinger  ~]
\also
        \pre TISValidateFinger \equiv 
%\\      \t2 
[~IDStation; RealWorld | status = gotFinger  ~]
\also
        \pre TISWriteUserToken \equiv 
%\\      \t2 
[~IDStation; RealWorld | status = waitingUpdateToken  ~]
\also
        \pre TISValidateEntry \equiv 
%\\      \t2 
[~IDStation; RealWorld | status = waitingEntry  ~]
\also
        \pre TISUnlockDoor \equiv 
%\\      \t2 
[~IDStation; RealWorld | status = waitingRemoveTokenSuccess  ~]
\also
        \pre TISCompleteFailedAccess \equiv 
%\\      \t2 
[~IDStation; RealWorld | status = waitingRemoveTokenFail  ~]
\end{argue}

We decompose $TISAdminOp$ and consider the preconditions of each of
the components.

\begin{argue}
        \pre TISStartArchiveLog \equiv 
\\      \t2 [~IDStation; RealWorld | enclaveStatus =
waitingStartAdminOp 
\\      \t3 \land currentAdminOp \neq \Nil \land currentAdminOp = archiveLog ~]
\also
        \pre TISFinishArchiveLog \equiv 
\\      \t2 [~IDStation; RealWorld | enclaveStatus =
waitingFinishAdminOp 
\\      \t3 \land currentAdminOp \neq \Nil \land currentAdminOp = archiveLog ~]
\also
        \pre TISStartUpdateConfigData \equiv 
\\      \t2 [~IDStation; RealWorld | enclaveStatus =
waitingStartAdminOp 
\\      \t3 \land currentAdminOp \neq \Nil \land currentAdminOp = updateConfigData ~]
\also
        \pre TISFinishUpdateConfigData \equiv 
\\      \t2 [~IDStation; RealWorld | enclaveStatus =
waitingFinishAdminOp 
\\      \t3 \land currentAdminOp \neq \Nil \land currentAdminOp = updateConfigData ~]
\also
        \pre TISShutdownOp \equiv 
\\      \t2 [~IDStation; RealWorld | enclaveStatus =
waitingStartAdminOp 
\\      \t3 \land currentAdminOp \neq \Nil \land currentAdminOp = shutdownOp ~]
\also
        \pre TISOverrideDoorLockOp \equiv 
\\      \t2 [~IDStation; RealWorld | enclaveStatus =
waitingStartAdminOp 
\\      \t3 \land currentAdminOp \neq \Nil \land currentAdminOp = overrideLock ~]
\also
\end{argue}
From these and the $IDStation$ invariants:
\[
        enclaveStatus \in \{~ waitingStartAdminOp, waitingFinishAdminOp ~\} \iff currentAdminOp \neq \Nil
\also
       (currentAdminOp \neq \Nil \land \The currentAdminOp \in \{~
shutdownOp, overrideLock ~\}) 
\\ \t2          \implies enclaveStatus = waitingStartAdminOp

\]
we can deduce
\begin{argue}
        \pre TISAdminOp \equiv 
\\      \t2 [~IDStation; RealWorld |
 enclaveStatus \in \{~
waitingStartAdminOp, waitingFinishAdminOp ~\} ~]
\end{argue}

We decompose $TISEnrol$ and consider the preconditions of each of
the components.

\begin{argue}
        \pre ReadEnrolmentData \equiv 
         [~IDStation; RealWorld | enclaveStatus = notEnrolled  ~]
\also
        \pre ValidateEnrolmentData \equiv 
         [~IDStation; RealWorld | enclaveStatus = waitingEnrol  ~]
\also
        \pre CompleteFailedEnrolment \equiv 
         [~IDStation; RealWorld | enclaveStatus = waitingEndEnrol  ~]
\end{argue}

We decompose $TISAdminLogon$ and consider the preconditions of each of
the components.

\begin{argue}
        \pre TISReadAdminToken \equiv 
\\      \t2 [~IDStation; RealWorld |
\\      \t3      enclaveStatus = enclaveQuiescent \land adminTokenPresence = present 
\\ \t3          \land status \in \{ quiescent, waitingRemoveTokenFail \} ~]
\also
        \pre TISValidateAdminToken \equiv 
         [~IDStation; RealWorld | enclaveStatus = gotAdminToken  ~]
\also
        \pre TISCompleteFailedAdminLogon \equiv 
\\      \t2 [~IDStation; RealWorld | enclaveStatus = waitingRemoveAdminTokenFail  ~]
\end{argue}
We decompose $TISAdminLogout$ and consider the preconditions of each of
the components.
\begin{argue}
        \pre TokenRemovedAdminLogout \equiv 
\\      \t2 [~IDStation; RealWorld | rolePresent \neq \Nil
\\      \t3     \land enclaveStatus = enclaveQuiescent \land adminTokenPresence = abscent 
         ~]
\also
        \pre AdminTokenTimeout \equiv 
\\      \t2 [~IDStation; RealWorld | rolePresent \neq \Nil
\\      \t3     \land enclaveStatus = enclaveQuiescent \land
adminTokenPresence = present \land \lnot AdminTokenOK 
         ~]
\also
        \pre TISCompleteTimeoutAdminLogout \equiv 
\\      \t2 [~IDStation; RealWorld |  enclaveStatus = waitingRemoveAdminTokenFail  ~]
\end{argue}


The remaining operations are $TISStartAdminOp$, $TISAdminLogout$ and
$TISIdle$:
\begin{argue}
        \pre TISStartAdminOp \equiv 
\\      \t2 [~IDStation; RealWorld | rolePresent \neq \Nil
\\      \t3      \land enclaveStatus = enclaveQuiescent \land adminTokenPresence = present 
\\ \t3          \land status \in \{ quiescent, waitingRemoveTokenFail \} ~]
\also
        \pre TISIdle \equiv 
\\      \t2 [~IDStation; RealWorld |  rolePresent = \Nil
\\      \t3     \land enclaveStatus = enclaveQuiescent \land
adminTokenPresence = absent 
\\ \t3          \land status = quiescent, \land userTokenPresence =
absent ~]
\end{argue}

By considering properties of conjunction and disjunction we can deduce
\begin{argue}
\pre (TISStartAdminOp \lor TISAdminLogout \lor TISIdle 
 \\ \t4         \lor TISReadUserToken \lor TISReadAdminToken) 
\\ \t1 \equiv
 [~ IDStation; RealWorld | 
\\      \t3 (status = quiescent \land enclaveStatus = enclaveQuiescent ) 
\\      \t3 \lor (status = waitingRemoveTokenFail \land enclaveStatus =
enclaveQuiescent 
\\ \t4 \land adminTokenPresence = present ) 
\\      \t3 \lor (status = quiescent \land enclaveStatus =
waitingRemoveAdminTokenFail 
\\ \t4 \land userTokenPresence = present ) 
\\      \t3 \lor (enclaveStatus = enclaveQuiescent \land rolePresent
\neq \Nil 
\\ \t4  \land adminTokenPresence = absent)
\\      \t3 \lor (enclaveStatus = enclaveQueiscent \land rolePresent
\neq \Nil 
\\ \t4  \land adminTokenPresence = present \land \lnot AdminTokenOK ~]    
\end{argue}

By considering the coverage of the values of $status$ and
$enclaveStatus$ we can deduce
\begin{argue}
 \pre (TISEnrolOp \lor TISUserEntryOp \lor TISAdminLogon 
\\ \t3                  \lor TISStartAdminOp \lor  TISAdminOp \lor
TISAdminLogout \lor TISIdle) 
\\ \t1  \equiv [~ IDStation; RealWorld | 
\\      \t4 (status = quiescent \land enclaveStatus = enclaveQuiescent ) 
\\      \t4 \lor (status \neq quiescent) 
\\      \t4 \lor (enclaveStatus \neq enclaveQuiescent \land enclaveStatus \neq
shutdown) ~]
\end{argue}
rearranging this gives:
\begin{argue}
 \pre (TISEnrolOp \lor TISUserEntryOp \lor TISAdminLogon 
\\ \t3                  \lor TISStartAdminOp \lor  TISAdminOp \lor
TISAdminLogout \lor TISIdle 
\\ \t1 \equiv [~ IDStation; RealWorld |  status \neq quiescent  \lor  enclaveStatus \neq
shutdown ~]
\end{argue}

We have already argued that $LogChange$ does not effect the
precondition, and hence the required result follows.

%.....................................................
\subsection{Justification of FS.TIS.Poll.PreExp}
%.....................................................

\begin{Zpobtrace}{FS.TIS.Poll.PreExp}
Polling is not total, it relies on changes to the tokens in the token
reader occurring sufficiently slowly that the absence of a token is
observered before the presence of a second token is observed.
\begin{theorem}
WorldChangesSlowly \thrm \pre TISPoll
\end{theorem}
\end{Zpobtrace}

To demonstrate this we notice that there are no constraints on the initial
state in the $TISPoll$ schema. 
It is thus sufficient to check that all system invariants are
maintained. 
The only invariants that need be checked are those that
involve entities that change. There are two of these:

Firstly
\[ 
	status \in \{~ gotFinger, waitingFinger, waitingUpdateToken, waitingEntry~\} \implies
\\ \t1		 (( \exists ValidToken @ 
			goodT(\theta ValidToken) = currentUserToken' )
\\ \t2  \lor ( \exists TokenWithValidAuth @ 
			goodT(\theta TokenWithValidAuth) = currentUserToken'))
\]
This holds under the assumptions of $WorldChangesSlowly$ since by the
definition of $PollUserToken$ we can deduce $currentUserToken' =
currentUserToken$ under these conditions.

The second invariant
\[
        rolePresent \neq \Nil \implies       
\\ \t1		 ( \exists TokenWithValidAuth @ 
			goodT(\theta TokenWithValidAuth) =
			currentAdminToken' )
\]
Again, this holds under the assumptions of $WorldChangesSlowly$ since
by the definition of $PollAdminToken$ we can deduce
$currentAdminToken' = currentAdminToken$ under these conditions. 


%.....................................................
\subsection{Justification of FS.TIS.EarlyUpdate.PreTotal}
%.....................................................

\begin{Zpobtrace}{FS.TIS.EarlyUpdate.PreTotal}
\begin{theorem}
IDStation; RealWorld \thrm \pre TISEarlyUpdate
\end{theorem}
\end{Zpobtrace}

To demonstrate this we notice that there are no constraints on the initial
state in the $TISEarlyUpdate$ schema. 
It is thus sufficient to check that all system invariants are
maintained. 
The only invariants that need be checked are those that
involve entities that change, as there are no invariants involving
state changed by this operation we can deduce that the operation is total.


%.....................................................
\subsection{Justification of FS.TIS.Update.PreTotal}
%.....................................................

\begin{Zpobtrace}{FS.TIS.Update.PreTotal}
\begin{theorem}
IDStation; RealWorld \thrm \pre TISUpdate
\end{theorem}
\end{Zpobtrace}

To demonstrate this we notice that there are no constraints on the initial
state in the $TISUpdate$ schema. 
It is thus sufficient to check that all system invariants are
maintained. 
The only invariants that need be checked are those that
involve entities that change, as there are no invariants involving
state changed by this operation we can deduce that the operation is total.


