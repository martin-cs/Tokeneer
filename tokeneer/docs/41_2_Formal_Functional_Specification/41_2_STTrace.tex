\chapter{Tracing of ST Requirements}
\label{sec:STTrace}
\section{Mapping of Functional Security Requirements}
\label{sec:mapST}
This section justifies the manner in which this specification
satisifies the security requirements presented in the Security Target
\cite{ST}. 


{\footnotesize \sf 
SFP.DAC (new table):
} 
\newline
	$Admin$ invariants define available Ops for each
role. $ValidateOpRequestOK$ checks the requested op against the
role. $ValidateAdminTokenOK$ logs the user on based on their
Authorisation Certificate. 
Viewing is controlled by $UpdateScreen$, which defines the information
displayed on the screen. It displays configuration data only if the
security officer is present. System statistics are deamed non-secure,
and are displayed when any administrator is logged on.

{\footnotesize \sf 
FAU\_ARP.1.1:
} 
\newline
	Defined class of alarms are: door
	open, latched, too long; audit files truncating with loss. 
\newline
	$DoorLatchAlarm$ invariant defines the internal decision to
	alarm the door. This is effected in the real world with
	$UpdateAlarm$, done twice on each main loop. 
\newline
	$AddElementsToLog$ and $ClearLog$ modify the audit alarm
	internally.  This is effected in the real world with
	$UpdateAlarm$, done twice on each main loop. 
\newline
	Alarming is audited in $AuditAlarm$ and $AuditLogAlarm$, part of
	$LogChange$. invoked once every main loop. 

{\footnotesize \sf 
FAU\_GEN.1.1:
}
\newline
	Every significant schema audits. A check will need to be done
	at the code level that the actual collection of events is as
	desired. 
	Confirm necessary list of audit events only once the other
	exclusions are in place. 

{\footnotesize \sf 
FAU\_GEN.1.2:
}
\newline
	Free type Audit is not elaborated in this specification. Will
	need to be checked when elaborated in the design. 

{\footnotesize \sf 
FAU\_GEN.2.1:
}
\newline
	See FAU\_GEN.1.2

{\footnotesize \sf 
FAU\_SAA.1.1:
}
\newline
	Alarming is the mechanism for responding immediately to
	audited events, and so this requirement is met by the
	FAU\_ARP.1.1 above. 

{\footnotesize \sf 
FAU\_SAA.1.2:
}
\newline
	See FAU\_SAA.1.1

{\footnotesize \sf 
FAU\_STG.2.3:
}
\newline
	$AddElementsToLog$

{\footnotesize \sf 
FAU\_STG.4.1:
}
\newline
	$AddElementsToLog$

{\footnotesize \sf 
FCO\_NRO.2.1:
}
\newline
	$ValidToken$ defines some of the checks on all the
	certificates, $TokenWithValidAuthCert$ including the
	Authorisation Certificate. This is invoked in $UserTokenOK$
	and $UserTokenWithOKAuthCert$, which add the remaining
	checks. 

{\footnotesize \sf 
FCO\_NRO.2.2:
}
\newline
	See FCO\_NRO.2.1

{\footnotesize \sf 
FCO\_NRO.2.3:
}
\newline
	See FCO\_NRO.2.1

{\footnotesize \sf 
FDP\_ACC.1.1:
}
\newline
	Access to system objects is controlled by giving admin users
	roles, and controlling which operations they have access
	to. This is covered in SFP\_DAC. Access to user objects
	(certificates on tokens) is restricted by the basic design of
	the system: certificates are read and validated during user
	entry ($TISUserEntryOp$) for a single user at a time, and then
	discarded. There is no opportunity for any other user or
	administrator to access these objects. 

{\footnotesize \sf 
FDP\_ACF.1.1:
}
\newline
	see FDP\_ACC.1.1

{\footnotesize \sf 
FDP\_ACF.1.2:
}
\newline
	see FDP\_ACC.1.1

{\footnotesize \sf 
FDP\_ACF.1.3:
}
\newline
	see FDP\_ACC.1.1

{\footnotesize \sf 
FDP\_ACF.1.4:
}
\newline
	see FDP\_ACC.1.1
	
{\footnotesize \sf 
FDP\_DAU.2.1:
}
\newline
	This refers to ensuring validity of the users' credentials,
	i.e. the cryptographic processes that ensure the certificates
	are signed correctly. This is therefore covered in
	$ValidToken$, and $TokenWithValidAuthCert$
	which define some of the checks on all the certificates,
	including the Authorisation Certificate, and the invoking
	$UserTokenOK$ and $UserTokenWithOKAuthCert$, which add the
	remaining checks. Ensuring the Authorisation Certificate is
	valid is carried out in $TISWriteUserToken$. But note that as
	the cryto is simulated, this validity checking and guarantee
	is not real. 

{\footnotesize \sf 
FDP\_DAU.2.2:
}
\newline
	See FDP\_DAU.2.1

{\footnotesize \sf 
FDP\_RIP.2.1:
}
\newline
	Not really a functional requirement. But the specification
	overall does not provide any function to access the previous
	information in a resource. This requirement is deferred.

{\footnotesize \sf 
FDP\_UIT.1.1:
}
\newline
	Authorisation Certificates are created by $NewAuthCert$, which
	ensures they can be validated by the key of the ID Station. 

{\footnotesize \sf 
FDP\_UIT.1.2:
}
\newline
	See FDP\_UIT.1.1.


{\footnotesize \sf 
FIA\_UAU.2.1:
}
\newline
	This is the basic operation of the TIS, and is implemented by
	the whole behaviour - not permitting user entry until
	authorisation and not letting administrators carry out
	operations until they have logged on. This cannot be mapped to
	any specific part of the specification, but the property is
	captured in the formal security policy, and the proof follows
	from there. 

{\footnotesize \sf 
FIA\_UAU.3.1:
}
\newline
	Only detection of forgery of token information is in
	scope. This maps to $ValidToken$ and the cryptographic check
	implied by this. 

{\footnotesize \sf 
FIA\_UAU.3.2:
}
\newline
	The design choice of using biometric data means that copying
	the token is not an effective attack, because it is not
	possible to copy the other person's actual biometric object
	(e.g. their finger). Copying the authorisation certificate
	without copying the token is prevented by $ValidToken$, which
	checks the token ID in the certificate is the same as on the
	Token, and $NewAuthCert$, which constructs auth certs with the
	correct token ID in them. This does depend on the token ID
	being unique. 

{\footnotesize \sf 
FIA\_UAU.7.1:
}
\newline
	The series of updates to the display given in the
	$TISUserEntryOp$ operation presents information to the user only
	at major milestones in the authentication process, and so
	leaks no useful information to the user. 

{\footnotesize \sf 
FIA\_UID.2.1:
}
\newline
	The first action a user must carry out is to present their
	token, from which the certificates are extracted
	($ReadUserToken$ and $ReadAdminToken$). These certificates
	identify the user to the system before they can do anything
	else. 

{\footnotesize \sf 
FIA\_USB.1.1:
}
\newline
	For user entry, no operation is possible other than to unlock
	the door, which is done only if the user is allowed entry, so
	this requirement does not really apply. For administrators,
	the actions allowed to a user are based on their role, as
	covered by SFP.DAC. 

{\footnotesize \sf 
FMT\_MOF.1.1:
}
\newline
	This table has been modified by the ST, making it clear that
	only configuration data will be modifiable, and then only by
	the security officer. The allocation of operations to the
	security officer is covered in the Admin invariant, which
	gives him (and only him) only
	$updateConfigData$. $ValidateOpRequest$ will therefore limit him
	to modification of the configuration data. 

{\footnotesize \sf 
FMT\_MSA.1.1:
}
\newline
	See SFP.DAC.

{\footnotesize \sf 
FMT\_MSA.2.1:
}
\newline
	Configuration data and enrolment data, the only security
	information, is entered via a floppy. The definition of the
	possible content of a floppy includes 
	$enrolmentFile \ldata ValidEnrol \rdata$, $configFile \ldata
	Config \rdata$ and $badFloppy$. The invariants on
	$ValidEnrol$ and $Config$ ensure that if the floppy is regarded as
	having enrolment or configuration data, it will be correctly
	constructed (secure). If the data is not correct, the floppy
	will be regarded as having $badFloppy$ data, and $EnrolmentDataOK$
	and $FinishUpdateConfigDataOK$ will not read it. (Invariants
	will need to be defined for all security values in $Config$
	and $ValidEnrol$ in discussion with the client).

{\footnotesize \sf 
FMT\_MSA.3.1:
}
\newline
	InitIDStation

{\footnotesize \sf 
FMT\_MSA.3.2:
}
\newline
	No information is created by the Security Officer, and so this requirement does not need to be implemented.

{\footnotesize \sf 
FMT\_MTD.1.1:
}
\newline
	See SFP.DAC

{\footnotesize \sf 
FMT\_MTD.3.1:
}
\newline
	See FMT\_MSA.2.1

{\footnotesize \sf 
FMT\_SAE.1.1:
}
\newline
        See SFP.DAC.

{\footnotesize \sf 
FMT\_SAE.1.2:
}
\newline
	Not implemented. Behaviourally, whether the authorisation cert is there and out of validity, or not there, is invisible to the user.

{\footnotesize \sf 
FMT\_SMR.2.1:
}
\newline
	See SFP.DAC

{\footnotesize \sf 
FMT\_SMR.2.2:
}
\newline
	$ValidateAdminTokenOK$ associates a user with the role given to them in their authorisation certificate.

{\footnotesize \sf 
FMT\_SMR.3.1:
}
\newline
	The Administrator must insert their admin token and have it read by the system ($ReadAdminTokenOK$) before they are given any admin role.

{\footnotesize \sf 
FPT\_FLS.1.1:
}
\newline
	Power failure or crash will require the system to be re-started. There is no persistent state, and a re-start will require the system to be re-started in a secure state, as with any start-up.

{\footnotesize \sf 
FPT\_RVM.1.1:
}
\newline
	The invariant in $DoorLatchAlarm$ ensures that the latch never
	remains unlocked beyond the time set by the TIS. It also
	ensures that the alarm will sound if the door remains open for
	too long. 
	The variables $status$ and $enclaveStatus$ act as state-machine
	controls, and ensure that each operation can only be performed
	when the system is in the correct state. 

{\footnotesize \sf 
FPT\_STM.1.1:
}
\newline
	$PollTime$ reads the time from an external source (assumed
	reliable). This is done once per main loop. 

{\footnotesize \sf 
FRU\_PRS.1.1:
}
\newline
	The variables $status$ and $enclaveStatus$ act as state-machine
	controls and ensure that once User Entry or Admin validation
	have started, they continue to appropriate points, ignoring
	other accesses. This in effect assures that an action in hand
	has priority over new actions - new subjects have lower
	priority to old subjects. 

{\footnotesize \sf 
FRU\_PRS.1.2:
}
\newline
	See FRU\_PRS.1.1




\section{Requirements out of scope}
This section lists the requirements that are not referenced from this
document, with a justification for their omission.

\subsection{Deferred}

The following requirements have been deferred to the design due to their
detailed nature.

\begin{traceunit}{FS.NotInScope.Deferred}
\traceto{FAU\_GEN.1.2}
\traceto{FAU\_GEN.2.1}
\traceto{FDP\_RIP.2.1}

\end{traceunit}%Deferred

\subsection{NotImplemented}

The following requirements have not been implemented, the
justification for their non-implementation is covered in section \ref{sec:mapST}.

\begin{traceunit}{FS.NotInScope.STNotImplemented}
\traceto{FMT\_SAE.1.2}
\end{traceunit}%Deferred

%------------------------------------------------------------------------
\section{General Requirements}
%------------------------------------------------------------------------
Several of the security requirements are of a general nature and demonstration
of their satisfaction by this specification requires analysis of the
whole specification rather than reference to a single (or small number
of) operation schemas. The justification of satisfaction of each of
these requirements is presented in section \ref{sec:mapST}.

\begin{traceunit}{FS.General.STRequirements}
\traceto{FRU\_PRS.1.1}
\traceto{FRU\_PRS.1.2}
\end{traceunit}%STRequirements








